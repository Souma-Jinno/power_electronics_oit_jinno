% 第2章 パワー半導体の動作原理

\chapter{パワー半導体の動作原理}

\section{はじめに}

\subsection{本章の目的と学習目標}

本章では、パワーエレクトロニクスで使用される半導体スイッチング素子の動作原理について学びます。前章で学んだ半導体の物理的性質を基礎として、実際のパワー半導体素子がどのように動作するのかを理解することが目標です。

本章の学習目標は以下の通りです:

\begin{enumerate}
\item スイッチング素子の物理を理解する
\item 半導体の構造からスイッチの制御の原理を理解する
\item ワイドギャップ半導体の特徴を理解する
\end{enumerate}

\begin{figure}[H]
\centering
\fbox{\includegraphics[width=0.95\textwidth]{chapters/chapter03/images/page-02.pdf}}
\caption{本日の目標}
\label{fig:objectives}
\end{figure}

\subsection{本章で学ぶスイッチの種類}

パワーエレクトロニクスで使用される主要なスイッチング素子には、以下のようなものがあります:

\begin{itemize}
\item \textbf{ダイオード}:最も基本的な半導体素子で、一方向にのみ電流を流す
\item \textbf{サイリスタ}:制御可能なスイッチング素子
\item \textbf{トランジスタ}:電流または電圧で制御できるスイッチング素子
\item \textbf{IGBT}:絶縁ゲートバイポーラトランジスタ
\item \textbf{MOSFET}:金属酸化膜半導体電界効果トランジスタ
\end{itemize}

\begin{figure}[H]
\centering
\fbox{\includegraphics[width=0.95\textwidth]{chapters/chapter03/images/page-03.pdf}}
\caption{本日習うスイッチの種類}
\label{fig:switch_types}
\end{figure}

これらの素子は、それぞれ異なる特性を持ち、用途に応じて使い分けられます。

\section{半導体スイッチの応用先}

\subsection{用途に応じたスイッチの使い分け}

半導体スイッチは、その出力容量と動作周波数によって使い分けられます。図\ref{fig:applications}に示すように、サイリスタは電鉄や製鉄などの大容量・低周波数の用途に、IGBTは電力機器や自動車などの中容量・中周波数の用途に、MOSFETは電子レンジや洗濯機などの小容量・高周波数の用途に使用されます。

\begin{figure}[H]
\centering
\fbox{\includegraphics[width=0.95\textwidth]{chapters/chapter03/images/page-04.pdf}}
\caption{半導体スイッチの応用先}
\label{fig:applications}
\end{figure}

近年では、SiC(シリコンカーバイド)やGaN(窒化ガリウム)などのワイドギャップ半導体の応用により、さらなる大容量化や高速化が進んでいます。これにより、従来は困難だった領域への応用が可能になっています。

\section{パワー半導体スイッチに求められる機能}

\subsection{電力変換用途での要求特性}

パワー半導体スイッチは、電力変換の用途で使用されるため、以下の特性が求められます:

\begin{enumerate}
\item \textbf{高耐圧}(オフ時):スイッチがオフの状態では、高電圧がかかっても破壊されない耐圧性が必要
\item \textbf{低抵抗}(オン時):スイッチがオンの状態では、大電流が流れても損失が小さい低抵抗性が必要
\end{enumerate}

\begin{figure}[H]
\centering
\fbox{\includegraphics[width=0.95\textwidth]{chapters/chapter03/images/page-05.pdf}}
\caption{パワー半導体スイッチに求められる機能}
\label{fig:requirements}
\end{figure}

オン時の電力損失は、電流$I$とオン抵抗$R_{\text{on}}$の関係から$P = IR_{\text{on}}^2$で表されます。したがって、オン抵抗を小さくすることが、効率の良いスイッチング素子を実現する上で重要です。

\subsection{パワー半導体の耐圧と損失のトレードオフ}

パワー半導体素子の設計においては、耐圧と損失の間にトレードオフの関係があります。図\ref{fig:tradeoff}に示すように、pn接合の耐圧を上げるためにはドーピング濃度を小さくする必要がありますが、一方でドリフト電流による損失を小さくするためにはドーピング濃度を大きくする必要があります。

\begin{figure}[H]
\centering
\fbox{\includegraphics[width=0.95\textwidth]{chapters/chapter03/images/page-06.pdf}}
\caption{パワー半導体の耐圧と損失のトレードオフ}
\label{fig:tradeoff}
\end{figure}

最大電界$F_{\text{max}}$は、空乏層幅$d$と耐圧$V$の関係から以下の式で表されます:

\begin{equation}
F_{\text{max}} = \frac{2V}{d}
\end{equation}

また、空乏層幅$d$は以下の式で計算されます:

\begin{equation}
d = \sqrt{\frac{2\varepsilon V}{e}\left(\frac{1}{N_a} + \frac{1}{N_d}\right)}
\end{equation}

ここで、$\varepsilon$は誘電率、$e$は電荷素量、$N_a$はアクセプタ濃度、$N_d$はドナー濃度です。

右側のグラフは、不純物濃度と抵抗率の関係を示しています。n型半導体とp型半導体では、不純物濃度が増加すると抵抗率が減少することがわかります。

\section{スイッチの制御について}

\subsection{制御可能性による分類}

半導体スイッチは、その制御方法によって分類できます。図\ref{fig:control}に示すように、スイッチの制御には以下の2つのタイプがあります:

\begin{enumerate}
\item \textbf{好きなときに切り替え可能}:電圧または電流で制御できるスイッチ(例:トランジスタ、MOSFET、IGBT)
\item \textbf{ある条件で切り替わる}:特定の条件下でのみオン・オフが切り替わるスイッチ(例:ダイオード)
\end{enumerate}

\begin{figure}[H]
\centering
\fbox{\includegraphics[width=0.95\textwidth]{chapters/chapter03/images/page-07.pdf}}
\caption{スイッチの制御について}
\label{fig:control}
\end{figure}

制御可能なスイッチは、ゲート電圧$v_g$やベース電流$i_g$によって、主電流$i$を制御することができます。

\begin{figure}[H]
\centering
\fbox{\includegraphics[width=0.95\textwidth]{chapters/chapter03/images/page-08.pdf}}
\caption{本日習うスイッチの種類(回路記号)}
\label{fig:switch_symbols}
\end{figure}

\section{ダイオードの動作原理}

\subsection{ダイオードの基本構造}

ダイオードは、pn接合を基本とした最もシンプルな半導体素子です。p型半導体とn型半導体を接合することで、一方向にのみ電流を流す整流作用を実現します。ダイオードは制御端子を持たず、印加される電圧によって自動的にオン・オフが切り替わります。

\subsection{ダイオードのスイッチ特性}

図\ref{fig:diode_characteristics}に、ダイオードの電圧-電流特性を示します。

\begin{figure}[H]
\centering
\fbox{\includegraphics[width=0.95\textwidth]{chapters/chapter03/images/page-09.pdf}}
\caption{ダイオードのスイッチ特性}
\label{fig:diode_characteristics}
\end{figure}

ダイオードの特性は以下のようになります:

\begin{itemize}
\item \textbf{順方向バイアス時}(アノードがカソードより高電位):
\begin{itemize}
\item pn接合の拡散電位(約0.7 V)を超えると電流が流れ始める
\item オン電圧$v_{\text{fwd}} \approx 0.7$ V(シリコンダイオードの場合)
\item 実際のダイオードでは、理想的なゼロ抵抗ではなく、わずかな電圧降下が存在する
\end{itemize}

\item \textbf{逆方向バイアス時}(カソードがアノードより高電位):
\begin{itemize}
\item 空乏層が広がり、電流はほとんど流れない
\item 逆方向電圧が破壊電圧を超えると、絶縁が破れて電流が流れる(破壊)
\end{itemize}
\end{itemize}

右側のグラフは、実際のダイオード(U10LC48)の電圧-電流特性を示しています。温度が高くなると(150℃)、順方向電圧が小さくなることがわかります。これは、温度上昇により真性キャリア濃度が増加するためです。

\subsection{電磁気学的観点からのダイオードの動作原理}

ダイオードの整流作用を深く理解するためには、外部電圧印加時のバンドの曲がりと電位分布を電磁気学的に理解する必要があります。

\subsubsection{熱平衡状態(外部電圧なし)}

第2章で学んだように、pn接合を形成すると、熱平衡状態では以下の状態になります:

\begin{itemize}
\item 拡散により空乏層が形成される
\item n型側に正の固定電荷($+eN_d$)、p型側に負の固定電荷($-eN_a$)が存在
\item 空乏層内に電場$\mathcal{E}$が形成される(n型→p型の向き)
\item 拡散電位$V_{bi} \approx 0.7$ V(シリコンの場合)
\item バンドはn型側が低く、p型側が高い(エネルギー差$eV_{bi}$)
\item フェルミ準位$E_F$は全体で一定(水平)
\end{itemize}

この状態では、拡散電流とドリフト電流が釣り合い、正味の電流はゼロです。

\subsubsection{順方向バイアス時のバンドの変化}

p型側(アノード)に正電圧$V_F$を、n型側(カソード)に負電圧を印加すると、以下の変化が起こります:

\textbf{ステップ1:外部電圧による電位分布の変化}

\begin{itemize}
\item p型側の電位が$+V_F$だけ上昇
\item n型側の電位は基準(グラウンド)
\item 空乏層にかかる電位差:$V_{bi} - V_F$に\textbf{減少}
\end{itemize}

\textbf{ステップ2:バンドの曲がりの変化}

電子のエネルギーは電位と逆の関係($E = -e\phi$)にあるため:

\begin{itemize}
\item p型側の電位上昇 → p型側のバンド端が\textbf{下がる}
\item 空乏層でのバンドの曲がり(傾き)が\textbf{緩やかになる}
\item エネルギー障壁の高さ:$eV_{bi}$から$e(V_{bi} - V_F)$に\textbf{減少}
\end{itemize}

\textbf{ステップ3:キャリアの拡散電流の増加}

エネルギー障壁が低くなることで:

\begin{enumerate}
\item n型側の電子がp型側へ拡散しやすくなる(障壁が低い)
\item p型側の正孔がn型側へ拡散しやすくなる(障壁が低い)
\item 拡散電流が\textbf{ドリフト電流を上回る}
\item 正味の電流が流れる
\end{enumerate}

拡散電流は、エネルギー障壁の高さに指数関数的に依存します:

\begin{equation}
I = I_0 \left( e^{\frac{eV_F}{kT}} - 1 \right)
\end{equation}

ここで、$I_0$は逆方向飽和電流、$k$はボルツマン定数、$T$は絶対温度です。

\textbf{ステップ4:電流の流れ}

\begin{itemize}
\item 電子:n型→p型へ拡散(電流の向きはp型→n型)
\item 正孔:p型→n型へ拡散(電流の向きはp型→n型)
\item 両者の合計が順方向電流となる
\end{itemize}

\subsubsection{逆方向バイアス時のバンドの変化}

n型側(カソード)に正電圧$V_R$を、p型側(アノード)に負電圧を印加すると、以下の変化が起こります:

\textbf{ステップ1:外部電圧による電位分布の変化}

\begin{itemize}
\item n型側の電位が$+V_R$だけ上昇
\item p型側の電位は基準(グラウンド)
\item 空乏層にかかる電位差:$V_{bi} + V_R$に\textbf{増加}
\end{itemize}

\textbf{ステップ2:バンドの曲がりの変化}

\begin{itemize}
\item n型側の電位上昇 → n型側のバンド端が\textbf{下がる}
\item 空乏層でのバンドの曲がり(傾き)が\textbf{急になる}
\item エネルギー障壁の高さ:$eV_{bi}$から$e(V_{bi} + V_R)$に\textbf{増加}
\end{itemize}

\textbf{ステップ3:キャリアの拡散の抑制}

エネルギー障壁が高くなることで:

\begin{enumerate}
\item n型側の電子がp型側へ拡散できない(障壁が高すぎる)
\item p型側の正孔がn型側へ拡散できない(障壁が高すぎる)
\item 拡散電流がほぼゼロになる
\item ごくわずかな逆方向飽和電流$I_0$のみが流れる
\end{enumerate}

\textbf{ステップ4:空乏層の拡大}

\begin{itemize}
\item 空乏層幅$d$が増加:$d \propto \sqrt{V_{bi} + V_R}$
\item 空乏層内の電場$\mathcal{E}$が増加
\item 静電容量$C$が減少:$C \propto 1/d$
\end{itemize}

\subsubsection{バンド図と電位・電場の対応関係}

以下の表に、ダイオードの各状態における電磁気学的な量の対応関係をまとめます:

\begin{table}[H]
\centering
\caption{ダイオードの各バイアス状態における物理量の比較}
\begin{tabular}{|l|c|c|c|}
\hline
\textbf{物理量} & \textbf{熱平衡} & \textbf{順バイアス} & \textbf{逆バイアス} \\
\hline
空乏層の電位差 & $V_{bi}$ & $V_{bi} - V_F$ & $V_{bi} + V_R$ \\
\hline
エネルギー障壁 & $eV_{bi}$ & $e(V_{bi} - V_F)$ & $e(V_{bi} + V_R)$ \\
\hline
バンドの曲がり & 中間 & 緩やか & 急 \\
\hline
空乏層幅 $d$ & 基準 & 減少 & 増加 \\
\hline
拡散電流 & ドリフト電流と平衡 & 増大 & ほぼゼロ \\
\hline
正味の電流 & ゼロ & 大(指数関数的) & ほぼゼロ($I_0$) \\
\hline
\end{tabular}
\end{table}

\textbf{重要なまとめ:ダイオードの整流作用の本質}

\begin{enumerate}
\item \textbf{順バイアス}:外部電圧がエネルギー障壁を\textbf{下げる} → 拡散電流が増大 → 電流が流れる
\item \textbf{逆バイアス}:外部電圧がエネルギー障壁を\textbf{上げる} → 拡散電流が抑制 → 電流が流れない
\item \textbf{整流作用}:バンドの曲がりを制御することで、一方向にのみ電流を流す
\end{enumerate}

この電磁気学的な理解は、BJT、MOSFET、IGBTなどの他のパワー半導体素子の動作原理を理解する上でも基礎となります。

\textbf{順方向電圧降下の物理的意味}

pn接合のエネルギー障壁は完全にゼロにはならないため($V_{bi} - V_F > 0$)、ダイオードには必ず順方向電圧降下が存在します。この電圧降下は、ダイオードの損失の原因となります。

\section{バイポーラトランジスタ(BJT)の動作原理}

\subsection{BJTの基本構造}

バイポーラトランジスタ(BJT: Bipolar Junction Transistor)は、npn型またはpnp型の3層構造を持つ半導体素子です。図\ref{fig:bjt_structure}に示すように、エミッタ(E)、ベース(B)、コレクタ(C)の3つの端子を持ち、矢印の向きに電流が流れるように設計されています。

\begin{figure}[H]
\centering
\fbox{\includegraphics[width=0.95\textwidth]{chapters/chapter03/images/page-10.pdf}}
\caption{バイポーラトランジスタ(BJT)の動作原理}
\label{fig:bjt_structure}
\end{figure}

BJTでは、エミッタからコレクタへ電子を流すために、中央のp型領域(ベース)を通過する必要があります。ここで重要なのは、ベース領域を非常に薄く作ることです。

\subsection{BJTの動作メカニズム}

BJTの動作原理を理解するためには、まず通常の動作時のバイアス条件を確認する必要があります:

\begin{itemize}
\item \textbf{エミッタ-ベース接合}:順方向バイアス(約0.7 V)
\item \textbf{ベース-コレクタ接合}:逆方向バイアス(数V〜数十V)
\end{itemize}

この条件下で、以下のプロセスが進行します:

\begin{enumerate}
\item \textbf{エミッタ(n型)からベース(p型)への電子の注入}
\begin{itemize}
\item エミッタは高濃度にドーピングされたn型半導体($N_d \approx 10^{19}$ cm$^{-3}$)
\item 順方向バイアスにより、エミッタから大量の電子がベース領域へ注入される
\end{itemize}

\item \textbf{ベース領域での電子の挙動}
\begin{itemize}
\item ベース領域は薄く(典型的には0.1〜数$\mu$m)、かつ低濃度にドーピングされたp型半導体($N_a \approx 10^{16}$ cm$^{-3}$)
\item ベース領域に注入された電子は、2つの経路をたどる:
\begin{enumerate}
\item \textbf{再結合経路}:ベース領域の正孔と再結合して消滅(少数)
\item \textbf{通過経路}:ベース領域を拡散してコレクタへ到達(多数)
\end{enumerate}
\end{itemize}

\item \textbf{コレクタ(n型)へ電子が到達}
\begin{itemize}
\item ベース-コレクタ接合の逆バイアスにより、強い電界が形成されている
\item ベース領域からコレクタ側へ到達した電子は、この電界で加速されてコレクタ電極へ引き込まれる
\end{itemize}
\end{enumerate}

図\ref{fig:bjt_structure}の下部に示されているバンド図では、エミッタ、ベース、コレクタの各領域におけるエネルギーバンドの関係が示されています。電子(黒丸)はエミッタから注入され、ベース領域の正孔(白丸)と再結合することなく、コレクタへと流れます。

\textbf{なぜ電流増幅が起こるのか?}

BJTの電流増幅の鍵は、\textbf{ベース領域が非常に薄く、かつ低濃度にドーピングされている}ことにあります。これにより、以下の現象が生じます:

\begin{itemize}
\item \textbf{ベースが薄い}:電子がベース領域を通過する時間が短い(典型的には数ps〜数十ps)
\item \textbf{ベースの正孔濃度が低い}:電子が正孔と遭遇する確率が低い
\item \textbf{エミッタの電子濃度が高い}:大量の電子が注入される
\end{itemize}

その結果、エミッタから注入された電子のうち、\textbf{大部分(例えば99\%以上)がベース領域を通過してコレクタに到達}し、\textbf{ごく一部(例えば1\%未満)だけがベース領域で再結合}します。

\textbf{具体例による理解:}

例えば、エミッタから100個の電子が注入されたとします:
\begin{itemize}
\item 99個の電子がベースを通過してコレクタに到達 → \textbf{コレクタ電流} $I_C$
\item 1個の電子がベースで正孔と再結合 → この再結合を補うために外部から正孔(電流)が供給される → \textbf{ベース電流} $I_B$
\end{itemize}

この場合、電流増幅率$\beta$は:
\[
\beta = \frac{I_C}{I_B} = \frac{99}{1} = 99
\]

つまり、\textbf{わずか1単位のベース電流を流すだけで、99単位のコレクタ電流を制御できる}のです。これがBJTの電流増幅作用の本質です。

\textbf{重要なポイント:}

\begin{itemize}
\item ベース電流$I_B$は、ベース領域での再結合を補うための電流
\item コレクタ電流$I_C$は、ベース領域を通過した電子の流れ
\item $I_C \gg I_B$となるのは、ベース領域が薄く低濃度であるため、再結合が少ないから
\item ベース電流を制御することで、エミッタからの電子注入量が変化し、結果としてコレクタ電流が変化する
\end{itemize}

このメカニズムにより、小さなベース電流で大きなコレクタ電流を制御することができ、これがトランジスタの増幅作用の基本原理となります。

\subsection{なぜベース電流が必要なのか?}

ここで重要な疑問が生じます:\textbf{「エミッタ-ベース接合に順バイアスをかけると拡散電流が流れるはずなのに、なぜ外部からベース電流を供給する必要があるのか?」}

この疑問に答えるために、pn接合の順バイアス時の電流の流れを詳しく見てみましょう。

\textbf{エミッタ-ベース接合の順バイアス時の拡散電流:}

エミッタ-ベース接合に順バイアスを印加すると、確かに拡散電流が流れます。しかし、この拡散電流には\textbf{2つの成分}があります:

\begin{enumerate}
\item \textbf{電子の拡散}:エミッタ(n型)$\rightarrow$ベース(p型)
\begin{itemize}
\item エミッタは超高濃度にドーピングされている($N_d \approx 10^{19}$ cm$^{-3}$)
\item 大量の電子がベース領域へ注入される
\end{itemize}

\item \textbf{正孔の拡散}:ベース(p型)$\rightarrow$エミッタ(n型)
\begin{itemize}
\item ベースは低濃度にドーピングされている($N_a \approx 10^{16}$ cm$^{-3}$)
\item わずかな正孔しかエミッタへ注入されない
\end{itemize}
\end{enumerate}

エミッタとベースのドーピング濃度の差が$10^3$倍もあるため、\textbf{電子の拡散が圧倒的に支配的}です。つまり、順バイアスによる拡散電流の大部分は、エミッタからベースへの電子の流れです。

\textbf{ベース領域での正孔の消費:}

ベース領域に注入された電子の一部(例えば1\%)は、ベース領域の正孔と再結合します。この再結合により:

\begin{itemize}
\item ベース領域の\textbf{正孔が消費}される
\item 時間が経つと、ベース領域の正孔濃度が\textbf{減少}する
\end{itemize}

\textbf{なぜベース電流が必要か:}

もし外部からベース端子を通じて正孔を補給しなければ、以下のことが起こります:

\begin{enumerate}
\item ベース領域の正孔濃度が減少し続ける
\item 正孔濃度が減少すると、エミッタ-ベース接合の順バイアス状態が維持できなくなる
\item エミッタからの電子注入が停止する
\item \textbf{トランジスタ動作が停止}する
\end{enumerate}

したがって、\textbf{ベース電流は、再結合により消費された正孔を補充し、トランジスタ動作を持続させるために必要}なのです。

\textbf{電流の収支バランス:}

BJTの動作中の電流の流れを整理すると:

\begin{itemize}
\item \textbf{エミッタ電流} $I_E$:エミッタから注入される電子の流れ(最大)
\item \textbf{コレクタ電流} $I_C$:ベースを通過してコレクタに到達した電子の流れ($I_E$の約99\%)
\item \textbf{ベース電流} $I_B$:ベースで再結合した電子を補うために、ベース端子から供給される正孔の流れ($I_E$の約1\%)
\end{itemize}

キルヒホッフの電流則により:
\begin{equation}
I_E = I_C + I_B
\end{equation}

\textbf{具体例:}

100 mAのエミッタ電流が流れている場合:
\begin{itemize}
\item 99 mAの電子がコレクタに到達 → $I_C = 99$ mA
\item 1 mAの電子がベースで再結合 → この再結合を補うために $I_B = 1$ mA の正孔をベース端子から供給
\item $I_E = I_C + I_B = 99 + 1 = 100$ mA
\end{itemize}

\textbf{重要なまとめ:}

\begin{itemize}
\item 順バイアスによる拡散電流は確かに存在するが、それは主に\textbf{エミッタからベースへの電子の流れ}
\item ベース領域で電子が正孔と再結合すると、\textbf{正孔が消費}される
\item この消費された正孔を補充するために、\textbf{外部からベース電流を供給}する必要がある
\item ベース電流がないと、正孔が枯渇してトランジスタ動作が停止する
\item ベース電流を制御することで、エミッタからの電子注入量を制御でき、結果としてコレクタ電流を制御できる
\end{itemize}

これがBJTにおいてベース電流が必要不可欠である理由です。

\subsubsection{ベース電流と正孔補充のメカニズム}

ベース電流の役割をより深く理解するために、\textbf{キャリア補充の観点}から詳しく見ていきましょう。

\textbf{ステップ1:電子の注入と再結合}

エミッタ-ベース接合に順バイアスを印加すると:

\begin{enumerate}
\item エミッタ(n$^+$型)から大量の電子がベース(p型)へ注入される
\item ベース領域に注入された電子の挙動:
\begin{itemize}
\item \textbf{大部分(約99\%)}:ベースを通過してコレクタへ到達
\item \textbf{一部(約1\%)}:ベース領域の正孔と再結合して消滅
\end{itemize}
\end{enumerate}

\textbf{ステップ2:正孔の消費}

再結合により、ベース領域の正孔が\textbf{消費}されます:

\begin{itemize}
\item 電子(●) + 正孔(○) → 消滅
\item ベース領域の正孔濃度が減少する傾向
\end{itemize}

\textbf{ステップ3:ベース電流による正孔補充}

もし外部からベース端子を通じて正孔を補給しなければ:

\begin{enumerate}
\item ベース領域の正孔濃度が減少し続ける
\item 正孔濃度が減少すると、エミッタ-ベース接合の\textbf{順バイアス状態が維持できなくなる}
\item エミッタからの電子注入が停止する
\item \textbf{トランジスタ動作が停止}する
\end{enumerate}

したがって、\textbf{ベース電流$I_B$は、再結合により消費された正孔を補充し、トランジスタ動作を持続させるために必要}です。

\textbf{電流の収支バランス:}

\begin{itemize}
\item \textbf{エミッタ電流} $I_E$:エミッタから注入される電子の流れ(最大)
\item \textbf{コレクタ電流} $I_C$:ベースを通過してコレクタに到達した電子の流れ($I_E$の約99\%)
\item \textbf{ベース電流} $I_B$:ベースで再結合した電子を補うために、ベース端子から供給される正孔の流れ($I_E$の約1\%)
\end{itemize}

キルヒホッフの電流則により:
\begin{equation}
I_E = I_C + I_B
\end{equation}

\textbf{具体例:}

100 mAのエミッタ電流が流れている場合(電流増幅率$\beta = 99$として):

\begin{itemize}
\item 99 mAの電子がコレクタに到達 → $I_C = 99$ mA
\item 1 mAの電子がベースで再結合 → この再結合を補うために $I_B = 1$ mA の正孔をベース端子から供給
\item $I_E = I_C + I_B = 99 + 1 = 100$ mA
\end{itemize}

\textbf{なぜベース電流が小さいのか:}

ベース電流が小さい($I_B \ll I_C$)のは、以下の設計によります:

\begin{enumerate}
\item \textbf{ベースを薄くする}:電子がベース領域を通過する時間が短い(典型的には数ps〜数十ps)
\item \textbf{ベースの正孔濃度を低くする}:電子が正孔と遭遇する確率が低い($N_a \approx 10^{16}$ cm$^{-3}$)
\item \textbf{エミッタの電子濃度を高くする}:大量の電子が注入される($N_d \approx 10^{19}$ cm$^{-3}$)
\end{enumerate}

その結果、エミッタから注入された電子のうち、\textbf{大部分(例えば99\%以上)がベース領域を通過してコレクタに到達}し、\textbf{ごく一部(例えば1\%未満)だけがベース領域で再結合}します。

\textbf{重要なまとめ:}

\begin{itemize}
\item ベース電流は、ベース領域で消費された正孔を補充するために必要
\item ベース電流がないと、正孔が枯渇してトランジスタ動作が停止する
\item ベース電流を制御することで、エミッタからの電子注入量を制御でき、結果としてコレクタ電流を制御できる
\item この仕組みにより、小さなベース電流で大きなコレクタ電流を制御する\textbf{電流増幅}が実現される
\end{itemize}

\subsection{BJTのバイアス条件}

BJTをスイッチング素子として動作させるためには、適切なバイアス条件が必要です。図\ref{fig:bjt_bias}に、順バイアスと逆バイアスの状態を示します。

\begin{figure}[H]
\centering
\fbox{\includegraphics[width=0.95\textwidth]{chapters/chapter03/images/page-11.pdf}}
\caption{バイポーラトランジスタ(BJT)のバイアス条件}
\label{fig:bjt_bias}
\end{figure}

\textbf{順バイアス状態}では:

\begin{itemize}
\item エミッタ-ベース間に順方向電圧を印加
\item エミッタから電子(黒丸)がベース領域へ注入される
\item ベース領域のp型半導体中には正孔(白丸)が存在
\item 注入された電子の一部は、ベース領域の正孔と\textbf{拡散→再結合}により消滅
\item しかし、ベース領域が十分薄いため、多くの電子は再結合せずにコレクタまで到達
\end{itemize}

図の注釈にある「拡散→再結合により電子がコレクタまで流れない」という現象は、BJTの重要な損失メカニズムです。ベース電流$I_B$は、この再結合により消費される電子の流れに相当します。

一方、\textbf{逆バイアス状態}では、エミッタ-ベース間の電位関係が逆転し、電子の注入が阻止されます。

BJTの電流増幅率$\beta$(ベータ)は、コレクタ電流$I_C$とベース電流$I_B$の比として定義されます:

\begin{equation}
\beta = \frac{I_C}{I_B}
\end{equation}

通常、$\beta$は数十から数百の値を持ち、小さなベース電流で大きなコレクタ電流を制御できることを意味します。

\subsection{電磁気学的観点からのBJTの動作原理}

BJTの動作を深く理解するためには、2つのpn接合(エミッタ-ベース接合とベース-コレクタ接合)のバンドの曲がりと電位分布を電磁気学的に理解する必要があります。

\subsubsection{アクティブモード時のバンド図}

BJTが正常に動作するアクティブモードでは、以下のバイアス条件が設定されます:

\begin{itemize}
\item エミッタ-ベース接合(J1):\textbf{順方向バイアス}($V_{BE} \approx 0.7$ V)
\item ベース-コレクタ接合(J2):\textbf{逆方向バイアス}($V_{CB}$は数V〜数十V)
\end{itemize}

\textbf{各接合でのバンドの曲がり:}

\begin{enumerate}
\item \textbf{J1(エミッタ-ベース)接合}:順バイアス
\begin{itemize}
\item ダイオードの順バイアスと同様に、エネルギー障壁が\textbf{低下}
\item 障壁の高さ:$e(V_{bi} - V_{BE})$
\item エミッタ(n$^+$)からベース(p)へ電子が拡散しやすくなる
\item バンドの曲がりが\textbf{緩やか}になる
\end{itemize}

\item \textbf{J2(ベース-コレクタ)接合}:逆バイアス
\begin{itemize}
\item ダイオードの逆バイアスと同様に、エネルギー障壁が\textbf{増加}
\item 障壁の高さ:$e(V_{bi} + V_{CB})$
\item ベース(p)からコレクタ(n)への正孔の拡散が抑制される
\item バンドの曲がりが\textbf{急}になる
\item 空乏層が拡大し、強い電界が形成される
\end{itemize}
\end{enumerate}

\textbf{電子の流れとエネルギーダイアグラム:}

npn型BJTの場合、電子のエネルギーダイアグラムは以下のようになります:

\begin{verbatim}
   E (n+)    B (p)        C (n)
   ┌───┐   ┌──────┐   ┌───┐  Ec(伝導帯下端)
   │   │   │      │   │   │
   │ ●│→ →│ ●  → │ → │ ● │  電子の流れ
   │   │   │   ↓  │   │   │
   └───┘   └──────┘   └───┘  Ev(価電子帯上端)
     J1        J2
  (順バイアス)(逆バイアス)
\end{verbatim}

\textbf{エミッタ-ベース接合(J1)での電子注入:}

\begin{itemize}
\item エミッタ(n$^+$)は高い電子濃度を持つ($N_d \approx 10^{19}$ cm$^{-3}$)
\item 順バイアスにより、エネルギー障壁が低下
\item エミッタの伝導帯の電子は、低下した障壁を越えてベースの伝導帯へ拡散
\item ベース領域に入った電子は、ベース領域の正孔より\textbf{高いエネルギー状態}にある
\end{itemize}

\textbf{ベース領域での電子の挙動:}

\begin{itemize}
\item ベース領域に注入された電子は、\textbf{少数キャリア}として存在
\item 電子は濃度勾配により、J2方向へ\textbf{拡散}
\item ベースが薄いため、大部分の電子はJ2に到達する前に正孔と再結合しない
\end{itemize}

\textbf{ベース-コレクタ接合(J2)での電子の加速:}

\begin{itemize}
\item J2の逆バイアスにより、ベース→コレクタの方向に強い電界が形成
\item ベースからJ2の空乏層に到達した電子は、この電界で\textbf{ドリフト}(加速)される
\item 電子は急峻なエネルギー勾配(バンドの傾き)を「滑り降りる」ようにコレクタへ移動
\item コレクタに到達した電子は、コレクタ電極へ引き込まれる
\end{itemize}

\subsubsection{電位分布とエネルギー障壁}

BJTの電位分布を電磁気学的に理解することで、電子の流れがより明確になります:

\textbf{各領域の電位:}

\begin{itemize}
\item エミッタ電位:$\phi_E = 0$(基準)
\item ベース電位:$\phi_B = +V_{BE} \approx +0.7$ V
\item コレクタ電位:$\phi_C = +V_{CE}$(例:$+5$ V〜$+100$ V)
\end{itemize}

\textbf{電子のポテンシャルエネルギー:}

電子のエネルギーは電位と逆の関係($E = -e\phi$)にあるため:

\begin{itemize}
\item エミッタ領域:電子のエネルギーが\textbf{最も高い}($E_E = 0$)
\item ベース領域:電子のエネルギーが\textbf{やや低い}($E_B = -eV_{BE}$)
\item コレクタ領域:電子のエネルギーが\textbf{最も低い}($E_C = -eV_{CE}$)
\end{itemize}

この電子エネルギーの傾斜により、電子はエミッタ→ベース→コレクタの方向へ流れます。

\subsubsection{バンド図と電流の対応関係}

以下の表に、BJTの各接合における電磁気学的な量の対応関係をまとめます:

\begin{table}[H]
\centering
\caption{BJTアクティブモードにおける物理量の比較}
\begin{tabular}{|l|c|c|}
\hline
\textbf{物理量} & \textbf{J1(E-B)} & \textbf{J2(B-C)} \\
\hline
バイアス状態 & 順バイアス & 逆バイアス \\
\hline
エネルギー障壁 & $e(V_{bi} - V_{BE})$ & $e(V_{bi} + V_{CB})$ \\
\hline
バンドの曲がり & 緩やか & 急 \\
\hline
空乏層幅 & 小さい & 大きい \\
\hline
電場の強さ & 弱い & 強い \\
\hline
電子の移動機構 & 拡散(濃度勾配) & ドリフト(電界) \\
\hline
主な電流 & 拡散電流 & ドリフト電流 \\
\hline
\end{tabular}
\end{table}

\textbf{重要なまとめ:BJTの電流増幅の本質}

\begin{enumerate}
\item \textbf{J1の順バイアス}:エネルギー障壁を下げ、エミッタから大量の電子を注入
\item \textbf{薄いベース}:注入された電子の大部分が再結合せずにJ2に到達
\item \textbf{J2の逆バイアス}:強い電界で電子をコレクタへ加速・引き込み
\item \textbf{ベース電流}:ベースで再結合した電子を補うための小さな電流
\item \textbf{電流増幅}:小さなベース電流で大きなコレクタ電流を制御
\end{enumerate}

この電磁気学的な理解により、BJTがなぜ電流増幅素子として機能するのか、そしてベース領域を薄くすることがなぜ重要なのかが明確になります。

\section{MOSFET(金属酸化膜半導体電界効果トランジスタ)}

\subsection{MOSFETの構造}

MOSFET(Metal-Oxide-Semiconductor Field-Effect Transistor)は、トランジスタ(BJT)と構造がよく似ていますが、制御方法が異なります。図\ref{fig:mosfet_structure}に示すように、MOSFETはnpn型の基本構造に、絶縁体(酸化膜)と導体(金属ゲート)が追加された構造を持っています。

\begin{figure}[H]
\centering
\fbox{\includegraphics[width=0.95\textwidth]{chapters/chapter03/images/page-15.pdf}}
\caption{MOSFET(金属酸化膜半導体電界効果トランジスタ)の構造}
\label{fig:mosfet_structure}
\end{figure}

MOSFETの主な構成要素は以下の通りです:

\begin{itemize}
\item \textbf{ソース(S)}:電子の供給源となるn型領域
\item \textbf{ドレイン(D)}:電子の排出先となるn型領域
\item \textbf{ゲート(G)}:絶縁体を介して電圧を印加する電極
\item \textbf{p型基板}:ソースとドレインの間のp型領域
\end{itemize}

BJTではベース電流$I_B$で制御しますが、MOSFETではゲート電圧$V_G$で制御します。これは、絶縁体を介して電界を加えることで、p型領域に反転層(n型のチャネル)を形成し、電流を流す経路を作るためです。

\subsection{MOSFETの動作原理}

MOSFETの動作原理は、電界効果を利用しています:

\begin{enumerate}
\item ゲート電圧$V_G$を印加しない状態では、ソース-ドレイン間は2つのpn接合によって遮断されています
\item ゲート電圧$V_G$を印加すると、p型基板表面に電子が引き寄せられ、反転層(nチャネル)が形成されます
\item この反転層を通じて、ソースからドレインへ電流$i$が流れます
\end{enumerate}

図\ref{fig:mosfet_structure}の右側に示された3次元構造図では、絶縁体(酸化膜)を介してゲート電極が配置され、その下にp型基板があることが明確に示されています。ゲート電圧によって形成される反転層が、ソースとドレインをつなぐ導電路となります。

\subsection{反転層の形成メカニズム}

MOSFETの動作の核心は、\textbf{反転層(inversion layer)}の形成にあります。図\ref{fig:mosfet_inversion}に、反転層形成の詳細なメカニズムを示します。

\begin{figure}[H]
\centering
\fbox{\includegraphics[width=0.95\textwidth]{chapters/chapter03/images/page-16.pdf}}
\caption{MOSFETの反転層形成メカニズム}
\label{fig:mosfet_inversion}
\end{figure}

図の左側には、\textbf{バンド図}が示されています。ここで重要なエネルギー準位は以下の通りです:

\begin{itemize}
\item $E_c$:伝導帯の下端(conduction band edge)
\item $E_F$:フェルミ準位(Fermi level)
\item $E_v$:価電子帯の上端(valence band edge)
\end{itemize}

\textbf{ゲート電圧を印加していない状態}では:

\begin{itemize}
\item p型基板のフェルミ準位$E_F$は、価電子帯$E_v$に近い位置にある
\item 絶縁体によって金属ゲートと半導体が分離されている
\item p型基板表面には正孔(白丸)が多数存在
\end{itemize}

\textbf{ゲート電圧$V_G$を印加すると}(図の右側の注釈「ゲートに電位差をかけるとオンにできる」):

\begin{itemize}
\item 金属ゲートが正電位になると、絶縁体を介して電界が形成される
\item この電界により、p型基板表面に電子(黒丸)が引き寄せられる
\item p型基板表面の電子濃度が正孔濃度を超えると、\textbf{反転層}が形成される
\item この反転層は、実質的にn型の導電チャネルとして機能する
\end{itemize}

図の左側のバンド図を見ると、電子(黄色の点で表示)がフェルミ準位$E_F$付近に多数存在していることがわかります。反転層が形成されると、ソース(S)とドレイン(D)の間にn型の導電路ができ、電流が流れるようになります。

\textbf{重要なポイント:}

MOSFETでは、BJTと異なり、制御端子(ゲート)に電流を流す必要がありません。ゲート電圧によって電界を形成するだけで、主電流を制御できます。これにより、\textbf{極めて低い入力電力}でスイッチングが可能となります。

\subsection{電磁気学的観点からのMOSFETの動作原理}

MOSFETの反転層形成を電磁気学とバンド理論の観点から理解することで、電界効果トランジスタの本質が明確になります。

\subsubsection{ゲート電圧印加によるバンドの曲がり}

\textbf{ゲート電圧なし($V_G = 0$)の状態:}

\begin{itemize}
\item p型基板のフェルミ準位$E_F$は、価電子帯$E_v$に近い
\item 基板表面でもバンドは平坦(電界なし)
\item 正孔が多数キャリア、電子は少数キャリア
\end{itemize}

\textbf{正のゲート電圧印加($V_G > 0$)の状態:}

ゲート電極に正電圧を印加すると、以下の変化が起こります:

\begin{enumerate}
\item \textbf{電界の形成}:
\begin{itemize}
\item ゲート電極(正)と基板(負)の間に垂直方向の電場が形成
\item 電場の強さ:$\mathcal{E} = V_G / t_{ox}$($t_{ox}$は酸化膜厚)
\item 酸化膜は絶縁体なので、電流は流れない(電界のみ)
\end{itemize}

\item \textbf{基板表面での電荷の再分布}:
\begin{itemize}
\item 電界により、p型基板表面の正孔が\textbf{下方へ押し出される}(反発)
\item 同時に、電子が基板表面へ\textbf{引き寄せられる}(吸引)
\item 基板表面の電位が上昇
\end{itemize}

\item \textbf{バンドの曲がり}(基板表面での電位変化):
\begin{itemize}
\item 基板表面の電位上昇 → 電子のエネルギー $E = -e\phi$ が\textbf{低下}
\item バンド端($E_c$、$E_v$)が基板表面で\textbf{下方へ曲がる}
\item 表面での伝導帯端$E_c$がフェルミ準位$E_F$に\textbf{近づく}
\end{itemize}

\item \textbf{反転層の形成}:
\begin{itemize}
\item ゲート電圧が閾値電圧$V_{th}$を超えると、$E_c$が$E_F$より下になる
\item 表面の伝導帯に電子が占有される(フェルミ-ディラック分布)
\item 電子濃度が正孔濃度を超える → \textbf{反転}
\item 表面が実質的にn型となる(nチャネル形成)
\end{itemize}
\end{enumerate}

\subsubsection{閾値電圧の物理的意味}

閾値電圧$V_{th}$は、バンドの曲がりの観点から以下のように理解できます:

\begin{itemize}
\item $V_G < V_{th}$:基板表面の$E_c$がまだ$E_F$より上 → 電子濃度が低い → オフ状態
\item $V_G = V_{th}$:基板表面の$E_c$が$E_F$と一致 → 反転が始まる
\item $V_G > V_{th}$:基板表面の$E_c$が$E_F$より下 → 多数の電子が存在 → オン状態
\end{itemize}

閾値電圧は、p型基板をn型に反転させるのに必要な電位の変化量に対応します。

\subsubsection{電界効果による制御の本質}

MOSFETの動作原理の本質は、\textbf{電界によってバンドを曲げ、キャリア濃度を制御する}ことにあります:

\begin{table}[H]
\centering
\caption{ゲート電圧とバンドの曲がりの対応関係}
\begin{tabular}{|l|c|c|c|}
\hline
\textbf{状態} & \textbf{$V_G = 0$} & \textbf{$V_G = V_{th}$} & \textbf{$V_G > V_{th}$} \\
\hline
電界 & なし & 中程度 & 強い \\
\hline
バンドの曲がり & 平坦 & 曲がり始め & 大きく曲がる \\
\hline
表面電子濃度 & 極小 & 増加開始 & 高濃度 \\
\hline
チャネル & なし & 形成開始 & 形成完了 \\
\hline
動作状態 & オフ & 閾値 & オン \\
\hline
\end{tabular}
\end{table}

\textbf{重要なまとめ:MOSFET制御の電磁気学的メカニズム}

\begin{enumerate}
\item \textbf{ゲート電圧}:絶縁体を介して電界を形成
\item \textbf{電界}:基板表面の電位を変化させる
\item \textbf{電位変化}:バンドを曲げる($E = -e\phi$)
\item \textbf{バンドの曲がり}:伝導帯端を下げ、電子を表面に引き寄せる
\item \textbf{反転層形成}:表面がn型に反転し、導電チャネルができる
\item \textbf{電流制御}:チャネルを通じてソース-ドレイン間に電流が流れる
\end{enumerate}

この電界による制御方式により、MOSFETは極めて小さい入力電力(ゲート電圧のみ、電流なし)で大きな出力電流を制御できます。

\subsection{MOSFETのスイッチ特性}

MOSFETの大きな特徴の一つは、pn接合のエネルギー障壁がないことです。図\ref{fig:mosfet_characteristics}に示すように、これによりオン電圧が小さくなります。

\begin{figure}[H]
\centering
\fbox{\includegraphics[width=0.95\textwidth]{chapters/chapter03/images/page-20.pdf}}
\caption{MOSFETのスイッチ特性}
\label{fig:mosfet_characteristics}
\end{figure}

グラフは、ドレイン-ソース間電圧$v_{\text{ds}}$とドレイン電流$i_d$の関係を、ゲート-ソース間電圧$V_{\text{gs}}$をパラメータとして示しています。ゲート電圧を変化させることで、ドレイン電流を制御できることがわかります。

図中に示されているように、pn接合のエネルギー障壁がないため、オン電圧が小さいという利点があります。これにより、スイッチング損失を低減できます。

\section{IGBT(絶縁ゲートバイポーラトランジスタ)}

\subsection{IGBTの動作原理}

IGBT(Insulated Gate Bipolar Transistor)は、MOSFETの制御の容易さとBJTの低オン抵抗の特性を組み合わせた素子です。図\ref{fig:igbt}に示すように、IGBTはnpnp型の4層構造を持ち、ゲート電圧で制御されます。

\begin{figure}[H]
\centering
\fbox{\includegraphics[width=0.95\textwidth]{chapters/chapter03/images/page-25.pdf}}
\caption{IGBTの動作原理}
\label{fig:igbt}
\end{figure}

IGBTの構造的特徴は以下の通りです:

\begin{itemize}
\item n型、p型、n型、p型の4層構造
\item ゲートに電圧をかけることで、絶縁体を介して制御
\item エミッタ(E)、コレクタ(C)、ゲート(G)の3端子構造
\end{itemize}

バイアスをかけて電流を流れやすくした場合、図の下部に示されているように、n型領域とp型領域が交互に配置された構造となります。ゲート電圧によって、エミッタからコレクタへの電流経路が制御されます。

\subsection{IGBTのハイブリッド構造}

IGBTの最大の特徴は、\textbf{MOSFETの入力特性}と\textbf{BJTの出力特性}を組み合わせたハイブリッド構造にあります。図\ref{fig:igbt_hybrid}に、ゲート電圧を印加した際の動作メカニズムを示します。

\begin{figure}[H]
\centering
\fbox{\includegraphics[width=0.95\textwidth]{chapters/chapter03/images/page-26.pdf}}
\caption{IGBTのハイブリッド構造と拡散層の形成}
\label{fig:igbt_hybrid}
\end{figure}

図に示されているように、\textbf{ゲート電圧をかけた場合}には以下の動作が起こります:

\begin{enumerate}
\item \textbf{MOSFET部分の動作}:
\begin{itemize}
\item ゲート電圧により、p型領域表面に反転層(nチャネル)が形成される
\item この反転層を通じて、エミッタからn型領域へ電子が流入する
\end{itemize}

\item \textbf{拡散層の形成}(図の注釈「拡散層が形成され、電子が流れる」):
\begin{itemize}
\item n型領域に注入された電子(黒丸)が、p型領域へ拡散していく
\item 同時に、p型領域から正孔(白丸)がn型領域へ注入される
\item この双方向のキャリア注入により、\textbf{導電率変調(conductivity modulation)}が発生
\end{itemize}

\item \textbf{導電率変調の効果}:
\begin{itemize}
\item n型ドリフト領域に正孔が注入されることで、この領域の導電率が大幅に向上
\item 結果として、オン抵抗が低減される
\item これがBJTの特性を活かした低損失化のメカニズム
\end{itemize}
\end{enumerate}

図の下部のバンド図を見ると、n-p-n-pの4層構造が明確に示されています。ゲート電圧によって形成されたチャネルを通じて電子が流れ(ピンクの矢印)、同時に正孔も注入されることで(青い矢印)、全体として大きな電流を流すことができます。

\subsubsection{IGBTの動作メカニズムの詳細}

IGBTの最大の特徴である\textbf{伝導度変調(conductivity modulation)}について、詳しく見ていきましょう。

\textbf{ステップ1:MOSFET部分の動作(電子の流れ開始)}

\begin{enumerate}
\item ゲート電圧$V_G$を印加すると、p型領域表面に反転層(nチャネル)が形成される
\item エミッタのn$^+$領域から、この反転層を通じて電子(●)がn$^-$ドリフト領域へ流入する
\item この段階では、MOSFETと同様のユニポーラ動作(電子のみの流れ)
\end{enumerate}

\textbf{ステップ2:正孔の注入(バイポーラ動作の開始)}

\begin{enumerate}
\item n$^-$ドリフト領域に電子が流入すると、コレクタ側のp$^+$層とn$^-$層の接合が順バイアス状態になる
\item p$^+$層(コレクタ)から正孔(○)がn$^-$ドリフト層へ注入される
\item これにより、\textbf{双方向のキャリア注入}が実現される
\end{enumerate}

\textbf{ステップ3:伝導度変調の発生}

n$^-$ドリフト領域に正孔が注入されると、以下の現象が起こります:

\begin{itemize}
\item \textbf{キャリア濃度の増加}:
\begin{equation}
n_{\text{total}} = n_{\text{electron}} + p_{\text{hole}} \gg n_{\text{electron}}
\end{equation}

\item \textbf{導電率の向上}:
\begin{equation}
\sigma = e(n\mu_n + p\mu_p) \quad \text{($p$の寄与で大幅に増加)}
\end{equation}

\item \textbf{抵抗の低下}:
\begin{equation}
R = \frac{1}{\sigma A/L} \quad \text{($\sigma$が増加すると$R$が減少)}
\end{equation}
\end{itemize}

\textbf{伝導度変調の効果:}

正孔注入により、n$^-$ドリフト層の導電率が以下のように変化します:

\begin{itemize}
\item \textbf{注入前}:低濃度n型($N_d \approx 10^{14}$ cm$^{-3}$) → 高抵抗
\item \textbf{注入後}:高濃度キャリア(電子+正孔 $\approx 10^{16}$〜$10^{17}$ cm$^{-3}$) → \textbf{低抵抗}
\end{itemize}

この導電率の変調により、オン抵抗が大幅に低減されます。

\textbf{MOSFETとの比較:}

\begin{table}[H]
\centering
\caption{MOSFETとIGBTの電流メカニズムの比較}
\begin{tabular}{|l|c|c|}
\hline
\textbf{項目} & \textbf{MOSFET} & \textbf{IGBT} \\
\hline
キャリア & 電子のみ & 電子+正孔 \\
\hline
動作モード & ユニポーラ & バイポーラ \\
\hline
ドリフト層の抵抗 & 高い & 低い(伝導度変調) \\
\hline
オン抵抗 & 大きい & 小さい \\
\hline
\end{tabular}
\end{table}

\textbf{なぜIGBTは低オン抵抗を実現できるのか:}

\begin{enumerate}
\item \textbf{MOSFET}では、ドリフト層の抵抗がオン抵抗の主要因
\begin{itemize}
\item 高耐圧を得るために、ドリフト層を厚くかつ低濃度にする必要がある
\item これにより、オン抵抗が増大する
\end{itemize}

\item \textbf{IGBT}では、伝導度変調により抵抗を低減
\begin{itemize}
\item ドリフト層は低濃度でも、正孔注入により実効的に高濃度化
\item 高耐圧と低オン抵抗を両立できる
\end{itemize}
\end{enumerate}

\textbf{電流の内訳:}

IGBTのコレクタ電流$I_C$は、以下の2つの成分から構成されます:

\begin{equation}
I_C = I_{\text{electron}} + I_{\text{hole}}
\end{equation}

\begin{itemize}
\item $I_{\text{electron}}$:MOSFETチャネルを通じた電子電流
\item $I_{\text{hole}}$:コレクタから注入された正孔電流(伝導度変調に寄与)
\end{itemize}

この二重の電流メカニズムにより、IGBTは大電流を低損失で流すことができます。

\subsubsection{電磁気学的観点からのIGBT動作}

IGBTの動作を、バンドダイアグラムと電位分布の観点から理解することで、なぜ伝導度変調が起こるのかが明確になります。

\textbf{オン状態でのバンド図(簡略化):}

IGBTをオンにすると、以下のバンドの曲がりが形成されます:

\begin{enumerate}
\item \textbf{MOSFETチャネル部}:ゲート電界でバンドが曲がり、反転層(nチャネル)形成
\item \textbf{n$^+$-p接合(エミッタ側)}:短絡されているため、バンドの曲がりはほぼなし
\item \textbf{p-n$^-$接合(中央部)}:最初は逆バイアスだが、電子が流れ込むと順バイアス化
\item \textbf{n$^-$-p$^+$接合(コレクタ側)}:\textbf{順バイアス状態になる}
\end{enumerate}

\textbf{伝導度変調のメカニズム(バンドダイアグラムの観点):}

\begin{itemize}
\item MOSFETチャネルからn$^-$ドリフト層へ電子が注入される
\item n$^-$層の電位が上昇し、n$^-$-p$^+$接合が順バイアス状態になる
\item n$^-$-p$^+$接合のエネルギー障壁が低下($eV_{bi} \rightarrow e(V_{bi} - V_F)$)
\item p$^+$層(コレクタ)から正孔がn$^-$層へ\textbf{注入}される
\item n$^-$層のキャリア濃度が増加 → 導電率上昇 → 抵抗低下
\end{itemize}

\textbf{重要なポイント:}

IGBTの低オン抵抗は、\textbf{pn接合の順バイアス時の両極性キャリア注入}という、ダイオードやBJTと同じ物理現象を利用しています。ゲート制御でこの状態を作り出すことが、IGBTの巧妙な設計です。

\subsection{IGBTの利点と特性}

IGBTは、MOSFETとBJTの長所を組み合わせることで、以下の優れた特性を実現しています:

\begin{itemize}
\item \textbf{電圧駆動}:MOSFETと同様に、ゲート電圧で制御できるため、駆動回路が簡単
\item \textbf{低オン抵抗}:導電率変調により、BJTのような低いオン抵抗を実現
\item \textbf{高耐圧}:npnp構造により、高電圧に耐えられる
\item \textbf{大電流容量}:導電率変調により、大電流を流すことが可能
\end{itemize}

これらの特性により、IGBTは中~大容量の電力変換器で広く使用されています。特に、電気自動車のインバータや産業用モータドライブなどで重要な役割を果たしています。

\textbf{MOSFETとIGBTの使い分け}:

\begin{itemize}
\item \textbf{MOSFET}:低電圧・高周波数・小~中容量の用途に適する
\item \textbf{IGBT}:高電圧・中周波数・中~大容量の用途に適する
\end{itemize}

\subsection{IGBTのオフ状態における電位分布}

IGBTはnpnp型の4層構造を持つため、3つのpn接合が存在します。オフ状態(ゲート電圧なし)でコレクタ-エミッタ間に電圧を印加すると、\textbf{なぜ特定の接合に電圧が集中するのか}という疑問が生じます。

\textbf{IGBTの3つのpn接合:}

\begin{verbatim}
E側   n+ | p | n- | p+   C側
        J1   J2   J3
\end{verbatim}

\begin{itemize}
\item \textbf{J1(エミッタ側)}:n$^+$-p接合
\item \textbf{J2(中央)}:p-n$^-$接合
\item \textbf{J3(コレクタ側)}:n$^-$-p$^+$接合
\end{itemize}

\subsubsection{オフ状態での各接合の役割}

コレクタが正、エミッタが負の電圧を印加した場合(通常の使用状態):

\begin{enumerate}
\item \textbf{J1(n$^+$-p)接合}:
\begin{itemize}
\item エミッタ電極がn$^+$層とp層の\textbf{両方に接触}している
\item 実質的に\textbf{短絡(ショート)}されている
\item ほとんど電位差がかからない
\item 抵抗:$R_1 \approx 0$
\end{itemize}

\item \textbf{J2(p-n$^-$)接合}:
\begin{itemize}
\item この接合は\textbf{逆バイアス}状態
\item 空乏層が広がり、\textbf{ほぼ全ての電圧をここで支える}
\item 耐圧を決定する重要な接合
\item 抵抗:$R_2 \gg R_1, R_3$(非常に大きい)
\end{itemize}

\item \textbf{J3(n$^-$-p$^+$)接合}:
\begin{itemize}
\item 順バイアス方向(p$^+$が正)
\item しかし、電流は流れない(後述の理由による)
\item 電圧降下は小さい
\item 抵抗:$R_3$ は小さい(順バイアスだが電流がないため)
\end{itemize}
\end{enumerate}

\textbf{なぜJ2に電圧が集中するのか:}

電圧は\textbf{抵抗が高い部分}に多くかかります。3つの接合を直列接続した抵抗と考えると:

\begin{equation}
V_{\text{CE}} = V_{J1} + V_{J2} + V_{J3}
\end{equation}

\begin{itemize}
\item $V_{J1} \approx 0$(J1は短絡)
\item $V_{J2} \approx V_{\text{CE}}$(J2が高抵抗)
\item $V_{J3} \approx 0$(順バイアスだが電流なし)
\end{itemize}

\subsubsection{キャリア補充経路の重要性}

\textbf{重要な疑問:}「J3が順バイアスなら電流が流れるはずでは?」

実際には、J3を通じて連続的な電流は流れません。その理由を\textbf{キャリア補充の観点}から説明します。

\textbf{もしJ3で電流が流れようとすると:}

\begin{enumerate}
\item J3(n$^-$-p$^+$)が順バイアスで、正孔がp$^+$からn$^-$へ注入される
\item 同時に、電子がn$^-$からp$^+$へ移動する
\item しかし、n$^-$層の電子を\textbf{補充する経路}が必要
\end{enumerate}

\textbf{キャリア補充経路の遮断:}

\begin{verbatim}
E側   n+ | p | n- | p+   C側
        J1   J2   J3
             ↑
            逆バイアス
          (電子が通れない)
\end{verbatim}

\begin{itemize}
\item J3で電子が消費されると、n$^-$層の電子が枯渇する
\item 電子を補充するには、J2を通って左側(p層)から供給される必要がある
\item しかし、\textbf{J2は逆バイアス}で遮断されている
\item 電子の補充経路がない
\end{itemize}

\textbf{結果:}

\begin{itemize}
\item n$^-$層の電子が枯渇して電流は止まる
\item ごく微小な\textbf{漏れ電流}しか流れない
\item \textbf{電圧はJ2で支えられる}
\end{itemize}

\textbf{電流の連続性:}

電流が連続的に流れるためには、全ての接合で電流が等しくなければなりません:

\begin{equation}
I_{J1} = I_{J2} = I_{J3}
\end{equation}

J2が逆バイアスで電流をほぼ遮断 → 全体の電流がほぼゼロ → J3も実質的に電流が流れない → \textbf{電圧はJ2の空乏層で支えられる}

\textbf{まとめ:}

\begin{itemize}
\item J1:エミッタ電極で短絡 → 電圧降下なし
\item J2:逆バイアスで高抵抗 → \textbf{ほぼ全電圧を支える}
\item J3:順バイアス方向だがキャリア補充経路がない → 電流が流れず、電圧降下小
\end{itemize}

この原理は、IGBTだけでなく、サイリスタなどの他の多層構造デバイスにも共通する重要な概念です。

\section{サイリスタの動作原理}

\subsection{サイリスタの基本動作}

サイリスタは、4層構造(npnp)を持つ半導体素子で、一度オンすると外部から制御しない限りオン状態を保持する特性を持ちます。図\ref{fig:thyristor}に示すように、何も接続していない場合の熱平衡状態では、各層のバンドが図のように形成されます。

\begin{figure}[H]
\centering
\fbox{\includegraphics[width=0.95\textwidth]{chapters/chapter03/images/page-30.pdf}}
\caption{サイリスタの動作原理}
\label{fig:thyristor}
\end{figure}

サイリスタの構造は、IGBTと同様にnpnpの4層構造ですが、制御方法が異なります。サイリスタでは、ゲート端子(G)に信号を与えることで、素子をオン状態にすることができます。

熱平衡状態のバンド図を見ると、n型とp型の接合部分でバンドが曲がっていることがわかります。この状態では、電子(黒丸)と正孔(白丸)が各領域に留まっており、電流は流れません。

ゲート信号によってサイリスタがオンすると、エミッタ(E)からコレクタ(C)へ電流が流れ始め、その後はゲート信号がなくても電流が流れ続けます。この特性は、ラッチアップと呼ばれます。

\section{ワイドギャップ半導体}

\subsection{ワイドギャップ半導体とは}

ワイドギャップ半導体は、従来のシリコン(Si)よりも大きなバンドギャップを持つ半導体材料です。図\ref{fig:widegap}に示すように、代表的なワイドギャップ半導体には、4H-SiC(シリコンカーバイド)、GaN(窒化ガリウム)、ダイヤモンドなどがあります。

\begin{figure}[H]
\centering
\fbox{\includegraphics[width=0.95\textwidth]{chapters/chapter03/images/page-35.pdf}}
\caption{ワイドギャップ半導体}
\label{fig:widegap}
\end{figure}

表に示されているように、ワイドギャップ半導体は以下の特性を持ちます:

\begin{itemize}
\item \textbf{大きなバンドギャップ}:Siの1.12 eVに対し、4H-SiCは3.26 eV、GaNは3.39 eV
\item \textbf{高い絶縁破壊電界強度}:Siの0.3 MV/cmに対し、4H-SiCは2.5 MV/cm、GaNは3.3 MV/cm
\item \textbf{高い熱伝導度}:Siの1.5 W/cmKに対し、4H-SiCは4.9 W/cmK、GaNは2 W/cmK
\end{itemize}

これらの特性により、高耐圧・低損失化が可能となり、放熱にも優れています。ただし、デメリットとして生産が難しいという課題があります。

\subsection{ワイドギャップ半導体のその他の特徴}

ワイドギャップ半導体の優れた特性は、デバイスの小型化にも貢献します。図\ref{fig:widegap_features}に示すように、容量成分が小さくなることで、以下のメリットが得られます:

\begin{figure}[H]
\centering
\fbox{\includegraphics[width=0.95\textwidth]{chapters/chapter03/images/page-40.pdf}}
\caption{ワイドギャップ半導体のその他の特徴}
\label{fig:widegap_features}
\end{figure}

\begin{enumerate}
\item \textbf{容量成分が小さい}:入力容量と出力容量が小さくなる
\item \textbf{スイッチングスピードが速くなる}:容量が小さいため、充放電時間が短縮
\item \textbf{スイッチング周波数が高くなる}:より高速なスイッチングが可能
\item \textbf{受動部品(LやC)を小さくできる}:高周波化により、インダクタやコンデンサを小型化可能
\item \textbf{製品がより小型に}:全体としてデバイスの小型化が実現
\end{enumerate}

表には、SiC MOSFETとSi MOSFET、GaN HEMTの比較が示されており、特に入力容量、出力容量、逆方向転送容量において、GaN HEMTが優れた特性を示しています。

グラフでは、GaN HEMTとSi MOSFETのスイッチング特性の比較が示されており、GaN HEMTの方がスイッチング時間が短く、高速動作が可能であることがわかります。

\section{まとめ}

\subsection{本章の要点}

本章では、スイッチング素子の物理について説明しました。主な内容は以下の通りです:

\begin{enumerate}
\item \textbf{半導体の構造からスイッチの制御の原理を説明した}:
\begin{itemize}
\item BJTはベース電流で制御
\item MOSFETはゲート電圧で制御(電界効果)
\item IGBTはMOSFETとBJTの特性を組み合わせた素子
\item サイリスタはラッチアップ特性を持つ素子
\end{itemize}

\item \textbf{ワイドギャップ半導体の特徴を説明した}:
\begin{itemize}
\item 大きなバンドギャップにより高耐圧化が可能
\item 高い絶縁破壊電界強度により薄型化が可能
\item 高い熱伝導度により放熱性能が向上
\item 小さい容量により高速スイッチングが可能
\item デバイスの小型化・高効率化が実現
\end{itemize}
\end{enumerate}

\begin{figure}[H]
\centering
\fbox{\includegraphics[width=0.95\textwidth]{chapters/chapter03/images/page-41.pdf}}
\caption{本日のまとめ}
\label{fig:summary}
\end{figure}

\subsection{次回の予告}

次回の講義では、これらのスイッチング素子を用いた実際の電力変換回路について学習します。具体的には、以下の内容を扱う予定です:

\begin{itemize}
\item 整流回路
\item インバータ回路
\item DC-DCコンバータ
\item PWM制御技術
\end{itemize}

本章で学んだスイッチング素子の特性を理解していることが、次回の内容を理解する上で重要となります。

\subsection{演習問題}

本章の理解を深めるために、以下の演習問題に取り組んでください:

\begin{enumerate}
\item BJTとMOSFETの制御方法の違いを説明してください。

\item パワー半導体において、耐圧と損失の間にトレードオフがある理由を、ドーピング濃度の観点から説明してください。

\item ワイドギャップ半導体がSiに比べて優れている点を3つ挙げ、それぞれについて説明してください。

\item IGBTがMOSFETとBJTの特性をどのように組み合わせているかを説明してください。

\item サイリスタのラッチアップ特性について説明し、この特性がどのような応用に適しているかを考察してください。
\end{enumerate}
