% パワーエレクトロニクス教科書 メインファイル
% 大阪工業大学 工学部 電気電子システム工学科
% 2025年度 後期

\documentclass[a4paper,11pt]{ltjsbook}

%========================================
% パッケージの読み込み
%========================================
\usepackage{graphicx}                 % 画像の挿入
\graphicspath{{./}{chapters/chapter01/images/}{chapters/chapter02/images/}{chapters/chapter03/images/}}  % 画像検索パス
\usepackage{amsmath,amssymb}         % 数式
\usepackage{bm}                       % 太字ベクトル
\usepackage{fancyhdr}                 % ヘッダー・フッター
\usepackage[top=25mm,bottom=25mm,left=25mm,right=25mm]{geometry}  % 余白設定
\usepackage{here}                     % 図の位置指定
\usepackage{url}                      % URL
\usepackage{ascmac}                   % 枠囲み
\usepackage{color}                    % 色
\usepackage{listings}                 % ソースコード(将来使用する可能性)
\usepackage{tikz}                     % TikZ描画
\usepackage{pgfplots}                 % グラフ描画
\pgfplotsset{compat=1.18}
\usepackage{circuitikz}               % 回路図描画

%========================================
% ヘッダーとフッターの設定
%========================================
\pagestyle{fancy}
\fancyhf{}
\fancyhead[LE,RO]{\thepage}
\fancyhead[RE]{\leftmark}
\fancyhead[LO]{\rightmark}
\renewcommand{\headrulewidth}{0.4pt}
\setlength{\footskip}{15mm}

%========================================
% タイトル情報
%========================================
\title{%
  \Huge パワーエレクトロニクス\\[10mm]
  \LARGE 講義教科書
}
\author{%
  大阪工業大学\\
  工学部\\
  電気電子システム工学科
}
\date{2025年度 後期}

%========================================
% 本文開始
%========================================
\begin{document}

%========================================
% 前付け
%========================================
\frontmatter

% 表紙
\maketitle
\thispagestyle{empty}
\clearpage

% まえがき(オプション)
\chapter*{まえがき}
\addcontentsline{toc}{chapter}{まえがき}

本教科書は、大阪工業大学工学部電気電子システム工学科の「パワーエレクトロニクス」講義のために作成されました。

パワーエレクトロニクスは、電力を効率的に制御・変換する技術であり、現代社会において極めて重要な役割を果たしています。電気自動車、再生可能エネルギーシステム、産業用モータ制御、家電製品など、私たちの身の回りには数多くのパワーエレクトロニクス機器が存在しています。

本教科書では、パワーエレクトロニクスの基礎となる半導体物理から始まり、各種パワー半導体素子の動作原理、電力変換回路の設計、制御技術に至るまで、体系的に学習します。

各章は、講義1回分に対応しています。図や数式を用いて丁寧に解説し、理解を深めるための演習問題も用意しました。初学者でも無理なく理解できるよう、基礎から段階的に説明しています。

本教科書が、皆さんのパワーエレクトロニクスの学習に役立つことを願っています。

\vspace{5mm}
\hfill 2025年10月

\clearpage

% 目次
\tableofcontents
\clearpage

%========================================
% 本文
%========================================
\mainmatter

%========================================
% 各章の読み込み
%========================================

% 第1章:パワーエレクトロニクスの概要
\chapter{パワーエレクトロニクスの概要}

\section{はじめに}

\subsection{講義の進め方}

本講義では、パワーエレクトロニクスの基礎から応用までを体系的に学習する。資料などの共有事項は専用のwebページに掲載されており、主にスライドを使って説明を行う。板書は補足で説明するときに利用し、講義の後に演習問題を出す。演習課題を解いて復習することが重要である。わからない点があれば質問してほしい。

\begin{figure}[H]
\centering
\fbox{\includegraphics[width=0.95\textwidth]{chapters/chapter01/images/page-02.pdf}}
\caption{講義の進め方}
\label{fig:lecture_method}
\end{figure}

\subsection{本講義の目標}

本講義では、以下の4つの目標を達成することを目指す:

\begin{itemize}
\item パワーエレクトロニクスの概要を理解する
\item 可変抵抗を用いた電力変換を理解する
\item 理想的なスイッチと現実との違いを理解する
\item 電力損失の計算を理解する
\end{itemize}

\begin{figure}[H]
\centering
\fbox{\includegraphics[width=0.95\textwidth]{chapters/chapter01/images/page-03.pdf}}
\caption{本講義の目標}
\label{fig:objectives}
\end{figure}

\begin{figure}[H]
\centering
\fbox{\includegraphics[width=0.95\textwidth]{chapters/chapter01/images/page-04.pdf}}
\caption{本日の目標}
\label{fig:todays_objectives}
\end{figure}

\section{パワーエレクトロニクスとは}

\subsection{パワーエレクトロニクスの定義}

パワーエレクトロニクスとは、「パワー」と「エレクトロニクス」を組み合わせた学問分野である。

\begin{itemize}
\item パワー:電力・エネルギー
\item エレクトロニクス:半導体
\end{itemize}

つまり、\textbf{半導体を使って電力を変換・制御する工学}であり、全ては\textbf{エネルギーを効率よく使うため}の技術である。

\begin{figure}[H]
\centering
\fbox{\includegraphics[width=0.95\textwidth]{chapters/chapter01/images/page-05.pdf}}
\caption{パワーエレクトロニクスとは?}
\label{fig:what_is_pe}
\end{figure}

\subsection{電力変換の例}

電力変換の代表的な例として、交流から直流への変換がある。例えば、コンセントから得られる100Vの交流電圧を、ノートパソコンなどで使用する15Vの直流電圧に変換する。この変換を実現するために、以下の3つの主要な素子が用いられる:

\begin{enumerate}
\item キャパシタ(コンデンサ):電荷を蓄える
\item インダクタ(コイル):磁気エネルギーを蓄える
\item スイッチング素子(半導体):電流のオン・オフを制御
\end{enumerate}

\begin{figure}[H]
\centering
\fbox{\includegraphics[width=0.95\textwidth]{chapters/chapter01/images/page-06.pdf}}
\caption{電力変換の例(AC-DC変換)}
\label{fig:ac_dc_conversion}
\end{figure}

\subsection{身の回りの電力変換}

現代社会では、あらゆる場所で電力変換が行われている。スマートフォンの充電器、ノートパソコンのACアダプタ、電気自動車の充電システム、太陽光発電システムなど、私たちの生活は電力変換技術なしには成り立たない。

\begin{figure}[H]
\centering
\fbox{\includegraphics[width=0.95\textwidth]{chapters/chapter01/images/page-07.pdf}}
\caption{身の回りの電力変換}
\label{fig:power_conversion_examples}
\end{figure}

\begin{figure}[H]
\centering
\fbox{\includegraphics[width=0.95\textwidth]{chapters/chapter01/images/page-08.pdf}}
\caption{電力変換の応用例}
\label{fig:applications}
\end{figure}

\subsection{電力変換の種類}

電力変換には、以下の4つの基本的な種類がある:

\begin{enumerate}
\item \textbf{AC-DC変換(整流)}:交流を直流に変換(例:スマホ充電器)
\item \textbf{DC-AC変換(インバータ)}:直流を交流に変換(例:太陽光発電)
\item \textbf{DC-DC変換(チョッパ)}:直流電圧を昇圧・降圧(例:USB充電)
\item \textbf{AC-AC変換(サイクロコンバータ)}:交流の周波数や電圧を変換
\end{enumerate}

\begin{figure}[H]
\centering
\fbox{\includegraphics[width=0.95\textwidth]{chapters/chapter01/images/page-09.pdf}}
\caption{電力変換の種類}
\label{fig:conversion_types}
\end{figure}

\section{パワーエレクトロニクスで扱う3つの要素}

パワーエレクトロニクスは、以下の3つの要素の組み合わせで構成される:

\begin{enumerate}
\item \textbf{スイッチングデバイス(半導体)}:電力の流れを制御
\item \textbf{電気・電子回路}:回路の設計と解析
\item \textbf{制御}:スイッチングのタイミングと方法
\end{enumerate}

これらの要素は、電子物性論、電子デバイス工学、電気電子材料、電子回路工学、電磁気学、電磁界理論、電気回路、ディジタル電子回路、制御工学などの基礎知識の組み合わせで成り立っている。

\begin{figure}[H]
\centering
\fbox{\includegraphics[width=0.95\textwidth]{chapters/chapter01/images/page-10.pdf}}
\caption{パワーエレクトロニクスで扱う3つの要素}
\label{fig:three_elements}
\end{figure}

\section{電気回路の基礎}

\subsection{回路解析の基本法則}

パワーエレクトロニクス回路を理解するためには、電気回路の基本法則を理解することが不可欠である。

\begin{figure}[H]
\centering
\fbox{\includegraphics[width=0.95\textwidth]{chapters/chapter01/images/page-11.pdf}}
\caption{電気回路の基礎}
\label{fig:circuit_basics}
\end{figure}

\subsection{キルヒホッフの法則}

電気回路の解析には、キルヒホッフの2つの法則が基本となる:

\subsubsection{キルヒホッフの電圧則(KVL: Kirchhoff's Voltage Law)}

閉回路における電圧の総和はゼロである。

\begin{equation}
\sum V = 0
\end{equation}

\subsubsection{キルヒホッフの電流則(KCL: Kirchhoff's Current Law)}

ノード(接続点)に流入する電流の総和と流出する電流の総和は等しい。

\begin{equation}
\sum I_{\text{in}} = \sum I_{\text{out}}
\end{equation}

\begin{figure}[H]
\centering
\fbox{\includegraphics[width=0.95\textwidth]{chapters/chapter01/images/page-12.pdf}}
\caption{キルヒホッフの法則}
\label{fig:kirchhoff}
\end{figure}

\subsection{オームの法則}

抵抗$R$に流れる電流$I$と電圧$V$の関係は、オームの法則で表される:

\begin{equation}
V = RI
\end{equation}

この法則は、パワーエレクトロニクス回路の解析において最も基本的な関係式である。

\begin{figure}[H]
\centering
\fbox{\includegraphics[width=0.95\textwidth]{chapters/chapter01/images/page-13.pdf}}
\caption{オームの法則}
\label{fig:ohms_law}
\end{figure}

\subsection{回路問題の定式化}

回路問題を解く際には、KVL(キルヒホッフの電圧則)とKCL(キルヒホッフの電流則)、そして枝構成式(オームの法則など)を全て書き出すことで、どんな回路の問題も解くことができる。

\begin{figure}[H]
\centering
\fbox{\includegraphics[width=0.95\textwidth]{chapters/chapter01/images/page-14.pdf}}
\caption{回路問題の解法}
\label{fig:circuit_solution}
\end{figure}

\begin{figure}[H]
\centering
\fbox{\includegraphics[width=0.95\textwidth]{chapters/chapter01/images/page-15.pdf}}
\caption{回路問題の定式化}
\label{fig:circuit_formulation}
\end{figure}

\section{可変抵抗を用いた電力変換}

\subsection{可変抵抗による電圧制御}

最も単純な電力変換の方法は、可変抵抗を用いた電圧制御である。抵抗値を変化させることで、負荷に加わる電圧を調整することができる。

\begin{figure}[H]
\centering
\fbox{\includegraphics[width=0.95\textwidth]{chapters/chapter01/images/page-16.pdf}}
\caption{可変抵抗を用いた電力変換}
\label{fig:variable_resistor}
\end{figure}

\subsection{可変抵抗の問題点}

可変抵抗を用いた電力変換には、重大な問題点がある。それは、\textbf{効率が非常に悪い}ことである。可変抵抗で電圧を下げる際、余分なエネルギーは熱として失われてしまう。

\begin{figure}[H]
\centering
\fbox{\includegraphics[width=0.95\textwidth]{chapters/chapter01/images/page-17.pdf}}
\caption{可変抵抗の問題点}
\label{fig:resistor_problem}
\end{figure}

\subsection{電力変換の効率}

電力変換の効率$\eta$は、出力電力$P_{\text{out}}$と入力電力$P_{\text{in}}$の比で定義される:

\begin{equation}
\eta = \frac{P_{\text{out}}}{P_{\text{in}}}
\end{equation}

ここで、電力$P$は電圧$V$と電流$I$の積である:

\begin{equation}
P = VI
\end{equation}

\begin{figure}[H]
\centering
\fbox{\includegraphics[width=0.95\textwidth]{chapters/chapter01/images/page-18.pdf}}
\caption{電力変換の効率}
\label{fig:efficiency}
\end{figure}

\subsection{効率計算の例}

例えば、12Vから5Vに電圧を変換する場合を考える。負荷電流が0.5Aのとき、可変抵抗を用いた変換では:

\begin{itemize}
\item 入力電力:$P_{\text{in}} = 12 \times 0.5 = 6$ W
\item 出力電力:$P_{\text{out}} = 5 \times 0.5 = 2.5$ W
\item 効率:$\eta = 2.5/6 = 41.7$\%
\end{itemize}

このように、可変抵抗を用いた電力変換では、半分以上のエネルギーが無駄になってしまう。

\begin{figure}[H]
\centering
\fbox{\includegraphics[width=0.95\textwidth]{chapters/chapter01/images/page-19.pdf}}
\caption{効率計算の具体例}
\label{fig:efficiency_calculation}
\end{figure}

\begin{figure}[H]
\centering
\fbox{\includegraphics[width=0.95\textwidth]{chapters/chapter01/images/page-20.pdf}}
\caption{可変抵抗の電力変換の効率は?}
\label{fig:resistor_efficiency}
\end{figure}

\section{スイッチングによる高効率電力変換}

\subsection{スイッチングの原理}

可変抵抗の問題を解決するために、スイッチングという手法が用いられる。抵抗値を連続的に変化させるのではなく、スイッチのオン・オフを高速で切り替えることで、効率的な電力変換を実現する。

\begin{figure}[H]
\centering
\fbox{\includegraphics[width=0.95\textwidth]{chapters/chapter01/images/page-21.pdf}}
\caption{スイッチングによる電力変換}
\label{fig:switching}
\end{figure}

\subsection{デューティ比}

スイッチングにおいて重要な概念が\textbf{デューティ比(Duty ratio)}$D$である。これは、スイッチがオンになっている時間の割合を表す:

\begin{equation}
D = \frac{T_{\text{on}}}{T_{\text{on}} + T_{\text{off}}} = \frac{T_{\text{on}}}{T}
\end{equation}

ここで、$T$は周期、$T_{\text{on}}$はオン時間、$T_{\text{off}}$はオフ時間である。

\begin{figure}[H]
\centering
\fbox{\includegraphics[width=0.95\textwidth]{chapters/chapter01/images/page-22.pdf}}
\caption{デューティ比}
\label{fig:duty_ratio}
\end{figure}

\subsection{スイッチングによる電圧制御}

スイッチングを用いることで、平均電圧を制御できる。デューティ比$D$を調整することで、出力電圧$V_{\text{out}}$を入力電圧$V_{\text{in}}$に対して以下のように制御できる:

\begin{equation}
V_{\text{out}} = D \cdot V_{\text{in}}
\end{equation}

\begin{figure}[H]
\centering
\fbox{\includegraphics[width=0.95\textwidth]{chapters/chapter01/images/page-23.pdf}}
\caption{スイッチングによる電圧制御}
\label{fig:switching_control}
\end{figure}

\begin{figure}[H]
\centering
\fbox{\includegraphics[width=0.95\textwidth]{chapters/chapter01/images/page-24.pdf}}
\caption{スイッチングの効果}
\label{fig:switching_effect}
\end{figure}

\section{理想的なスイッチと実際のスイッチ}

\subsection{理想的なスイッチの性質}

理想的なスイッチは、以下の3つの性質を持つ:

\begin{enumerate}
\item 0秒でスイッチをon・offできる(瞬間的な切り替え)
\item スイッチoff時は電流は流れない(完全な遮断)
\item スイッチon時は電圧降下はない(完全な導通)
\end{enumerate}

理想的なスイッチでは、電力損失$P_{\text{loss}} = V \times I = 0$となる。これは、off時は$I = 0$、on時は$V = 0$であるためである。

\begin{figure}[H]
\centering
\fbox{\includegraphics[width=0.95\textwidth]{chapters/chapter01/images/page-25.pdf}}
\caption{理想的なスイッチの性質}
\label{fig:ideal_switch}
\end{figure}

\subsection{実際のスイッチング素子}

実際の半導体スイッチング素子(MOSFET、IGBT、ダイオードなど)は、理想的なスイッチとは異なる特性を持つ:

\begin{enumerate}
\item スイッチングに有限の時間がかかる(スイッチング損失)
\item on状態でも電圧降下が存在する(導通損失)
\item off状態でもわずかな漏れ電流が流れる
\end{enumerate}

\begin{figure}[H]
\centering
\fbox{\includegraphics[width=0.95\textwidth]{chapters/chapter01/images/page-26.pdf}}
\caption{実際のスイッチング素子}
\label{fig:real_switch}
\end{figure}

\subsection{スイッチング損失}

実際のスイッチング素子では、on・off切り替え時に電圧と電流が同時に存在する期間があり、この間に電力損失が発生する。これを\textbf{スイッチング損失}と呼ぶ。

\begin{figure}[H]
\centering
\fbox{\includegraphics[width=0.95\textwidth]{chapters/chapter01/images/page-27.pdf}}
\caption{スイッチング損失}
\label{fig:switching_loss}
\end{figure}

\subsection{導通損失}

スイッチがon状態のとき、理想的にはゼロであるべき電圧降下が実際には存在する。この電圧降下によって発生する損失を\textbf{導通損失}と呼ぶ。

\begin{equation}
P_{\text{conduction}} = V_{\text{on}} \cdot I
\end{equation}

ここで、$V_{\text{on}}$はon時の電圧降下である。

\begin{figure}[H]
\centering
\fbox{\includegraphics[width=0.95\textwidth]{chapters/chapter01/images/page-28.pdf}}
\caption{導通損失}
\label{fig:conduction_loss}
\end{figure}

\subsection{トータル損失}

パワー半導体素子の全体的な損失は、スイッチング損失と導通損失の和として表される:

\begin{equation}
P_{\text{total}} = P_{\text{switching}} + P_{\text{conduction}}
\end{equation}

高効率な電力変換を実現するためには、これらの損失を最小化することが重要である。

\begin{figure}[H]
\centering
\fbox{\includegraphics[width=0.95\textwidth]{chapters/chapter01/images/page-29.pdf}}
\caption{トータル損失}
\label{fig:total_loss}
\end{figure}

\section{まとめ}

本章では、パワーエレクトロニクスの概要について学習した。主な内容は以下の通りである:

\begin{itemize}
\item 電力変換の意味と日常生活での応用例
\item 可変抵抗を用いた電力変換(効率が悪い)
\item 効率の良い電力変換の方法→スイッチング
\item 実際のスイッチング素子(半導体)の損失
\end{itemize}

パワーエレクトロニクスは、半導体を使って電力を効率よく変換・制御する工学である。可変抵抗を用いた方法では効率が悪いため、スイッチングという手法を用いることで高効率な電力変換が実現できる。しかし、実際のスイッチング素子には、スイッチング損失と導通損失が存在し、これらを考慮した設計が必要である。

\begin{figure}[H]
\centering
\fbox{\includegraphics[width=0.95\textwidth]{chapters/chapter01/images/page-30.pdf}}
\caption{まとめ}
\label{fig:summary}
\end{figure}

\subsection{次回の予告}

次回は、半導体の物理について学習する。パワーエレクトロニクスで用いられる半導体素子の動作原理を理解するために、半導体の基礎的な物理現象について詳しく説明する。

\subsection{演習問題}

\begin{enumerate}
\item パワーエレクトロニクスの定義を説明せよ。
\item AC-DC変換、DC-AC変換、DC-DC変換、AC-AC変換のそれぞれについて、具体的な応用例を挙げよ。
\item 可変抵抗を用いた電力変換の問題点を説明せよ。
\item 電源電圧が20V、負荷電圧が8V、負荷電流が2Aの場合、可変抵抗を用いた電力変換の効率を計算せよ。
\item スイッチングによる電力変換が高効率である理由を、理想的なスイッチの性質を用いて説明せよ。
\item 実際のスイッチング素子で発生する2種類の損失について説明せよ。
\end{enumerate}


% 第2章:半導体の物理
% 第2章:半導体の物理
\chapter{半導体の物理}

\section{はじめに}

\subsection{本章の目的と学習目標}

パワーエレクトロニクスは、電力を効率的に制御・変換する技術であり、現代社会において欠かせない重要な技術分野です。電気自動車、太陽光発電システム、産業用モータ制御、スマートフォンの充電器など、私たちの身の回りには数多くのパワーエレクトロニクス機器が存在しています。

この技術の中心には「半導体素子」があります。半導体素子は、電力を高効率で制御するスイッチとして機能し、パワーエレクトロニクスの性能を大きく左右します。本章では、これらの半導体素子がどのような物理原理で動作しているのかを、基礎から丁寧に学んでいきます。

\textbf{学習目標:}
\begin{itemize}
\item スイッチング素子である半導体の物理を理解する
\item 半導体の基本であるバンド理論について理解する
\item pn接合で起こる物理現象を理解する
\item 金属-半導体接合で起こる物理現象を理解する
\item 電磁気学の観点からバンドが描ける
\end{itemize}

\subsection{パワーエレクトロニクスと半導体素子}

パワーエレクトロニクスで使用される半導体素子は、主に「スイッチ」として機能します。理想的なスイッチは、以下の特性を持ちます。

\begin{itemize}
\item \textbf{オン状態}:電気抵抗がゼロで、大電流を流すことができる
\item \textbf{オフ状態}:電気抵抗が無限大で、電流を完全に遮断できる
\item \textbf{高速動作}:オンとオフを高速に切り替えられる
\item \textbf{低損失}:スイッチング時のエネルギー損失が小さい
\end{itemize}

実際の半導体素子は理想スイッチには及びませんが、機械式のスイッチに比べて圧倒的に高速で、しかも摩耗することなく動作できるという大きな利点があります。

\subsection{パワーエレクトロニクスで用いる半導体素子}

パワーエレクトロニクスで使用される主な半導体素子を図\ref{fig:devices}に示します。

\begin{figure}[H]
\centering
\fbox{\includegraphics[width=0.95\textwidth]{chapters/chapter02/images/page-03.pdf}}
\caption{パワーエレクトロニクスで用いる半導体素子と回路記号}
\label{fig:devices}
\end{figure}

\begin{itemize}
\item \textbf{ダイオード}:p型半導体とn型半導体を接合した最もシンプルな素子。一方向にのみ電流を流す整流作用を持つ。構造は「pn接合」。

\item \textbf{トランジスタ}:3層構造(npn型またはpnp型)を持ち、ベース電流により大きなコレクタ電流を制御できる増幅・スイッチング素子。

\item \textbf{サイリスタ}:4層構造(pnpn)を持ち、ゲート信号によりオン状態にできる素子。大電力用途に適している。

\item \textbf{MOSFET(Metal-Oxide-Semiconductor Field-Effect Transistor)}:金属・絶縁体・半導体の積層構造を持ち、ゲート電圧により電流を制御する。高速スイッチングが可能。

\item \textbf{IGBT(Insulated Gate Bipolar Transistor)}:MOSFETとトランジスタの長所を組み合わせた素子。高電圧・大電流用途に適している。
\end{itemize}

図\ref{fig:mosfet_igbt}に、MOSFETとIGBTの内部構造を示します。

\begin{figure}[H]
\centering
\fbox{\includegraphics[width=0.95\textwidth]{chapters/chapter02/images/page-04.pdf}}
\caption{MOSFETとIGBTの構造(pn接合+絶縁体+導体の組み合わせ)}
\label{fig:mosfet_igbt}
\end{figure}

これらの素子は、すべて半導体の\textbf{pn接合}を基本として動作しています。また、MOSFETやIGBTでは、半導体と金属、絶縁体を組み合わせた複雑な構造が使われています。したがって、pn接合および金属-半導体接合の物理を理解することが、これらの素子を理解する上で不可欠です。


\section{半導体とは何か}

\subsection{導体・半導体・絶縁体の違い}

物質は電気の流れやすさによって、大きく3つに分類されます。

\begin{itemize}
\item \textbf{導体(金属)}:電流が非常に流れやすい物質。銅、アルミニウム、金、銀など。
\item \textbf{絶縁体}:電流がほとんど流れない物質。ガラス、ゴム、プラスチックなど。
\item \textbf{半導体}:導体と絶縁体の中間的な性質を持つ物質。シリコン、ゲルマニウム、ガリウムヒ素など。
\end{itemize}

半導体の最大の特徴は、「電流を流したり、流さなかったりすることができる」ことです(図\ref{fig:what_is_semiconductor})。つまり、外部条件(温度、不純物、電圧など)によって電気伝導性を制御できるのです。

\begin{figure}[H]
\centering
\fbox{\includegraphics[width=0.95\textwidth]{chapters/chapter02/images/page-05.pdf}}
\caption{半導体とは?電流を流したり流さなかったりできる物質}
\label{fig:what_is_semiconductor}
\end{figure}

\subsection{電流と電荷の移動}

電流とは何でしょうか?電流は、\textbf{電荷(電子や正孔)が移動すること}で生じる現象です。

電流$I$は、単位時間あたりに流れる電荷$Q$の量として定義されます。
\begin{equation}
I = \frac{dQ}{dt}
\end{equation}

電子は負の電荷を持ち、電荷量は$-e = -1.602 \times 10^{-19}$ C(クーロン)です。正孔は正の電荷を持ち、電荷量は$+e$です。

したがって、導体・半導体・絶縁体の違いは、\textbf{電子(または正孔)の移動のしやすさ}によって決まります。

\begin{screen}
\textbf{重要な問い:}\\
電子の移動のしやすさは何によって決まるのでしょうか?
\end{screen}

この問いに答えるために、まず「電子がどこに存在しているか」を理解する必要があります。


\section{原子構造と電子配置}

\subsection{原子の構造}

すべての物質は原子から構成されています。原子は、中心にある\textbf{原子核}と、その周りを取り巻く\textbf{電子}から成り立っています。

\begin{itemize}
\item \textbf{原子核}:正電荷を持つ陽子と、電荷を持たない中性子から構成される
\item \textbf{電子}:負電荷を持ち、原子核の周りの特定の軌道上に存在する
\end{itemize}

原子核は非常に小さく(直径約$10^{-15}$ m)、原子全体のサイズ(直径約$10^{-10}$ m)に比べて10万分の1程度です。したがって、原子の大部分は空間であり、電子が原子核の周りを「雲」のように存在しています。

\subsection{電子殻と軌道}

電子は原子核の周りの\textbf{軌道}上に存在します(図\ref{fig:electron_orbit})。これらの軌道は「電子殻」と呼ばれ、内側から順にK殻、L殻、M殻...と名付けられています。

\begin{figure}[H]
\centering
\fbox{\includegraphics[width=0.95\textwidth]{chapters/chapter02/images/page-06.pdf}}
\caption{電子の軌道とエネルギー準位(ボーアの原子模型)}
\label{fig:electron_orbit}
\end{figure}

各電子殻は、さらに細かい軌道(s軌道、p軌道、d軌道など)に分かれています。

\begin{itemize}
\item \textbf{K殻(n=1)}:1s軌道のみ。最大2個の電子を収容。
\item \textbf{L殻(n=2)}:2s軌道と2p軌道。最大8個の電子を収容(2s: 2個、2p: 6個)。
\item \textbf{M殻(n=3)}:3s軌道、3p軌道、3d軌道。最大18個の電子を収容(3s: 2個、3p: 6個、3d: 10個)。
\end{itemize}

各軌道には、\textbf{パウリの排他原理}により、スピンが逆向きの電子が最大2個まで入ることができます。

\subsection{エネルギー準位}

重要なことは、各軌道が\textbf{固有のエネルギー}を持つということです(図\ref{fig:energy_level})。

\begin{figure}[H]
\centering
\fbox{\includegraphics[width=0.95\textwidth]{chapters/chapter02/images/page-07.pdf}}
\caption{軌道のエネルギー準位(低い方から順に占有される)}
\label{fig:energy_level}
\end{figure}

電子は、エネルギーの低い軌道から順番に占有されていきます。これは、自然界では系のエネルギーが最小になる状態が最も安定だからです。

最も外側の殻にある電子を\textbf{価電子}と呼び、この価電子が化学結合や電気伝導に重要な役割を果たします。

\subsection{シリコン原子の電子配置}

パワーエレクトロニクスで最も広く使われる半導体はシリコン(Si)です。シリコンは原子番号14の元素で、14個の電子を持ちます。

シリコン原子の電子配置は以下のようになります。
\begin{equation}
\text{Si: } 1s^2 \, 2s^2 \, 2p^6 \, 3s^2 \, 3p^2
\end{equation}

最も外側のM殻(n=3)には4個の電子があり、これが価電子です。シリコンは\textbf{4価の元素}と呼ばれます。

この4個の価電子が、シリコンの化学結合や電気的性質を決定します。


\section{バンド理論}

\subsection{孤立原子からバンドへ}

1個の原子では、電子はとびとびのエネルギー準位(1s, 2s, 2p, ...)を持ちます。しかし、原子が2個、3個と集まってくると、電子同士の相互作用により、状況が変わってきます。

原子が近づくと、各原子の電子軌道が重なり合い、電子は複数の原子にまたがって存在できるようになります。このとき、エネルギー準位が分裂し始めます。

固体のように原子の数が非常に多い場合(シリコンでは約$5 \times 10^{22}$個/cm$^3$)、エネルギー準位は無数に分裂し、事実上\textbf{連続したバンド(帯)}とみなすことができます(図\ref{fig:band_concept})。

\begin{figure}[H]
\centering
\fbox{\includegraphics[width=0.95\textwidth]{chapters/chapter02/images/page-08.pdf}}
\caption{バンド理論の概念(原子が集まるとエネルギー準位がバンドになる)}
\label{fig:band_concept}
\end{figure}

\textbf{バンドとは、電子が存在できるエネルギー準位の集まり}です。固体中の電子のエネルギー状態は、このバンドで説明されます。これが「バンド理論」の基本的な考え方です。

\subsection{価電子帯と伝導帯}

固体のバンド構造では、主に以下の2つのバンドが重要です。

\begin{itemize}
\item \textbf{価電子帯(Valence Band)}:価電子が占めるバンド。通常、電子で完全に満たされている。
\item \textbf{伝導帯(Conduction Band)}:電子が自由に移動できる空のバンド。
\end{itemize}

この2つのバンドの間には、電子が存在できないエネルギー領域があり、これを\textbf{バンドギャップ}($E_g$)と呼びます。

バンドギャップのエネルギーは物質によって異なります。
\begin{itemize}
\item シリコン(Si): $E_g \approx 1.1$ eV
\item ゲルマニウム(Ge): $E_g \approx 0.67$ eV
\item 炭化シリコン(SiC): $E_g \approx 3.3$ eV
\item 窒化ガリウム(GaN): $E_g \approx 3.4$ eV
\item ダイヤモンド(C): $E_g \approx 5.5$ eV
\end{itemize}

\subsection{導体・半導体・絶縁体のバンド図}

物質の電気的性質は、バンドギャップの大きさと伝導帯に電子が存在するかどうかによって決まります(図\ref{fig:band_materials})。

\begin{figure}[H]
\centering
\fbox{\includegraphics[width=0.95\textwidth]{chapters/chapter02/images/page-10.pdf}}
\caption{バンド図と物質の特徴(導体・半導体・絶縁体)}
\label{fig:band_materials}
\end{figure}

\begin{itemize}
\item \textbf{導体(金属)}:
\begin{itemize}
\item 伝導帯に電子が存在している
\item バンドギャップがほぼゼロか、価電子帯と伝導帯が重なっている
\item 室温で多数の自由電子が存在し、電流が非常に流れやすい
\item 代表例:銅、アルミニウム、金
\end{itemize}

\item \textbf{絶縁体}:
\begin{itemize}
\item バンドギャップが比較的大きい(通常5 eV以上)
\item 室温では熱エネルギー(約0.026 eV)では電子が伝導帯に励起されない
\item 電流がほとんど流れない
\item 代表例:酸化シリコン(SiO$_2$)、ガラス、プラスチック
\end{itemize}

\item \textbf{半導体}:
\begin{itemize}
\item バンドギャップが比較的小さい(通常0.5〜3 eV程度)
\item 室温でも熱エネルギーにより、わずかに電子が伝導帯に励起される
\item 外部からエネルギー(光、熱、電圧など)を加えることで電気伝導性を制御できる
\item 代表例:シリコン、ゲルマニウム、ガリウムヒ素
\end{itemize}
\end{itemize}

\subsection{フェルミエネルギーとフェルミ準位}

電子がどのエネルギー準位を占有しているかを表すために、\textbf{フェルミエネルギー}(または\textbf{フェルミ準位})$E_F$という概念を導入します。

フェルミエネルギー$E_F$は、「電子が存在する確率が50\%となるエネルギー準位」と定義されます。

\begin{itemize}
\item 絶対零度(0 K)では、$E_F$より低いエネルギー準位はすべて電子で満たされ、$E_F$より高いエネルギー準位は空である
\item 有限温度では、熱エネルギーにより、$E_F$付近のエネルギー分布がなだらかになる
\end{itemize}

真性半導体(不純物を含まない純粋な半導体)では、フェルミ準位はバンドギャップのほぼ中央に位置します。

\subsection{フェルミ・ディラック分布}

実際の物質では、電子を一つ一つ数えるのではなく、\textbf{統計的}に扱います(図\ref{fig:fermi_dirac})。

\begin{figure}[H]
\centering
\fbox{\includegraphics[width=0.95\textwidth]{chapters/chapter02/images/page-15.pdf}}
\caption{バンド理論における電子の統計的な取り扱い(フェルミ・ディラック分布)}
\label{fig:fermi_dirac}
\end{figure}

あるエネルギー$E$の状態に電子が存在する確率$f(E)$は、\textbf{フェルミ・ディラック分布}で表されます。

\begin{equation}
f(E) = \frac{1}{1 + \exp\left(\frac{E - E_F}{kT}\right)}
\end{equation}

ここで、
\begin{itemize}
\item $E$:電子のエネルギー [eV]
\item $E_F$:フェルミエネルギー [eV]
\item $k = 8.617 \times 10^{-5}$ eV/K:ボルツマン定数
\item $T$:絶対温度 [K]
\end{itemize}

この式の意味を考えてみましょう。

\begin{itemize}
\item $E = E_F$のとき、$f(E_F) = 1/(1+1) = 0.5$(50\%の確率)
\item $E \ll E_F$のとき、$f(E) \approx 1$(ほぼ100\%の確率で電子が存在)
\item $E \gg E_F$のとき、$f(E) \approx 0$(ほぼ0\%の確率、つまり電子がほとんど存在しない)
\end{itemize}

室温($T = 300$ K)では、$kT \approx 0.026$ eV(約26 meV)となります。この熱エネルギーにより、一部の電子が価電子帯から伝導帯へ励起(遷移)されます。

これにより、以下の現象が起こります。
\begin{itemize}
\item 伝導帯に電子が存在する確率が有限になる(わずかだが電子が存在する)
\item 価電子帯には電子が抜けた穴(\textbf{正孔})が残る
\end{itemize}

正孔は、あたかも正の電荷を持つ粒子のように振る舞い、電流を運ぶキャリアとなります。

\subsection{正孔の概念}

正孔(ホール)とは、価電子帯で電子が欠けた状態のことです。

価電子帯が完全に電子で満たされている場合、電子は移動することができません(隣の席もすべて埋まっている状態)。しかし、1つの電子が抜けて正孔ができると、隣の電子がその正孔に移動できるようになります。

このとき、電子が正孔に移動すると、元の位置に新たな正孔ができます。これは、正孔が電子とは逆方向に移動したように見えます。

正孔を1つの粒子として扱うと、以下の性質を持ちます。
\begin{itemize}
\item 電荷:$+e$(正の電荷)
\item 有効質量:正の値(電子とは異なる)
\item 移動方向:電場と同じ向き(電子とは逆)
\end{itemize}


\section{真性半導体と不純物半導体}

\subsection{真性半導体}

純粋な半導体を\textbf{真性半導体}(intrinsic semiconductor)と呼びます。真性半導体では、伝導帯の電子と価電子帯の正孔の数が等しくなります。

真性半導体中のキャリア濃度は、温度とバンドギャップに依存します。室温でのシリコンの真性キャリア濃度は約$n_i \approx 1.5 \times 10^{10}$ 個/cm$^3$です。

これは、シリコンの原子密度$5 \times 10^{22}$ 個/cm$^3$に比べて非常に小さい値です。つまり、真性半導体では、ほとんどの電子は価電子帯に留まっており、わずかな電子のみが伝導帯に励起されています。

\subsection{半導体の純度}

半導体素子を製造するには、極めて高純度のシリコンが必要です(図\ref{fig:impurity})。

\begin{figure}[H]
\centering
\fbox{\includegraphics[width=0.95\textwidth]{chapters/chapter02/images/page-20.pdf}}
\caption{半導体の不純物濃度とその純度}
\label{fig:impurity}
\end{figure}

シリコンの純度は99.999999999\%(イレブンナイン、11N)にも達します。これは、1兆個の原子のうち、不純物原子がわずか1個という驚異的な純度です。

シリコンの原子密度は約$5 \times 10^{22}$個/cm$^3$です。これに対して、制御して添加する不純物濃度は、$10^{13}$個/cm$^3$から$10^{20}$個/cm$^3$程度まで変化させることができます。

具体的な割合を計算してみましょう。

\begin{itemize}
\item 不純物濃度が$10^{13}$個/cm$^3$の場合:
\begin{equation}
\frac{10^{13}}{5 \times 10^{22}} = 2 \times 10^{-10} = 0.00000002\%
\end{equation}

\item 不純物濃度が$10^{20}$個/cm$^3$の場合:
\begin{equation}
\frac{10^{20}}{5 \times 10^{22}} = 0.002 = 0.2\%
\end{equation}
\end{itemize}

このように、ごくわずかな不純物でも、半導体の電気的性質は大きく変化します。これが半導体の特徴であり、電気伝導性を精密に制御できる理由です。

\subsection{ドーピングとは}

半導体に不純物を意図的に添加することを\textbf{ドーピング}(doping)と呼びます。ドーピングにより、半導体のキャリア濃度と電気伝導性を自在に制御することができます。

ドーピングに使われる不純物には、以下の2種類があります。

\begin{itemize}
\item \textbf{ドナー(Donor)}:電子を供給する不純物(5価元素)
\item \textbf{アクセプタ(Acceptor)}:正孔を供給する不純物(3価元素)
\end{itemize}

\subsection{n型半導体}

シリコン(4価)に対して、5価の元素(リン:P、ヒ素:As、アンチモン:Sbなど)を添加すると、\textbf{n型半導体}ができます。

5価元素は、シリコン結晶中で4つの電子を共有結合に使い、1つの電子が余ります。この余った電子は、容易に伝導帯に励起されます。

ドナーのイオン化エネルギーは非常に小さい(シリコン中のリンで約45 meV)ため、室温ではほとんどのドナー原子がイオン化し、電子を供給します。

n型半導体では、以下の関係が成り立ちます。
\begin{itemize}
\item \textbf{多数キャリア}:電子(濃度:$n$)
\item \textbf{少数キャリア}:正孔(濃度:$p$)
\item $n \gg p$
\item $n \approx N_d$($N_d$:ドナー濃度)
\end{itemize}

\subsection{p型半導体}

シリコン(4価)に対して、3価の元素(ホウ素:B、ガリウム:Ga、インジウム:Inなど)を添加すると、\textbf{p型半導体}ができます。

3価元素は、シリコン結晶中で4つの結合を作ろうとしますが、電子が1つ不足します。この不足した電子の位置は正孔となり、価電子帯に正孔を供給します。

p型半導体では、以下の関係が成り立ちます。
\begin{itemize}
\item \textbf{多数キャリア}:正孔(濃度:$p$)
\item \textbf{少数キャリア}:電子(濃度:$n$)
\item $p \gg n$
\item $p \approx N_a$($N_a$:アクセプタ濃度)
\end{itemize}

\subsection{キャリア濃度の積}

n型半導体でもp型半導体でも、電子と正孔の濃度の積は一定です。この関係を\textbf{質量作用の法則}といいます。

\begin{equation}
n \cdot p = n_i^2
\end{equation}

ここで、$n_i$は真性キャリア濃度です。室温のシリコンでは、$n_i \approx 1.5 \times 10^{10}$ 個/cm$^3$です。

例えば、n型半導体で$N_d = 10^{16}$ 個/cm$^3$の場合、
\begin{align}
n &\approx N_d = 10^{16} \text{ 個/cm}^3 \\
p &= \frac{n_i^2}{n} = \frac{(1.5 \times 10^{10})^2}{10^{16}} \approx 2.25 \times 10^{4} \text{ 個/cm}^3
\end{align}

このように、多数キャリアの濃度は不純物濃度とほぼ等しく、少数キャリアの濃度は非常に小さくなります。


\section{pn接合の物理}

\subsection{pn接合とは}

p型半導体とn型半導体を接合したものを\textbf{pn接合}と呼びます。pn接合は、ダイオードの基本構造であり、すべての半導体素子の基礎となる重要な構造です。

pn接合では、接合部で特徴的な現象が起こります。これを理解することが、半導体素子の動作を理解する鍵となります。

\subsection{キャリアの拡散}

p型半導体とn型半導体を接合すると、接合部でキャリアの濃度差が生じます。

\begin{itemize}
\item p型領域:正孔が多数、電子が少数
\item n型領域:電子が多数、正孔が少数
\end{itemize}

濃度差があると、\textbf{拡散}が発生します。拡散とは、濃度の高い場所から低い場所へ粒子が移動する現象です(インクを水に垂らすと広がるのと同じ原理)。

したがって、以下の拡散が起こります。
\begin{itemize}
\item n型領域の電子がp型領域へ拡散する
\item p型領域の正孔がn型領域へ拡散する
\end{itemize}

\subsection{空乏層の形成}

拡散により、接合部では以下の現象が起こります(図\ref{fig:pn_junction_mechanism})。

\begin{figure}[H]
\centering
\fbox{\includegraphics[width=0.95\textwidth]{chapters/chapter02/images/page-30.pdf}}
\caption{pn接合のバンドが曲がるメカニズム}
\label{fig:pn_junction_mechanism}
\end{figure}

\begin{enumerate}
\item \textbf{キャリアの拡散}:n型領域の電子がp型領域へ拡散し、p型領域の正孔と再結合して消滅します。同様に、p型領域の正孔がn型領域へ拡散し、n型領域の電子と再結合して消滅します。

\item \textbf{固定電荷の出現}:キャリアが拡散すると、以下の固定電荷(イオン)が残ります。
\begin{itemize}
\item n型領域:ドナーイオン($+$)が残る
\item p型領域:アクセプタイオン($-$)が残る
\end{itemize}
この領域を\textbf{空乏層}(depletion layer)と呼びます。空乏層では、移動可能なキャリア(電子や正孔)がほとんど存在しません。

\item \textbf{電場の発生}:固定電荷の分布により、空乏層内に電場が発生します。電場の向きは、n型からp型へ向かいます(正電荷から負電荷へ)。

\item \textbf{電位差の発生}:電場があると、空乏層内に電位差が生じます。この電位差を\textbf{拡散電位}(または\textbf{ビルトイン電位})$V_{bi}$と呼びます。

\item \textbf{バンドの傾き}:電位差により、空乏層内で電子のポテンシャルエネルギーが変化します。これがバンド図では「バンドの傾き(曲がり)」として表現されます。
\end{enumerate}

\textbf{重要なポイント:}
\begin{itemize}
\item バンドを曲げる向きには注意が必要です
\item 電場の向きとバンドが曲がる向きは対応しています
\item n型側のバンドが上がり、p型側のバンドが下がります
\end{itemize}

\subsection{電磁気学的な観点からのpn接合の理解}

pn接合の動作を深く理解するためには、電磁気学の基本原理に立ち返ることが重要です。ここでは、拡散による帯電、電場と電位の分布、電子のエネルギー変化、そしてバンドの曲がりの関係を、電磁気学の観点から詳しく説明します。

\subsubsection{拡散による帯電のメカニズム}

pn接合を形成すると、キャリアの濃度勾配により以下のプロセスが進行します:

\begin{enumerate}
\item \textbf{電子の拡散}:n型領域の電子はp型領域へ拡散し、p型領域の正孔と再結合して消滅します。
\item \textbf{正孔の拡散}:p型領域の正孔はn型領域へ拡散し、n型領域の電子と再結合して消滅します。
\item \textbf{固定電荷(イオン)の出現}:
\begin{itemize}
\item n型領域では、電子が拡散により失われると、動けないドナーイオン($\text{P}^+$など)が正電荷として残る
\item p型領域では、正孔が拡散により失われると、動けないアクセプタイオン($\text{B}^-$など)が負電荷として残る
\end{itemize}
\end{enumerate}

この過程で、\textbf{接合部付近に正負の固定電荷層が形成}されます。これが空乏層です。

\textbf{重要な理解:}
\begin{itemize}
\item 拡散するのは「移動可能なキャリア」(電子と正孔)
\item 残るのは「固定された不純物イオン」
\item このイオン化した不純物が空乏層内の電荷分布を形成する
\end{itemize}

\subsubsection{電荷分布と電場の形成}

空乏層内の電荷分布は、以下のようになります:

\begin{itemize}
\item \textbf{n型側の空乏層}:正の固定電荷(密度$+eN_d$)
\item \textbf{p型側の空乏層}:負の固定電荷(密度$-eN_a$)
\end{itemize}

この電荷分布により、\textbf{ポアソン方程式}に従って電場が発生します:

\begin{equation}
\frac{d\mathcal{E}}{dx} = \frac{\rho(x)}{\varepsilon}
\end{equation}

ここで、$\mathcal{E}$は電場、$\rho(x)$は電荷密度、$\varepsilon$は誘電率です。

空乏層内では:
\begin{align}
\text{n型側:} \quad \rho(x) &= +eN_d \\
\text{p型側:} \quad \rho(x) &= -eN_a
\end{align}

したがって、電場は空乏層の境界(接合面)で最大となり、\textbf{n型からp型へ向かう方向}に電場が形成されます。

\subsubsection{電場と電位の関係}

電場$\mathcal{E}$と電位$\phi$の関係は、電磁気学の基本式から:

\begin{equation}
\mathcal{E} = -\frac{d\phi}{dx}
\end{equation}

電場がn型からp型へ向かう($\mathcal{E} > 0$)ので、電位は\textbf{n型側で高く、p型側で低く}なります。

電位差(拡散電位)$V_{bi}$は、電場を積分して求められます:

\begin{equation}
V_{bi} = -\int_{p\text{型}}^{n\text{型}} \mathcal{E}\, dx = \phi_{n} - \phi_{p}
\end{equation}

ここで、$\phi_n$はn型側の電位、$\phi_p$はp型側の電位です。

\textbf{重要なポイント:}
\begin{itemize}
\item 電場はn型→p型の向き(正電荷から負電荷へ)
\item 電位はn型側が高い($\phi_n > \phi_p$)
\item 拡散電位$V_{bi} = \phi_n - \phi_p > 0$
\end{itemize}

\subsubsection{電子のポテンシャルエネルギーと電位の関係}

電子は負の電荷($-e$)を持つため、電子のポテンシャルエネルギー$U_e$と電位$\phi$の関係は:

\begin{equation}
U_e(x) = -e\phi(x)
\end{equation}

\textbf{符号に注意:}

\begin{itemize}
\item 電位が高い場所(n型側)では、電子のポテンシャルエネルギーは\textbf{低い}
\item 電位が低い場所(p型側)では、電子のポテンシャルエネルギーは\textbf{高い}
\end{itemize}

これは、電子が負電荷であるため、正の電場(高電位)に引き寄せられる(ポテンシャルエネルギーが低くなる)ことを意味します。

\subsubsection{バンド図におけるエネルギーの曲がり}

バンド図は、\textbf{電子のエネルギー状態}を表します。伝導帯端$E_c$と価電子帯端$E_v$は、電子がとりうるエネルギー準位の境界を示します。

電位$\phi(x)$の空間変化により、バンド端は以下のように変化します:

\begin{align}
E_c(x) &= E_{c0} - e\phi(x) \\
E_v(x) &= E_{v0} - e\phi(x)
\end{align}

ここで、$E_{c0}$、$E_{v0}$は電位がゼロの基準点でのバンド端のエネルギーです。

\textbf{バンドが曲がる理由:}

\begin{enumerate}
\item n型側では電位$\phi$が高い → $-e\phi$は負に大きい → バンド端が\textbf{下がる}
\item p型側では電位$\phi$が低い → $-e\phi$は負に小さい → バンド端が\textbf{上がる}
\item 結果として、n型からp型へ移動すると、バンド端が上昇する(バンドが曲がる)
\end{enumerate}

バンドの曲がりの大きさは、拡散電位$V_{bi}$に対応します:

\begin{equation}
\Delta E_c = E_c(\text{n型}) - E_c(\text{p型}) = -e(V_{bi}) = -eV_{bi}
\end{equation}

または、エネルギー単位で:

\begin{equation}
\Delta E_c = eV_{bi} \quad \text{(n型側が低く、p型側が高い)}
\end{equation}

\subsubsection{電位、電子エネルギー、バンド図の対応関係}

以下の表に、電位、電場、電子エネルギー、バンドの関係をまとめます:

\begin{table}[H]
\centering
\caption{pn接合における電磁気学的な量の対応関係}
\begin{tabular}{|l|c|c|}
\hline
\textbf{物理量} & \textbf{n型側} & \textbf{p型側} \\
\hline
電位 $\phi$ & 高い & 低い \\
\hline
電場 $\mathcal{E}$ & \multicolumn{2}{c|}{n型→p型の向き} \\
\hline
電子のポテンシャルエネルギー $U_e = -e\phi$ & 低い & 高い \\
\hline
バンド端 $E_c$, $E_v$ & 低い & 高い \\
\hline
フェルミ準位 $E_F$ & \multicolumn{2}{c|}{水平(一定)} \\
\hline
\end{tabular}
\end{table}

\textbf{まとめ:バンドが曲がるメカニズム}

\begin{enumerate}
\item 拡散により、接合部に固定電荷(イオン)が出現
\item 電荷分布により電場が発生(n型→p型)
\item 電場により電位が分布(n型側が高い)
\item 電位により電子のポテンシャルエネルギーが変化($U_e = -e\phi$)
\item 電子エネルギーの変化がバンド図に反映される(バンドが曲がる)
\item 熱平衡では、フェルミ準位が全体で一定(水平)になる
\end{enumerate}

この一連のプロセスを理解することで、pn接合の電気的特性(整流作用、容量特性など)を定量的に予測できるようになります。

\subsection{平衡状態のpn接合}

図\ref{fig:pn_junction_initial}に、pn接合を接続した直後(平衡状態)のバンド図を示します。

\begin{figure}[H]
\centering
\fbox{\includegraphics[width=0.95\textwidth]{chapters/chapter02/images/page-35.pdf}}
\caption{pn接合(接続直後の平衡状態)}
\label{fig:pn_junction_initial}
\end{figure}

平衡状態では、以下の2つの電流が釣り合っています。

\begin{itemize}
\item \textbf{拡散電流}:濃度勾配によってキャリアが移動する電流
\begin{itemize}
\item 電子の拡散電流:n型からp型へ(図のマゼンタの矢印)
\item 正孔の拡散電流:p型からn型へ(図のマゼンタの矢印)
\end{itemize}

\item \textbf{ドリフト電流}:電場によってキャリアが移動する電流
\begin{itemize}
\item 電子のドリフト電流:p型からn型へ(図のグレーの矢印)
\item 正孔のドリフト電流:n型からp型へ(図のグレーの矢印)
\end{itemize}
\end{itemize}

平衡状態では、拡散電流とドリフト電流が完全に釣り合い、正味の電流はゼロになります。

\begin{equation}
I_{\text{拡散}} + I_{\text{ドリフト}} = 0
\end{equation}

このとき、少数キャリアの拡散が電気伝導現象の主要な原因となります。これは、pn接合に電圧を印加したときの動作を理解する上で重要なポイントです。

\subsection{拡散電位}

pn接合の拡散電位$V_{bi}$(ビルトイン電位)は、\textbf{接合前のn型半導体とp型半導体のフェルミ準位の差}に対応します。

\textbf{重要な理解:}

拡散電位は、以下のように定義されます:

\begin{equation}
eV_{bi} = E_{F,n}^{\text{(接合前)}} - E_{F,p}^{\text{(接合前)}}
\end{equation}

または電圧で表すと:

\begin{equation}
V_{bi} = \frac{E_{F,n} - E_{F,p}}{e}
\end{equation}

\textbf{符号に注意:}

n型半導体のフェルミ準位$E_{F,n}$は、伝導帯に近い(高い位置)にあり、p型半導体のフェルミ準位$E_{F,p}$は、価電子帯に近い(低い位置)にあります。したがって:

\begin{equation}
E_{F,n} > E_{F,p} \quad \Rightarrow \quad V_{bi} > 0
\end{equation}

\begin{screen}
\textbf{よくある誤解:}\\
$V_{bi} = E_{F,p} - E_{F,n}$とすると負の値になってしまい、物理的におかしいです。\\
正しくは $V_{bi} = (E_{F,n} - E_{F,p})/e > 0$ です。
\end{screen}

\textbf{接合前と接合後の違い:}

\begin{itemize}
\item \textbf{接合前}:n型とp型は独立した系で、異なるフェルミ準位を持つ
\item \textbf{接合後(熱平衡)}:フェルミ準位は全体で一定(水平)
\item \textbf{拡散電位}:接合前のフェルミ準位の差が、接合後のバンドの曲がり(拡散電位)に変換される
\end{itemize}

pn接合の拡散電位$V_{bi}$は、以下の式で表されます。

\begin{equation}
V_{bi} = \frac{kT}{e} \ln\left(\frac{N_a N_d}{n_i^2}\right)
\end{equation}

ここで、
\begin{itemize}
\item $k$:ボルツマン定数($8.617 \times 10^{-5}$ eV/K)
\item $T$:絶対温度 [K]
\item $e$:電気素量($1.602 \times 10^{-19}$ C)
\item $N_a$:アクセプタ濃度 [個/cm$^3$]
\item $N_d$:ドナー濃度 [個/cm$^3$]
\item $n_i$:真性キャリア濃度 [個/cm$^3$]
\end{itemize}

室温($T = 300$ K)のシリコンで、$N_a = N_d = 10^{16}$ 個/cm$^3$の場合、
\begin{align}
V_{bi} &= \frac{0.026 \text{ V}}{1} \ln\left(\frac{10^{16} \times 10^{16}}{(1.5 \times 10^{10})^2}\right) \\
&= 0.026 \times \ln(4.44 \times 10^{11}) \\
&\approx 0.026 \times 27 \\
&\approx 0.7 \text{ V}
\end{align}

シリコンのpn接合の拡散電位は、通常0.6〜0.7 V程度です。

\subsection{空乏層幅}

空乏層の幅$d$は、不純物濃度に依存します(図\ref{fig:depletion_width})。

\begin{figure}[H]
\centering
\fbox{\includegraphics[width=0.95\textwidth]{chapters/chapter02/images/page-40.pdf}}
\caption{不純物濃度と空乏層幅の関係}
\label{fig:depletion_width}
\end{figure}

空乏層幅は、以下の式で表されます(演習問題で導出します)。

\begin{equation}
d = \sqrt{\frac{2\varepsilon V_{bi}}{e}\left(\frac{1}{N_a} + \frac{1}{N_d}\right)}
\end{equation}

ここで、
\begin{itemize}
\item $\varepsilon$:半導体の誘電率(シリコン:$\varepsilon = 11.7 \varepsilon_0 \approx 1.04 \times 10^{-12}$ F/cm)
\item $V_{bi}$:拡散電位 [V]
\item $e$:電気素量($1.602 \times 10^{-19}$ C)
\item $N_d$:ドナー濃度 [個/cm$^3$]
\item $N_a$:アクセプタ濃度 [個/cm$^3$]
\end{itemize}

\textbf{重要な性質:}
\begin{itemize}
\item 空乏層幅は、不純物濃度の平方根に反比例する
\item 不純物濃度が高いほど、空乏層幅は狭くなる
\item 空乏層は、不純物濃度の低い側により広く伸びる
\end{itemize}

例えば、$N_a = 10^{18}$ 個/cm$^3$、$N_d = 10^{16}$ 個/cm$^3$の場合、p型側の空乏層幅は、n型側の空乏層幅の約$\sqrt{10^{18}/10^{16}} = 10$倍小さくなります。

\subsubsection{空乏層幅の式の近似}

教科書や資料によっては、空乏層幅の式が以下のように簡略化されている場合があります:

\begin{equation}
d = \sqrt{\frac{4\varepsilon V_{bi}}{eN_a}}
\end{equation}

この式は、一般式と比較すると係数が異なります。どのような近似からこの式が導かれるのでしょうか?

一般式と簡略式を比較すると:

\begin{align}
\text{一般式:} \quad d &= \sqrt{\frac{2\varepsilon V_{bi}}{e}\left(\frac{1}{N_a} + \frac{1}{N_d}\right)} \\
\text{簡略式:} \quad d &= \sqrt{\frac{4\varepsilon V_{bi}}{eN_a}}
\end{align}

両式が等しいためには:

\begin{equation}
\frac{1}{N_a} + \frac{1}{N_d} = \frac{2}{N_a}
\end{equation}

この式を変形すると:

\begin{align}
\frac{1}{N_d} &= \frac{2}{N_a} - \frac{1}{N_a} = \frac{1}{N_a} \\
&\Rightarrow \quad \boxed{N_d = N_a}
\end{align}

\textbf{答え:対称接合(symmetric junction)の近似}

簡略式は、\textbf{$N_a = N_d$(対称接合)を仮定}しています。

\textbf{対称接合の物理的意味:}

\begin{verbatim}
p型        |        n型
Na         |        Nd
           |
空乏層     |     空乏層
←─ xp ─→|←─ xn ─→
\end{verbatim}

$N_a = N_d$のとき:

\begin{itemize}
\item 電荷の中性条件:$N_a \cdot x_p = N_d \cdot x_n$
\item したがって:$x_p = x_n$
\item \textbf{空乏層が両側に均等に広がる}
\end{itemize}

\textbf{計算の確認:}

\begin{align}
\frac{1}{N_a} + \frac{1}{N_d} &= \frac{1}{N_a} + \frac{1}{N_a} = \frac{2}{N_a} \\
d &= \sqrt{\frac{2\varepsilon V_{bi}}{e} \cdot \frac{2}{N_a}} = \sqrt{\frac{4\varepsilon V_{bi}}{eN_a}}
\end{align}

\textbf{その他の近似:片側接合(one-sided junction)}

もし一方のドーピング濃度が他方よりもはるかに高い場合(例:$N_d \gg N_a$):

\begin{equation}
\frac{1}{N_a} + \frac{1}{N_d} \approx \frac{1}{N_a} \quad (\text{$N_d \gg N_a$のとき})
\end{equation}

このとき:

\begin{equation}
d \approx \sqrt{\frac{2\varepsilon V_{bi}}{eN_a}}
\end{equation}

この場合、空乏層はほとんど低濃度側(p型側)に広がります。

\textbf{実際のデバイス設計:}

\begin{itemize}
\item \textbf{対称接合}($N_a = N_d$):教科書の例題や理論計算で使われる
\item \textbf{片側接合}($N_d \gg N_a$または$N_a \gg N_d$):実際のデバイスでよく使われる設計
\end{itemize}

実用デバイスでは、片側接合の方が一般的です。例えば、パワーダイオードでは、高耐圧を得るために低濃度のn型ドリフト層($N_d \approx 10^{14}$ cm$^{-3}$)に高濃度のp型層($N_a \approx 10^{19}$ cm$^{-3}$)を接合します。

\subsection{バンド図におけるエネルギー関係}

バンド図を描く際、以下の関係を覚えておくと便利です(図\ref{fig:band_energy})。

\begin{figure}[H]
\centering
\fbox{\includegraphics[width=0.95\textwidth]{chapters/chapter02/images/page-25.pdf}}
\caption{バンド理論におけるエネルギーの関係}
\label{fig:band_energy}
\end{figure}

\begin{itemize}
\item 電子のエネルギーは、バンド図で上に行くほど高い
\item 電場の向きは、バンドが上がる向き(正電荷から負電荷へ)
\item 電位は、バンドが下がる方が高い(電子にとってのポテンシャルエネルギーが低い=電位が高い)
\item フェルミ準位は、平衡状態では全体で一定
\end{itemize}

\subsubsection{フェルミ準位とバンド端の本質的な違い}

pn接合や金属-半導体接合を理解する上で、\textbf{「フェルミ準位は曲がらず、バンドが曲がる」}という原理は非常に重要です。しかし、多くの学生がこの違いを混同しやすいため、ここで本質的な違いを明確にします。

\textbf{バンド端($E_c$、$E_v$)とは:}

\begin{itemize}
\item \textbf{電子のエネルギー固有状態}を表す
\item 電子が実際に占有できるエネルギー準位の境界
\item 静電ポテンシャル$\phi(x)$の影響を直接受ける
\item \textbf{局所的な量}(場所ごとに定義される)
\item 例:$E_c(x) = E_{c0} - e\phi(x)$(電子のポテンシャルエネルギー)
\end{itemize}

\textbf{フェルミ準位($E_F$)とは:}

\begin{itemize}
\item \textbf{電子の化学ポテンシャル}(熱力学的な量)
\item 電子がどちらの方向に移動したがるかを決める基準
\item 物質間で電子をやり取りする際の平衡条件
\item \textbf{熱平衡状態では空間的に一定}(水平)
\end{itemize}

\textbf{なぜフェルミ準位は曲がらない(水平になる)のか:}

熱平衡状態では、系全体でフェルミ準位が一定になります。これは熱力学の基本原理です。

\begin{screen}
\textbf{重要な関係:}\\
熱平衡 $\Leftrightarrow$ フェルミ準位が空間的に一定
\end{screen}

もしフェルミ準位に傾き(空間的な差)があったら:

\begin{enumerate}
\item 場所Aと場所Bで $E_F(A) > E_F(B)$ となる
\item 電子が高いエネルギー(A)から低いエネルギー(B)へ移動する
\item 電流が流れる
\item これは熱平衡ではない!
\end{enumerate}

したがって、熱平衡状態では電流がゼロであり、フェルミ準位は空間的に一定(水平)となります。

\textbf{水の高さによる類推:}

この概念を理解するために、水の高さで例えてみましょう:

\begin{itemize}
\item \textbf{バンド端} = 容器の底の形(場所によって高さが違う)
\item \textbf{フェルミ準位} = 水面の高さ(つながっていれば同じ高さ)
\end{itemize}

\begin{verbatim}
容器A    容器B
┌─┐    ┌──┐
│~│────│~ │  ←水面(フェルミ準位)は同じ高さ
│~~~~~~~│
└──┘  └───┘  ←底の形(バンド)は違う
\end{verbatim}

容器の底の形が違っても、つながっている水面は同じ高さになります。もし水面に高低差があれば、水が流れて平衡状態ではありません。

\textbf{接合時の変化:}

pn接合や金属-半導体接合を形成すると:

\begin{enumerate}
\item 電子が移動する(高いフェルミ準位から低いフェルミ準位へ)
\item 界面に電荷が蓄積される
\item 電場が発生する
\item \textbf{バンド端が曲がる}(電場によるポテンシャルエネルギーの変化)
\item その結果、\textbf{フェルミ準位が全体で一定}になる
\end{enumerate}

\begin{table}[H]
\centering
\caption{フェルミ準位とバンド端の比較}
\begin{tabular}{|l|c|c|}
\hline
\textbf{性質} & \textbf{バンド端($E_c$, $E_v$)} & \textbf{フェルミ準位($E_F$)} \\
\hline
物理的意味 & エネルギー固有状態 & 化学ポテンシャル \\
\hline
ポテンシャルの影響 & 受ける(曲がる) & 受けない \\
\hline
熱平衡での振る舞い & 場所で変化可能 & 空間的に一定 \\
\hline
\end{tabular}
\end{table}

\textbf{まとめ:}

\begin{itemize}
\item フェルミ準位の水平性 = 熱平衡 = 電流ゼロ
\item バンドの曲がりは、電場による電子のポテンシャルエネルギーの変化を表す
\item この区別を理解することが、pn接合や金属-半導体接合を正しく理解する鍵となる
\end{itemize}


\section{金属-半導体接合}

半導体素子では、必ず金属と半導体が接合しています(図\ref{fig:metal_semiconductor_necessity})。金属は電極として、外部回路との電気的接続を提供します。したがって、金属-半導体接合の理解は、実際のデバイスの動作を理解する上で不可欠です。

\begin{figure}[H]
\centering
\fbox{\includegraphics[width=0.95\textwidth]{chapters/chapter02/images/page-42.pdf}}
\caption{金属と半導体の接合(回路素子では必ず金属と半導体が接合している)}
\label{fig:metal_semiconductor_necessity}
\end{figure}

金属-半導体接合には、主に2つのタイプがあります:
\begin{enumerate}
\item \textbf{整流性接触}:一方向にのみ電流を流す性質を持つ接合(ショットキー接合)
\item \textbf{抵抗性接触}:低抵抗で双方向に電流を流す接合(オーミック接合)
\end{enumerate}

半導体素子の配線には抵抗性接触(オーミック接合)が用いられています。

\subsection{金属と半導体のバンド構造の違い}

金属-半導体接合を理解するために、まず金属と半導体のバンド構造の違いを理解する必要があります。図\ref{fig:metal_semiconductor_band}に、金属、n型半導体、p型半導体を接合する前の独立した状態におけるバンド図を示します。

\begin{figure}[H]
\centering
\fbox{\includegraphics[width=0.95\textwidth]{chapters/chapter02/images/page-43.pdf}}
\caption{金属と半導体の接合前のバンド図(接合前の独立した平衡状態)}
\label{fig:metal_semiconductor_band}
\end{figure}

この図は、金属、n型半導体、p型半導体がそれぞれ独立して存在している状態を示しています。\textbf{接合前の状態では、各物質はそれぞれ独自のフェルミ準位を持っています}。ただし、真空準位$E_{\text{vacuum}}$は、全ての物質で共通の基準として定義されます。

\subsubsection{接合前の金属のバンド構造}

図\ref{fig:metal_semiconductor_band}の左側に示す金属のバンド構造には、以下の特徴があります:

\begin{itemize}
\item \textbf{価電子帯と伝導帯が重なっている}:金属では、価電子帯の上端と伝導帯の下端が重なっているか、または伝導帯に電子が部分的に満たされています

\item \textbf{フェルミ準位が伝導帯内にある}:金属のフェルミ準位$E_F$は、電子で満たされた領域の上端に位置します(図の黒い点で示された領域)。これは、室温でも多数の自由電子が存在することを意味します

\item \textbf{バンドギャップがない}:金属には半導体のようなバンドギャップ$E_g$が存在しません。そのため、電子は容易に移動でき、高い電気伝導性を持ちます

\item \textbf{多数の自由電子}:金属中の自由電子密度は非常に高く(約$10^{22}$〜$10^{23}$ 個/cm$^3$)、半導体の真性キャリア濃度(約$10^{10}$ 個/cm$^3$)と比べて10兆倍以上も多い

\item \textbf{仕事関数$\phi_m$}:金属から電子を取り出すために必要なエネルギーは、真空準位$E_{\text{vacuum}}$からフェルミ準位$E_F$までのエネルギー差として定義されます(図の青字で説明されている「物質の束縛から完全に解放するためのエネルギー」)
\end{itemize}

\subsubsection{接合前のn型半導体のバンド構造}

図\ref{fig:metal_semiconductor_band}の中央に示すn型半導体のバンド構造には、以下の特徴があります:

\begin{itemize}
\item \textbf{バンドギャップ$E_g$が存在}:伝導帯下端$E_c$と価電子帯上端$E_v$の間にバンドギャップが存在します(黄色の領域が伝導帯、オレンジ色の領域が価電子帯)

\item \textbf{フェルミ準位$E_F$は伝導帯$E_c$に近い}:n型半導体では、ドーピングによりフェルミ準位が伝導帯側に移動しています。これは、多数キャリアが電子であることを意味します

\item \textbf{電子親和力$\chi_s$}:真空準位$E_{\text{vacuum}}$から伝導帯下端$E_c$までのエネルギー差として定義されます。電子親和力は、半導体の種類(例:シリコン)で決まる固有の値であり、ドーピングには依存しません

\item \textbf{仕事関数$\phi_s$}:真空準位$E_{\text{vacuum}}$からフェルミ準位$E_F$までのエネルギー差として定義されます。仕事関数は、ドーピング濃度によって変化します
\end{itemize}

\subsubsection{接合前のp型半導体のバンド構造}

図\ref{fig:metal_semiconductor_band}の右側に示すp型半導体のバンド構造には、以下の特徴があります:

\begin{itemize}
\item \textbf{バンドギャップ$E_g$が存在}:n型半導体と同様に、伝導帯と価電子帯の間にバンドギャップが存在します

\item \textbf{フェルミ準位$E_F$は価電子帯$E_v$に近い}:p型半導体では、ドーピングによりフェルミ準位が価電子帯側に移動しています。これは、多数キャリアが正孔であることを意味します

\item \textbf{電子親和力$\chi_s$はn型と同じ}:同じ材料(例:シリコン)であれば、電子親和力は不純物の種類によらず一定です

\item \textbf{仕事関数$\phi_s$はn型と異なる}:フェルミ準位の位置が異なるため、p型半導体の仕事関数はn型半導体より大きくなります
\end{itemize}

\subsubsection{接合前の状態における重要なポイント}

接合する前の状態では、以下のことが重要です:

\begin{enumerate}
\item \textbf{真空準位は全ての物質で共通}:真空準位$E_{\text{vacuum}}$は、全ての物質に共通の基準エネルギーです

\item \textbf{フェルミ準位は物質ごとに異なる}:金属、n型半導体、p型半導体では、それぞれ異なるフェルミ準位を持っています

\item \textbf{電子親和力は材料固有の値}:同じ半導体材料(例:シリコン)であれば、n型でもp型でも電子親和力$\chi_s$は同じです

\item \textbf{仕事関数はフェルミ準位の位置で決まる}:仕事関数$\phi$は、真空準位からフェルミ準位までのエネルギー差なので、フェルミ準位の位置によって変わります
\begin{itemize}
\item n型半導体:$\phi_s = \chi_s + (E_c - E_F)$(小さい)
\item p型半導体:$\phi_s = \chi_s + E_g - (E_F - E_v)$(大きい)
\end{itemize}
\end{enumerate}

\subsubsection{接合によって起こること:フェルミ準位はどのように一致するのか?}

これらの異なるフェルミ準位を持つ物質を接合すると、熱平衡状態において系全体でフェルミ準位が一致します。しかし、\textbf{「フェルミ準位が一致する」とは、どういうメカニズムで起こるのでしょうか?}

\textbf{重要な誤解を避けるために:}

「フェルミ準位を一致させるために電子が移動する」という表現は、結果としては正しいのですが、\textbf{フェルミ準位自体が移動するわけではありません}。正確には、以下のプロセスが起こります。

\textbf{ステップ1:電子の移動}

接合直後、金属と半導体のフェルミ準位が異なる場合、電子はエネルギーの高い方から低い方へ移動しようとします。

例:金属の仕事関数$\phi_m$がn型半導体の仕事関数$\phi_s$より大きい場合($\phi_m > \phi_s$)
\begin{itemize}
\item 金属のフェルミ準位は半導体のフェルミ準位より\textbf{低い}位置にある(仕事関数が大きい=フェルミ準位が低い)
\item エネルギーの高い半導体の電子が、エネルギーの低い金属へ移動する
\item つまり、\textbf{n型半導体から金属へ電子が移動}する
\end{itemize}

\textbf{ステップ2:界面への電荷の蓄積}

電子が移動すると、界面付近に電荷が蓄積されます。
\begin{itemize}
\item n型半導体側:電子が抜けた後、ドナーイオン(正の固定電荷)が残る
\item 金属側:電子が蓄積され、負に帯電する
\end{itemize}

\textbf{ステップ3:電場の発生}

界面に電荷が蓄積されると、正電荷から負電荷へ向かう電場$\vec{E}$が発生します。この電場は、さらなる電子の移動を妨げる向きに作用します。

\textbf{ステップ4:バンドが曲がる}

電場が存在すると、電子のポテンシャルエネルギーが空間的に変化します。バンド図では、これが\textbf{「バンドが曲がる」}として表現されます。

\begin{itemize}
\item 電場の向き:n型半導体→金属(正電荷から負電荷へ)
\item 電子にとってのポテンシャルエネルギー:電場と逆向きに高くなる
\item したがって、n型半導体の界面付近でバンドが\textbf{上方に曲がる}
\end{itemize}

電位差$V$と電場$E$の関係は:
\begin{equation}
E = -\frac{dV}{dx}
\end{equation}

電子のポテンシャルエネルギー$U$は、電位$V$と電気素量$e$を用いて:
\begin{equation}
U = -eV
\end{equation}

バンド図における伝導帯下端$E_c$のエネルギーは、電子のポテンシャルエネルギーに対応するため、電位が高い場所ほどバンドは\textbf{下に}位置します(電子にとってエネルギーが低い=安定)。

\textbf{ステップ5:フェルミ準位が一致する}

バンドが曲がり続けると、ある時点で電子の移動が停止します。これは、バンドの曲がりによって生じたポテンシャル障壁が、フェルミ準位の差を相殺するからです。

熱平衡状態では、\textbf{系全体でフェルミ準位が一定}となります。これは、バンドが曲がることで実現されます。

\textbf{金属と半導体のどちらのフェルミ準位が変化するのか?}

ここで重要な質問は、「金属のフェルミ準位が変化するのか?半導体のフェルミ準位が変化するのか?」です。

答えは:\textbf{どちらも変化しますが、変化量が大きく異なります}

この非対称性の理由は、\textbf{キャリア密度(状態密度)の圧倒的な違い}にあります。

\begin{screen}
\textbf{キャリア密度の比較:}
\begin{itemize}
\item 金属:自由電子密度 $\approx 10^{22}$〜$10^{23}$ 個/cm$^3$
\item n型半導体:電子密度 $\approx 10^{16}$〜$10^{18}$ 個/cm$^3$
\end{itemize}
$\rightarrow$ \textbf{金属は半導体の約$10^5$〜$10^7$倍の電子を持つ}
\end{screen}

\begin{itemize}
\item \textbf{金属のフェルミ準位}:
\begin{itemize}
\item 金属中の自由電子密度は非常に高い($\sim 10^{22}$〜$10^{23}$ 個/cm$^3$)
\item \textbf{電子の「海」から少し出ていくだけ}
\item $10^{23}$個のうちの$10^{16}$個程度 $\rightarrow$ 無視できる変化
\item わずかな電子の出入りでは、フェルミ準位はほとんど変化しない
\item 金属内部では、フェルミ準位は接合前とほぼ同じ位置にある
\item \textbf{金属は「フェルミ準位の基準」として振る舞う}
\end{itemize}

\item \textbf{半導体のフェルミ準位}:
\begin{itemize}
\item 半導体中のキャリア密度は比較的低い($\sim 10^{10}$〜$10^{16}$ 個/cm$^3$)
\item \textbf{もともと電子が少ない}
\item 相対的に大きな割合の電子が出入り
\item 電子の出入りにより、\textbf{界面付近でバンド全体が曲がる}
\item 半導体の内部(バルク)では、フェルミ準位は接合前と同じ位置にある
\item しかし、界面付近ではバンドが曲がっているため、見かけ上「フェルミ準位が変化した」ように見える
\item \textbf{半導体側が金属に「合わせに行く」}
\end{itemize}
\end{itemize}

\textbf{水槽とコップの例え:}

この状況を、大きな水槽と小さなコップを繋いだときの水の移動で例えることができます。

\begin{verbatim}
大きな水槽(金属)    小さなコップ(半導体)
┌─────────┐      ┌──┐
│~~~~~~│──→──│    │
│~~~~~~│ 同量  │~~ │
│~~~~~~│  移動  │~~ │
└─────────┘      └──┘
水位ほぼ不変          水位大変化!
\end{verbatim}

接合時に同じ量の電子が移動しても:
\begin{itemize}
\item \textbf{金属側}:膨大な電子の「海」から少量が移動するだけ → フェルミ準位はほぼ不変
\item \textbf{半導体側}:相対的に少ない電子から同量が移動 → フェルミ準位が大きく変化(バンドが大きく曲がる)
\end{itemize}

\textbf{正確な表現:}

したがって、より正確には:
\begin{itemize}
\item \textbf{フェルミ準位自体は移動しない}(各物質の内部では接合前と同じ)
\item \textbf{電子が移動する}(高エネルギー側から低エネルギー側へ)
\item \textbf{界面に電荷が蓄積される}
\item \textbf{電場が発生する}
\item \textbf{半導体側のバンドが曲がる}(金属側はほとんど変化しない)
\item \textbf{バンドが曲がった結果、熱平衡状態では系全体でフェルミ準位が一定になる}
\end{itemize}

これが、次節以降で詳しく説明する金属-半導体接合の形成メカニズムです。図2.16〜2.19で示されるバンド図は、このメカニズムによってバンドが曲がった後の平衡状態を表しています。

\subsection{仕事関数}

金属や半導体から電子を取り出すために必要なエネルギーを\textbf{仕事関数}(work function)$\phi$と呼びます。仕事関数は、真空準位からフェルミ準位までのエネルギー差として定義されます。

\begin{equation}
\phi = E_{\text{vacuum}} - E_F
\end{equation}

図\ref{fig:metal_semiconductor_band}の左側の注釈に示されているように、仕事関数は「物質の束縛から完全に解放するためのエネルギー」です。

代表的な物質の仕事関数を以下に示します。
\begin{itemize}
\item 金(Au): $\phi_m \approx 5.1$ eV
\item アルミニウム(Al): $\phi_m \approx 4.3$ eV
\item タングステン(W): $\phi_m \approx 4.5$ eV
\item n型シリコン(Si): $\phi_s \approx 4.0$ eV(ドーピング濃度に依存)
\item p型シリコン(Si): $\phi_s \approx 5.0$ eV(ドーピング濃度に依存)
\end{itemize}

\textbf{電子親和力:}

半導体には、仕事関数の他に\textbf{電子親和力}(electron affinity)$\chi_s$という重要なパラメータがあります。電子親和力は、真空準位から伝導帯下端までのエネルギー差として定義されます:

\begin{equation}
\chi_s = E_{\text{vacuum}} - E_c
\end{equation}

シリコンの場合、$\chi_s \approx 4.05$ eVです。電子親和力は、半導体の種類で決まる固有の値であり、ドーピングには依存しません。

仕事関数と電子親和力の関係は、以下のようになります:

\begin{itemize}
\item n型半導体: $\phi_s = \chi_s + (E_c - E_F)$
\item p型半導体: $\phi_s = \chi_s + E_g - (E_F - E_v)$
\end{itemize}

ドーピング濃度によってフェルミ準位$E_F$が変化するため、半導体の仕事関数$\phi_s$もドーピング濃度に依存します。

\subsection{金属-半導体接合の形成}

金属と半導体を接合すると、それぞれの仕事関数の違いにより、電子が移動します。接合後は、フェルミ準位が一致します(熱平衡状態)。

\textbf{重要な原理:}

熱平衡状態では、系全体でフェルミ準位$E_F$が一定となります。これは、pn接合と同じ原理です。フェルミ準位を一致させるために、電子が移動し、界面に電荷が蓄積され、バンドが曲がります。

\textbf{金属-半導体接合の形成過程:}

\begin{enumerate}
\item 接合前:金属と半導体はそれぞれ独立した系で、異なるフェルミ準位を持つ
\item 接合時:電子が移動し、界面に電荷が蓄積される
\item 接合後(平衡状態):フェルミ準位が一致し、電子の移動が停止する
\end{enumerate}

接合のタイプ(整流性かオーム性か)は、\textbf{金属の仕事関数$\phi_m$と半導体の仕事関数$\phi_s$の大小関係}、および\textbf{半導体の型(n型かp型か)}によって決まります。

\subsubsection{金属とn型半導体の接合}

図\ref{fig:metal_n_before}に、金属とn型半導体を接合する前の状態を示します。

\begin{figure}[H]
\centering
\fbox{\includegraphics[width=0.95\textwidth]{chapters/chapter02/images/page-44.pdf}}
\caption{金属とn型半導体の接合前のバンド図(フェルミエネルギーの大小関係で接合タイプが変わる)}
\label{fig:metal_n_before}
\end{figure}

接合前の状態では、左側にn型半導体($\phi_m < \phi_s$)、中央に金属、右側にn型半導体($\phi_m > \phi_s$)が独立して存在しています。それぞれのフェルミ準位は異なる位置にあります。

図\ref{fig:metal_n_after}に、金属とn型半導体を接合した後の状態を示します。

\begin{figure}[H]
\centering
\fbox{\includegraphics[width=0.95\textwidth]{chapters/chapter02/images/page-45.pdf}}
\caption{金属とn型半導体の接合後のバンド図(フェルミ準位が一致する)}
\label{fig:metal_n_after}
\end{figure}

\textbf{ケース1:$\phi_m > \phi_s$の場合(右側):整流性接触}

\begin{enumerate}
\item \textbf{接合前}:金属のフェルミ準位が半導体のフェルミ準位より低い位置にある

\item \textbf{電子の移動}:接合すると、エネルギーの高い位置(半導体のフェルミ準位)から低い位置(金属のフェルミ準位)へ電子が移動しようとする。つまり、n型半導体から金属へ電子が移動する

\item \textbf{界面の電荷分布}:
\begin{itemize}
\item n型半導体側:電子が抜けた後に、ドナーイオン($+$)が残る
\item 金属側:移動してきた電子により、負に帯電する
\end{itemize}

\item \textbf{空乏層の形成}:n型半導体の界面付近に、キャリアがほとんど存在しない空乏層が形成される。空乏層内には正の固定電荷(ドナーイオン)が存在する

\item \textbf{電場の発生}:正電荷(半導体側)から負電荷(金属側)へ向かう電場$F$が発生する

\item \textbf{バンドの曲がり}:電場により、n型半導体のバンドが界面で上方に曲がる(図\ref{fig:metal_n_after}右側)

\item \textbf{エネルギー障壁の形成}:界面にエネルギー障壁が形成され、電子が半導体から金属へ移動するのを妨げる。この障壁により、\textbf{整流性(一方向のみ電流を流す性質)}が生じる
\end{enumerate}

\textbf{ケース2:$\phi_m < \phi_s$の場合(左側):オーム性接触}

\begin{enumerate}
\item \textbf{接合前}:金属のフェルミ準位が半導体のフェルミ準位より高い位置にある

\item \textbf{電子の移動}:接合すると、エネルギーの高い位置(金属のフェルミ準位)から低い位置(半導体のフェルミ準位)へ電子が移動しようとする。つまり、金属からn型半導体へ電子が移動する

\item \textbf{界面の電荷分布}:
\begin{itemize}
\item n型半導体側:金属から移動してきた電子が蓄積され、負に帯電する
\item 金属側:電子が抜けた後、正に帯電する
\end{itemize}

\item \textbf{電場の発生}:正電荷(金属側)から負電荷(半導体側)へ向かう電場$F$が発生する

\item \textbf{バンドの曲がり}:電場により、n型半導体のバンドが界面で下方に曲がる(図\ref{fig:metal_n_after}左側)

\item \textbf{低抵抗接触}:界面にエネルギー障壁がほとんど存在しないか、非常に薄いため、電子が容易に通過できる。これにより、\textbf{オーム性接触(低抵抗の双方向接触)}となる
\end{enumerate}

\subsubsection{金属とp型半導体の接合}

次に、金属とp型半導体の接合について見ていきます。図\ref{fig:metal_p_before}に、金属とp型半導体を接合する前の状態を示します。

\begin{figure}[H]
\centering
\fbox{\includegraphics[width=0.95\textwidth]{chapters/chapter02/images/page-46.pdf}}
\caption{金属とp型半導体の接合前のバンド図(フェルミエネルギーの大小関係で接合タイプが変わる)}
\label{fig:metal_p_before}
\end{figure}

図\ref{fig:metal_p_after}に、金属とp型半導体を接合した後の状態を示します。

\begin{figure}[H]
\centering
\fbox{\includegraphics[width=0.95\textwidth]{chapters/chapter02/images/page-47.pdf}}
\caption{金属とp型半導体の接合後のバンド図(フェルミ準位が一致する)}
\label{fig:metal_p_after}
\end{figure}

\textbf{ケース3:$\phi_m > \phi_s$の場合(左側):オーム性接触}

\begin{enumerate}
\item \textbf{接合前}:金属のフェルミ準位がp型半導体のフェルミ準位より低い位置にある

\item \textbf{電子の移動}:接合すると、エネルギーの高い位置(p型半導体のフェルミ準位)から低い位置(金属のフェルミ準位)へ電子が移動する。つまり、p型半導体から金属へ電子が移動する

\item \textbf{界面の電荷分布}:
\begin{itemize}
\item p型半導体側:電子が抜けることは、正孔が増えることと等価である。つまり、p型半導体の界面付近に正孔が蓄積される(正に帯電する)
\item 金属側:移動してきた電子により、負に帯電する
\end{itemize}

\item \textbf{電場の発生}:正電荷(半導体側)から負電荷(金属側)へ向かう電場$F$が発生する

\item \textbf{バンドの曲がり}:電場により、p型半導体のバンドが界面で下方に曲がる(図\ref{fig:metal_p_after}左側)

\item \textbf{低抵抗接触}:p型半導体では、多数キャリアは正孔である。界面に正孔が蓄積されるため、電流が流れやすい\textbf{オーム性接触}となる
\end{enumerate}

\textbf{ケース4:$\phi_m < \phi_s$の場合(右側):整流性接触}

\begin{enumerate}
\item \textbf{接合前}:金属のフェルミ準位がp型半導体のフェルミ準位より高い位置にある

\item \textbf{電子の移動}:接合すると、エネルギーの高い位置(金属のフェルミ準位)から低い位置(p型半導体のフェルミ準位)へ電子が移動する。つまり、金属からp型半導体へ電子が移動する

\item \textbf{界面の電荷分布}:
\begin{itemize}
\item p型半導体側:金属から電子が注入されることは、正孔が減少することと等価である。界面付近の正孔が電子と再結合して消滅し、アクセプタイオン($-$)が残る(負に帯電する)
\item 金属側:電子が抜けた後、正に帯電する
\end{itemize}

\item \textbf{空乏層の形成}:p型半導体の界面付近に、キャリア(正孔)がほとんど存在しない空乏層が形成される

\item \textbf{電場の発生}:正電荷(金属側)から負電荷(半導体側)へ向かう電場$F$が発生する

\item \textbf{バンドの曲がり}:電場により、p型半導体のバンドが界面で上方に曲がる(図\ref{fig:metal_p_after}右側)

\item \textbf{エネルギー障壁の形成}:界面にエネルギー障壁が形成され、正孔が半導体から金属へ移動するのを妨げる。この障壁により、\textbf{整流性}が生じる
\end{enumerate}

\subsubsection{n型半導体とp型半導体での接合タイプの違い}

ここで重要なポイントは、\textbf{同じ仕事関数の大小関係でも、n型半導体とp型半導体では接合タイプが逆になる}ということです。

\begin{table}[H]
\centering
\caption{金属-半導体接合のタイプ(仕事関数の大小関係による分類)}
\begin{tabular}{|c|c|c|}
\hline
\textbf{仕事関数の関係} & \textbf{n型半導体との接合} & \textbf{p型半導体との接合} \\
\hline
$\phi_m > \phi_s$ & 整流性接触 & オーム性接触 \\
\hline
$\phi_m < \phi_s$ & オーム性接触 & 整流性接触 \\
\hline
\end{tabular}
\end{table}

\textbf{この違いが生じる理由:}

\begin{itemize}
\item \textbf{n型半導体}では、多数キャリアは電子である。$\phi_m > \phi_s$の場合、電子がn型半導体から金属へ移動し、界面に空乏層(正の固定電荷)が形成され、エネルギー障壁ができる(整流性)。

\item \textbf{p型半導体}では、多数キャリアは正孔である。$\phi_m > \phi_s$の場合、電子がp型半導体から金属へ移動するが、これは正孔が蓄積されることを意味する。正孔は多数キャリアなので、電流が流れやすくなる(オーム性)。

\item つまり、同じ$\phi_m > \phi_s$でも、n型では多数キャリア(電子)が減少して空乏層ができるのに対し、p型では多数キャリア(正孔)が増加して低抵抗接触となる。
\end{itemize}

\textbf{実際のデバイスへの応用:}

\begin{itemize}
\item 半導体素子の電極(配線接続部)には、オーム性接触が必要である
\item n型半導体の電極には、$\phi_m < \phi_s$となる金属を選ぶか、半導体表面を高濃度にドーピングする
\item p型半導体の電極には、$\phi_m > \phi_s$となる金属を選ぶか、半導体表面を高濃度にドーピングする
\item 高濃度ドーピング($> 10^{19}$ 個/cm$^3$)により空乏層幅を極めて薄くし、トンネル効果で電子が通過できるようにすることで、仕事関数の関係によらずオーム性接触を実現できる
\end{itemize}

\subsection{ショットキー接合}

前節で説明したように、金属とn型半導体を接合する際、$\phi_m > \phi_s$の場合に整流性接触(ショットキー接合)が形成されます。この接合は\textbf{ショットキー接合}と呼ばれ、pn接合と同様に整流性(一方向のみ電流を流す性質)を持ちます。

\textbf{ショットキー障壁:}

ショットキー接合では、界面にエネルギー障壁が形成されます。このエネルギー障壁の高さ$\phi_B$を\textbf{ショットキー障壁}と呼び、以下の式で表されます。
\begin{equation}
\phi_B = \phi_m - \chi_s
\end{equation}

ここで、$\chi_s$は半導体の電子親和力(真空準位から伝導帯下端までのエネルギー差)です。

\textbf{各記号の物理的意味:}

ショットキーバリアの式$\phi_B = \phi_m - \chi_s$を理解するために、各記号の意味を明確にします。

\begin{itemize}
\item \textbf{$\phi_m$(金属の仕事関数)}:
\begin{itemize}
\item 真空準位から金属のフェルミ準位までのエネルギー差
\item 金属から電子1個を真空中に取り出すのに必要なエネルギー
\item 例:Al $\approx$ 4.3 eV、Au $\approx$ 5.1 eV
\end{itemize}

\item \textbf{$\chi_s$(半導体の電子親和力)}:
\begin{itemize}
\item 真空準位から半導体の伝導帯端までのエネルギー差
\item 真空中の静止した電子を伝導帯底に入れたときに得られるエネルギー
\item 例:Si $\approx$ 4.05 eV、GaAs $\approx$ 4.07 eV
\item \textbf{重要}:電子親和力は半導体材料固有の値で、ドーピングには依存しない
\end{itemize}

\item \textbf{$\phi_B$(ショットキーバリア高さ)}:
\begin{itemize}
\item 金属のフェルミ準位から半導体の伝導帯端までのエネルギー差
\item 金属から半導体へ電子が移る際の障壁の高さ
\item この値が大きいほど、電子の移動が困難(整流性が強い)
\end{itemize}
\end{itemize}

\textbf{バンド図での理解:}

\begin{verbatim}
真空準位 ─────────────────────
        ↕ φm          ↕ χs
      金属EF        半導体Ec
        │            │
        └─ φB = φm - χs ─┘
\end{verbatim}

\textbf{具体例:Au(金)とn型Si接合}

具体的な数値で計算してみましょう:

\begin{itemize}
\item $\phi_m$(Au) $\approx$ 5.1 eV
\item $\chi_s$(Si) $\approx$ 4.05 eV
\item $\phi_B \approx 5.1 - 4.05 \approx$ \textbf{1.05 eV}
\end{itemize}

この1.05 eVが、金属から半導体への電子注入を妨げる障壁となります。この障壁の高さは、以下のデバイス特性を決定する最も重要なパラメータです:

\begin{itemize}
\item \textbf{整流特性}:障壁の高さが大きいほど、整流性が強くなる
\item \textbf{逆方向リーク電流}:$I_{\text{leak}} \propto \exp(-\phi_B/kT)$で減少
\item \textbf{接触抵抗}:オーム性接触では$\phi_B < 0$となるように設計
\end{itemize}

\textbf{接合タイプとの関係:}

\begin{itemize}
\item \textbf{$\phi_m > \chi_s$の場合}(整流性接触):
\begin{itemize}
\item $\phi_B = \phi_m - \chi_s > 0$ → 正のバリアが存在
\item 電子の流れが制限される
\item ショットキーダイオードとして機能
\end{itemize}

\item \textbf{$\phi_m < \chi_s$の場合}(オーム性接触):
\begin{itemize}
\item $\phi_B = \phi_m - \chi_s < 0$ → バリアが存在しない
\item 電子が金属→半導体へ容易に流れる
\item 低抵抗接触
\end{itemize}
\end{itemize}

\textbf{ショットキー接合の特徴と応用:}

\begin{itemize}
\item \textbf{順方向電圧降下が小さい}:pn接合のダイオードでは順方向電圧降下が約0.7 V(シリコンの場合)であるのに対し、ショットキーバリアダイオードでは約0.3〜0.4 Vと小さい。これにより、導通損失を低減できる

\item \textbf{高速スイッチング}:pn接合では少数キャリアの蓄積効果(リカバリ時間)があるが、ショットキー接合では多数キャリアのみが関与するため、高速スイッチングが可能

\item \textbf{逆方向耐圧が比較的低い}:ショットキー接合の逆方向耐圧は、pn接合に比べて低い(通常100 V以下)。これは、空乏層内の電場集中が大きいためである

\item \textbf{温度特性}:ショットキー障壁は温度に依存し、高温では逆方向漏れ電流が増加する

\item \textbf{主な応用例}:
\begin{itemize}
\item ショットキーバリアダイオード(SBD):低電圧・高速スイッチング用途
\item 整流回路:AC-DC変換器の二次側整流
\item フリーホイールダイオード:インバータ回路の環流ダイオード
\end{itemize}
\end{itemize}

\subsection{オーミック接合}

前節で説明したように、金属と半導体を接合する際、以下の条件でオーム性接触(オーミック接合)が形成されます:
\begin{itemize}
\item n型半導体との接合:$\phi_m < \phi_s$
\item p型半導体との接合:$\phi_m > \phi_s$
\end{itemize}

オーミック接合は低抵抗の接触となり、整流性を持ちません。半導体素子の電極として不可欠です。

\textbf{オーミック接合の特徴:}

\begin{itemize}
\item \textbf{線形な電流-電圧特性}:電流$I$と電圧$V$の関係がオームの法則$V = IR$に従う(線形)

\item \textbf{接触抵抗が非常に小さい}:接触抵抗$R_c$は、通常$10^{-6}$〜$10^{-8}$ $\Omega \cdot$cm$^2$と非常に小さい

\item \textbf{双方向に電流が流れる}:整流性がないため、正負両方向に電流が流れる

\item \textbf{温度安定性が良い}:ショットキー接合に比べて、温度による特性変化が小さい
\end{itemize}

\textbf{オーミック接合の実現方法:}

実際のデバイスでは、以下の方法でオーミック接合を実現します。

\begin{enumerate}
\item \textbf{適切な金属の選択}:
\begin{itemize}
\item n型半導体:仕事関数の小さい金属(例:アルミニウム、チタン)
\item p型半導体:仕事関数の大きい金属(例:金、白金)
\end{itemize}

\item \textbf{高濃度ドーピング}:半導体表面を高濃度にドーピング($> 10^{19}$ 個/cm$^3$)することで、空乏層幅を極めて薄くする(数nm以下)。これにより、電子が\textbf{トンネル効果}で障壁を通過できるようになり、仕事関数の関係によらずオーム性接触を実現できる

\item \textbf{合金化処理}:金属と半導体を高温で熱処理し、界面に金属シリサイド層を形成することで、低抵抗接触を実現する
\end{enumerate}

\textbf{主な応用例:}
\begin{itemize}
\item 半導体素子の電極(ソース、ドレイン、エミッタ、コレクタ電極)
\item 配線接続部
\item ボンディングパッド
\end{itemize}

金属-半導体接合の理解は、実際のパワー半導体デバイスを設計・製造する上で不可欠です。次章では、これらの接合を利用したダイオード、トランジスタ、MOSFETなどの動作原理を学びます。


\section{まとめ}

本章では、パワーエレクトロニクスで使用される半導体素子を理解するために必要な半導体の物理について学びました。

\subsection{重要なポイント}

\begin{enumerate}
\item \textbf{半導体の定義}:
\begin{itemize}
\item 半導体は、電流を流したり流さなかったりすることができる物質
\item 導体と絶縁体の中間的な性質を持つ
\item 外部条件によって電気伝導性を制御できる
\end{itemize}

\item \textbf{原子と電子}:
\begin{itemize}
\item 電子は原子核の周りの軌道上に存在する
\item 各軌道は固有のエネルギーを持つ
\item エネルギーの低い軌道から電子が詰まる
\end{itemize}

\item \textbf{バンド理論}:
\begin{itemize}
\item 固体では、多数の原子が集まることでエネルギー準位がバンドを形成する
\item 価電子帯(電子で満たされている)と伝導帯(電子が移動できる)がある
\item バンドギャップの大きさにより、導体・半導体・絶縁体が決まる
\end{itemize}

\item \textbf{フェルミ・ディラック分布}:
\begin{itemize}
\item 電子がどのエネルギー準位に存在するかを統計的に扱う
\item フェルミエネルギーは、電子が存在する確率が50\%となるエネルギー
\item 室温では熱エネルギーにより、わずかに電子が伝導帯に励起される
\end{itemize}

\item \textbf{ドーピング}:
\begin{itemize}
\item n型半導体:5価元素を添加し、電子が多数キャリア
\item p型半導体:3価元素を添加し、正孔が多数キャリア
\item わずかな不純物でも電気伝導性が大きく変化する
\end{itemize}

\item \textbf{pn接合}:
\begin{itemize}
\item キャリアの拡散により空乏層が形成される
\item 空乏層内に電場が発生し、バンドが曲げられる
\item 拡散電流とドリフト電流が釣り合い、平衡状態となる
\item 空乏層幅は不純物濃度の平方根に反比例する
\end{itemize}

\item \textbf{金属-半導体接合}:
\begin{itemize}
\item 仕事関数の違いにより、ショットキー接合またはオーミック接合が形成される
\item ショットキー接合は整流性を持つ
\item オーミック接合は低抵抗接続となり、電極として使用される
\end{itemize}
\end{enumerate}

\subsection{次回の予告}

次回以降の講義では、以下の内容について学びます。

\begin{itemize}
\item pn接合に電圧を印加したときの動作(順バイアス・逆バイアス)
\item ダイオードの電流-電圧特性
\item トランジスタの動作原理
\item MOSFET、IGBTなどのパワー半導体素子の動作原理
\item 半導体素子の特性と応用
\end{itemize}

本章で学んだ内容は、これらの素子を理解するための基礎となります。特に、バンド図の描き方、空乏層の形成メカニズム、キャリアの拡散とドリフトの概念は重要です。しっかりと復習しておきましょう。

\subsection{演習問題}

\begin{enumerate}
\item シリコンのバンドギャップが1.1 eVであることを使って、室温(300 K)で価電子帯から伝導帯に励起される電子の割合を概算せよ。(ヒント:フェルミ・ディラック分布を使用)

\item n型シリコンで、ドナー濃度が$N_d = 10^{17}$ 個/cm$^3$のとき、少数キャリア(正孔)の濃度を求めよ。真性キャリア濃度は$n_i = 1.5 \times 10^{10}$ 個/cm$^3$とする。

\item pn接合の空乏層幅の式を、ポアソン方程式を使って導出せよ。(ヒント:空乏層内の電荷密度から電場、電位を求め、境界条件を適用する)

\item 金($\phi_m = 5.1$ eV)とn型シリコン($\phi_s = 4.0$ eV)の接合は、ショットキー接合かオーミック接合か答えよ。また、その理由を説明せよ。

\item パワーエレクトロニクスで炭化シリコン(SiC)や窒化ガリウム(GaN)などのワイドバンドギャップ半導体が注目される理由を、バンドギャップの大きさと半導体の性質の関係から説明せよ。
\end{enumerate}


% 第3章:パワー半導体の動作原理
% 第2章 パワー半導体の動作原理

\chapter{パワー半導体の動作原理}

\section{はじめに}

\subsection{本章の目的と学習目標}

本章では、パワーエレクトロニクスで使用される半導体スイッチング素子の動作原理について学びます。前章で学んだ半導体の物理的性質を基礎として、実際のパワー半導体素子がどのように動作するのかを理解することが目標です。

本章の学習目標は以下の通りです:

\begin{enumerate}
\item スイッチング素子の物理を理解する
\item 半導体の構造からスイッチの制御の原理を理解する
\item ワイドギャップ半導体の特徴を理解する
\end{enumerate}

\begin{figure}[H]
\centering
\fbox{\includegraphics[width=0.95\textwidth]{chapters/chapter03/images/page-02.pdf}}
\caption{本日の目標}
\label{fig:objectives}
\end{figure}

\subsection{本章で学ぶスイッチの種類}

パワーエレクトロニクスで使用される主要なスイッチング素子には、以下のようなものがあります:

\begin{itemize}
\item \textbf{ダイオード}:最も基本的な半導体素子で、一方向にのみ電流を流す
\item \textbf{サイリスタ}:制御可能なスイッチング素子
\item \textbf{トランジスタ}:電流または電圧で制御できるスイッチング素子
\item \textbf{IGBT}:絶縁ゲートバイポーラトランジスタ
\item \textbf{MOSFET}:金属酸化膜半導体電界効果トランジスタ
\end{itemize}

\begin{figure}[H]
\centering
\fbox{\includegraphics[width=0.95\textwidth]{chapters/chapter03/images/page-03.pdf}}
\caption{本日習うスイッチの種類}
\label{fig:switch_types}
\end{figure}

これらの素子は、それぞれ異なる特性を持ち、用途に応じて使い分けられます。

\section{半導体スイッチの応用先}

\subsection{用途に応じたスイッチの使い分け}

半導体スイッチは、その出力容量と動作周波数によって使い分けられます。図\ref{fig:applications}に示すように、サイリスタは電鉄や製鉄などの大容量・低周波数の用途に、IGBTは電力機器や自動車などの中容量・中周波数の用途に、MOSFETは電子レンジや洗濯機などの小容量・高周波数の用途に使用されます。

\begin{figure}[H]
\centering
\fbox{\includegraphics[width=0.95\textwidth]{chapters/chapter03/images/page-04.pdf}}
\caption{半導体スイッチの応用先}
\label{fig:applications}
\end{figure}

近年では、SiC(シリコンカーバイド)やGaN(窒化ガリウム)などのワイドギャップ半導体の応用により、さらなる大容量化や高速化が進んでいます。これにより、従来は困難だった領域への応用が可能になっています。

\section{パワー半導体スイッチに求められる機能}

\subsection{電力変換用途での要求特性}

パワー半導体スイッチは、電力変換の用途で使用されるため、以下の特性が求められます:

\begin{enumerate}
\item \textbf{高耐圧}(オフ時):スイッチがオフの状態では、高電圧がかかっても破壊されない耐圧性が必要
\item \textbf{低抵抗}(オン時):スイッチがオンの状態では、大電流が流れても損失が小さい低抵抗性が必要
\end{enumerate}

\begin{figure}[H]
\centering
\fbox{\includegraphics[width=0.95\textwidth]{chapters/chapter03/images/page-05.pdf}}
\caption{パワー半導体スイッチに求められる機能}
\label{fig:requirements}
\end{figure}

オン時の電力損失は、電流$I$とオン抵抗$R_{\text{on}}$の関係から$P = IR_{\text{on}}^2$で表されます。したがって、オン抵抗を小さくすることが、効率の良いスイッチング素子を実現する上で重要です。

\subsection{パワー半導体の耐圧と損失のトレードオフ}

パワー半導体素子の設計においては、耐圧と損失の間にトレードオフの関係があります。図\ref{fig:tradeoff}に示すように、pn接合の耐圧を上げるためにはドーピング濃度を小さくする必要がありますが、一方でドリフト電流による損失を小さくするためにはドーピング濃度を大きくする必要があります。

\begin{figure}[H]
\centering
\fbox{\includegraphics[width=0.95\textwidth]{chapters/chapter03/images/page-06.pdf}}
\caption{パワー半導体の耐圧と損失のトレードオフ}
\label{fig:tradeoff}
\end{figure}

最大電界$F_{\text{max}}$は、空乏層幅$d$と耐圧$V$の関係から以下の式で表されます:

\begin{equation}
F_{\text{max}} = \frac{2V}{d}
\end{equation}

また、空乏層幅$d$は以下の式で計算されます:

\begin{equation}
d = \sqrt{\frac{2\varepsilon V}{e}\left(\frac{1}{N_a} + \frac{1}{N_d}\right)}
\end{equation}

ここで、$\varepsilon$は誘電率、$e$は電荷素量、$N_a$はアクセプタ濃度、$N_d$はドナー濃度です。

右側のグラフは、不純物濃度と抵抗率の関係を示しています。n型半導体とp型半導体では、不純物濃度が増加すると抵抗率が減少することがわかります。

\section{スイッチの制御について}

\subsection{制御可能性による分類}

半導体スイッチは、その制御方法によって分類できます。図\ref{fig:control}に示すように、スイッチの制御には以下の2つのタイプがあります:

\begin{enumerate}
\item \textbf{好きなときに切り替え可能}:電圧または電流で制御できるスイッチ(例:トランジスタ、MOSFET、IGBT)
\item \textbf{ある条件で切り替わる}:特定の条件下でのみオン・オフが切り替わるスイッチ(例:ダイオード)
\end{enumerate}

\begin{figure}[H]
\centering
\fbox{\includegraphics[width=0.95\textwidth]{chapters/chapter03/images/page-07.pdf}}
\caption{スイッチの制御について}
\label{fig:control}
\end{figure}

制御可能なスイッチは、ゲート電圧$v_g$やベース電流$i_g$によって、主電流$i$を制御することができます。

\begin{figure}[H]
\centering
\fbox{\includegraphics[width=0.95\textwidth]{chapters/chapter03/images/page-08.pdf}}
\caption{本日習うスイッチの種類(回路記号)}
\label{fig:switch_symbols}
\end{figure}

\section{ダイオードの動作原理}

\subsection{ダイオードの基本構造}

ダイオードは、pn接合を基本とした最もシンプルな半導体素子です。p型半導体とn型半導体を接合することで、一方向にのみ電流を流す整流作用を実現します。ダイオードは制御端子を持たず、印加される電圧によって自動的にオン・オフが切り替わります。

\subsection{ダイオードのスイッチ特性}

図\ref{fig:diode_characteristics}に、ダイオードの電圧-電流特性を示します。

\begin{figure}[H]
\centering
\fbox{\includegraphics[width=0.95\textwidth]{chapters/chapter03/images/page-09.pdf}}
\caption{ダイオードのスイッチ特性}
\label{fig:diode_characteristics}
\end{figure}

ダイオードの特性は以下のようになります:

\begin{itemize}
\item \textbf{順方向バイアス時}(アノードがカソードより高電位):
\begin{itemize}
\item pn接合の拡散電位(約0.7 V)を超えると電流が流れ始める
\item オン電圧$v_{\text{fwd}} \approx 0.7$ V(シリコンダイオードの場合)
\item 実際のダイオードでは、理想的なゼロ抵抗ではなく、わずかな電圧降下が存在する
\end{itemize}

\item \textbf{逆方向バイアス時}(カソードがアノードより高電位):
\begin{itemize}
\item 空乏層が広がり、電流はほとんど流れない
\item 逆方向電圧が破壊電圧を超えると、絶縁が破れて電流が流れる(破壊)
\end{itemize}
\end{itemize}

右側のグラフは、実際のダイオード(U10LC48)の電圧-電流特性を示しています。温度が高くなると(150℃)、順方向電圧が小さくなることがわかります。これは、温度上昇により真性キャリア濃度が増加するためです。

\textbf{重要なポイント:}

pn接合のエネルギー障壁は完全にゼロにはならないため、ダイオードには必ず順方向電圧降下が存在します。この電圧降下は、ダイオードの損失の原因となります。

\section{バイポーラトランジスタ(BJT)の動作原理}

\subsection{BJTの基本構造}

バイポーラトランジスタ(BJT: Bipolar Junction Transistor)は、npn型またはpnp型の3層構造を持つ半導体素子です。図\ref{fig:bjt_structure}に示すように、エミッタ(E)、ベース(B)、コレクタ(C)の3つの端子を持ち、矢印の向きに電流が流れるように設計されています。

\begin{figure}[H]
\centering
\fbox{\includegraphics[width=0.95\textwidth]{chapters/chapter03/images/page-10.pdf}}
\caption{バイポーラトランジスタ(BJT)の動作原理}
\label{fig:bjt_structure}
\end{figure}

BJTでは、エミッタからコレクタへ電子を流すために、中央のp型領域(ベース)を通過する必要があります。ここで重要なのは、ベース領域を非常に薄く作ることです。

\subsection{BJTの動作メカニズム}

BJTの動作原理を理解するためには、まず通常の動作時のバイアス条件を確認する必要があります:

\begin{itemize}
\item \textbf{エミッタ-ベース接合}:順方向バイアス(約0.7 V)
\item \textbf{ベース-コレクタ接合}:逆方向バイアス(数V〜数十V)
\end{itemize}

この条件下で、以下のプロセスが進行します:

\begin{enumerate}
\item \textbf{エミッタ(n型)からベース(p型)への電子の注入}
\begin{itemize}
\item エミッタは高濃度にドーピングされたn型半導体($N_d \approx 10^{19}$ cm$^{-3}$)
\item 順方向バイアスにより、エミッタから大量の電子がベース領域へ注入される
\end{itemize}

\item \textbf{ベース領域での電子の挙動}
\begin{itemize}
\item ベース領域は薄く(典型的には0.1〜数$\mu$m)、かつ低濃度にドーピングされたp型半導体($N_a \approx 10^{16}$ cm$^{-3}$)
\item ベース領域に注入された電子は、2つの経路をたどる:
\begin{enumerate}
\item \textbf{再結合経路}:ベース領域の正孔と再結合して消滅(少数)
\item \textbf{通過経路}:ベース領域を拡散してコレクタへ到達(多数)
\end{enumerate}
\end{itemize}

\item \textbf{コレクタ(n型)へ電子が到達}
\begin{itemize}
\item ベース-コレクタ接合の逆バイアスにより、強い電界が形成されている
\item ベース領域からコレクタ側へ到達した電子は、この電界で加速されてコレクタ電極へ引き込まれる
\end{itemize}
\end{enumerate}

図\ref{fig:bjt_structure}の下部に示されているバンド図では、エミッタ、ベース、コレクタの各領域におけるエネルギーバンドの関係が示されています。電子(黒丸)はエミッタから注入され、ベース領域の正孔(白丸)と再結合することなく、コレクタへと流れます。

\textbf{なぜ電流増幅が起こるのか?}

BJTの電流増幅の鍵は、\textbf{ベース領域が非常に薄く、かつ低濃度にドーピングされている}ことにあります。これにより、以下の現象が生じます:

\begin{itemize}
\item \textbf{ベースが薄い}:電子がベース領域を通過する時間が短い(典型的には数ps〜数十ps)
\item \textbf{ベースの正孔濃度が低い}:電子が正孔と遭遇する確率が低い
\item \textbf{エミッタの電子濃度が高い}:大量の電子が注入される
\end{itemize}

その結果、エミッタから注入された電子のうち、\textbf{大部分(例えば99\%以上)がベース領域を通過してコレクタに到達}し、\textbf{ごく一部(例えば1\%未満)だけがベース領域で再結合}します。

\textbf{具体例による理解:}

例えば、エミッタから100個の電子が注入されたとします:
\begin{itemize}
\item 99個の電子がベースを通過してコレクタに到達 → \textbf{コレクタ電流} $I_C$
\item 1個の電子がベースで正孔と再結合 → この再結合を補うために外部から正孔(電流)が供給される → \textbf{ベース電流} $I_B$
\end{itemize}

この場合、電流増幅率$\beta$は:
\[
\beta = \frac{I_C}{I_B} = \frac{99}{1} = 99
\]

つまり、\textbf{わずか1単位のベース電流を流すだけで、99単位のコレクタ電流を制御できる}のです。これがBJTの電流増幅作用の本質です。

\textbf{重要なポイント:}

\begin{itemize}
\item ベース電流$I_B$は、ベース領域での再結合を補うための電流
\item コレクタ電流$I_C$は、ベース領域を通過した電子の流れ
\item $I_C \gg I_B$となるのは、ベース領域が薄く低濃度であるため、再結合が少ないから
\item ベース電流を制御することで、エミッタからの電子注入量が変化し、結果としてコレクタ電流が変化する
\end{itemize}

このメカニズムにより、小さなベース電流で大きなコレクタ電流を制御することができ、これがトランジスタの増幅作用の基本原理となります。

\subsection{なぜベース電流が必要なのか?}

ここで重要な疑問が生じます:\textbf{「エミッタ-ベース接合に順バイアスをかけると拡散電流が流れるはずなのに、なぜ外部からベース電流を供給する必要があるのか?」}

この疑問に答えるために、pn接合の順バイアス時の電流の流れを詳しく見てみましょう。

\textbf{エミッタ-ベース接合の順バイアス時の拡散電流:}

エミッタ-ベース接合に順バイアスを印加すると、確かに拡散電流が流れます。しかし、この拡散電流には\textbf{2つの成分}があります:

\begin{enumerate}
\item \textbf{電子の拡散}:エミッタ(n型)$\rightarrow$ベース(p型)
\begin{itemize}
\item エミッタは超高濃度にドーピングされている($N_d \approx 10^{19}$ cm$^{-3}$)
\item 大量の電子がベース領域へ注入される
\end{itemize}

\item \textbf{正孔の拡散}:ベース(p型)$\rightarrow$エミッタ(n型)
\begin{itemize}
\item ベースは低濃度にドーピングされている($N_a \approx 10^{16}$ cm$^{-3}$)
\item わずかな正孔しかエミッタへ注入されない
\end{itemize}
\end{enumerate}

エミッタとベースのドーピング濃度の差が$10^3$倍もあるため、\textbf{電子の拡散が圧倒的に支配的}です。つまり、順バイアスによる拡散電流の大部分は、エミッタからベースへの電子の流れです。

\textbf{ベース領域での正孔の消費:}

ベース領域に注入された電子の一部(例えば1\%)は、ベース領域の正孔と再結合します。この再結合により:

\begin{itemize}
\item ベース領域の\textbf{正孔が消費}される
\item 時間が経つと、ベース領域の正孔濃度が\textbf{減少}する
\end{itemize}

\textbf{なぜベース電流が必要か:}

もし外部からベース端子を通じて正孔を補給しなければ、以下のことが起こります:

\begin{enumerate}
\item ベース領域の正孔濃度が減少し続ける
\item 正孔濃度が減少すると、エミッタ-ベース接合の順バイアス状態が維持できなくなる
\item エミッタからの電子注入が停止する
\item \textbf{トランジスタ動作が停止}する
\end{enumerate}

したがって、\textbf{ベース電流は、再結合により消費された正孔を補充し、トランジスタ動作を持続させるために必要}なのです。

\textbf{電流の収支バランス:}

BJTの動作中の電流の流れを整理すると:

\begin{itemize}
\item \textbf{エミッタ電流} $I_E$:エミッタから注入される電子の流れ(最大)
\item \textbf{コレクタ電流} $I_C$:ベースを通過してコレクタに到達した電子の流れ($I_E$の約99\%)
\item \textbf{ベース電流} $I_B$:ベースで再結合した電子を補うために、ベース端子から供給される正孔の流れ($I_E$の約1\%)
\end{itemize}

キルヒホッフの電流則により:
\begin{equation}
I_E = I_C + I_B
\end{equation}

\textbf{具体例:}

100 mAのエミッタ電流が流れている場合:
\begin{itemize}
\item 99 mAの電子がコレクタに到達 → $I_C = 99$ mA
\item 1 mAの電子がベースで再結合 → この再結合を補うために $I_B = 1$ mA の正孔をベース端子から供給
\item $I_E = I_C + I_B = 99 + 1 = 100$ mA
\end{itemize}

\textbf{重要なまとめ:}

\begin{itemize}
\item 順バイアスによる拡散電流は確かに存在するが、それは主に\textbf{エミッタからベースへの電子の流れ}
\item ベース領域で電子が正孔と再結合すると、\textbf{正孔が消費}される
\item この消費された正孔を補充するために、\textbf{外部からベース電流を供給}する必要がある
\item ベース電流がないと、正孔が枯渇してトランジスタ動作が停止する
\item ベース電流を制御することで、エミッタからの電子注入量を制御でき、結果としてコレクタ電流を制御できる
\end{itemize}

これがBJTにおいてベース電流が必要不可欠である理由です。

\subsection{BJTのバイアス条件}

BJTをスイッチング素子として動作させるためには、適切なバイアス条件が必要です。図\ref{fig:bjt_bias}に、順バイアスと逆バイアスの状態を示します。

\begin{figure}[H]
\centering
\fbox{\includegraphics[width=0.95\textwidth]{chapters/chapter03/images/page-11.pdf}}
\caption{バイポーラトランジスタ(BJT)のバイアス条件}
\label{fig:bjt_bias}
\end{figure}

\textbf{順バイアス状態}では:

\begin{itemize}
\item エミッタ-ベース間に順方向電圧を印加
\item エミッタから電子(黒丸)がベース領域へ注入される
\item ベース領域のp型半導体中には正孔(白丸)が存在
\item 注入された電子の一部は、ベース領域の正孔と\textbf{拡散→再結合}により消滅
\item しかし、ベース領域が十分薄いため、多くの電子は再結合せずにコレクタまで到達
\end{itemize}

図の注釈にある「拡散→再結合により電子がコレクタまで流れない」という現象は、BJTの重要な損失メカニズムです。ベース電流$I_B$は、この再結合により消費される電子の流れに相当します。

一方、\textbf{逆バイアス状態}では、エミッタ-ベース間の電位関係が逆転し、電子の注入が阻止されます。

BJTの電流増幅率$\beta$(ベータ)は、コレクタ電流$I_C$とベース電流$I_B$の比として定義されます:

\begin{equation}
\beta = \frac{I_C}{I_B}
\end{equation}

通常、$\beta$は数十から数百の値を持ち、小さなベース電流で大きなコレクタ電流を制御できることを意味します。

\section{MOSFET(金属酸化膜半導体電界効果トランジスタ)}

\subsection{MOSFETの構造}

MOSFET(Metal-Oxide-Semiconductor Field-Effect Transistor)は、トランジスタ(BJT)と構造がよく似ていますが、制御方法が異なります。図\ref{fig:mosfet_structure}に示すように、MOSFETはnpn型の基本構造に、絶縁体(酸化膜)と導体(金属ゲート)が追加された構造を持っています。

\begin{figure}[H]
\centering
\fbox{\includegraphics[width=0.95\textwidth]{chapters/chapter03/images/page-15.pdf}}
\caption{MOSFET(金属酸化膜半導体電界効果トランジスタ)の構造}
\label{fig:mosfet_structure}
\end{figure}

MOSFETの主な構成要素は以下の通りです:

\begin{itemize}
\item \textbf{ソース(S)}:電子の供給源となるn型領域
\item \textbf{ドレイン(D)}:電子の排出先となるn型領域
\item \textbf{ゲート(G)}:絶縁体を介して電圧を印加する電極
\item \textbf{p型基板}:ソースとドレインの間のp型領域
\end{itemize}

BJTではベース電流$I_B$で制御しますが、MOSFETではゲート電圧$V_G$で制御します。これは、絶縁体を介して電界を加えることで、p型領域に反転層(n型のチャネル)を形成し、電流を流す経路を作るためです。

\subsection{MOSFETの動作原理}

MOSFETの動作原理は、電界効果を利用しています:

\begin{enumerate}
\item ゲート電圧$V_G$を印加しない状態では、ソース-ドレイン間は2つのpn接合によって遮断されています
\item ゲート電圧$V_G$を印加すると、p型基板表面に電子が引き寄せられ、反転層(nチャネル)が形成されます
\item この反転層を通じて、ソースからドレインへ電流$i$が流れます
\end{enumerate}

図\ref{fig:mosfet_structure}の右側に示された3次元構造図では、絶縁体(酸化膜)を介してゲート電極が配置され、その下にp型基板があることが明確に示されています。ゲート電圧によって形成される反転層が、ソースとドレインをつなぐ導電路となります。

\subsection{反転層の形成メカニズム}

MOSFETの動作の核心は、\textbf{反転層(inversion layer)}の形成にあります。図\ref{fig:mosfet_inversion}に、反転層形成の詳細なメカニズムを示します。

\begin{figure}[H]
\centering
\fbox{\includegraphics[width=0.95\textwidth]{chapters/chapter03/images/page-16.pdf}}
\caption{MOSFETの反転層形成メカニズム}
\label{fig:mosfet_inversion}
\end{figure}

図の左側には、\textbf{バンド図}が示されています。ここで重要なエネルギー準位は以下の通りです:

\begin{itemize}
\item $E_c$:伝導帯の下端(conduction band edge)
\item $E_F$:フェルミ準位(Fermi level)
\item $E_v$:価電子帯の上端(valence band edge)
\end{itemize}

\textbf{ゲート電圧を印加していない状態}では:

\begin{itemize}
\item p型基板のフェルミ準位$E_F$は、価電子帯$E_v$に近い位置にある
\item 絶縁体によって金属ゲートと半導体が分離されている
\item p型基板表面には正孔(白丸)が多数存在
\end{itemize}

\textbf{ゲート電圧$V_G$を印加すると}(図の右側の注釈「ゲートに電位差をかけるとオンにできる」):

\begin{itemize}
\item 金属ゲートが正電位になると、絶縁体を介して電界が形成される
\item この電界により、p型基板表面に電子(黒丸)が引き寄せられる
\item p型基板表面の電子濃度が正孔濃度を超えると、\textbf{反転層}が形成される
\item この反転層は、実質的にn型の導電チャネルとして機能する
\end{itemize}

図の左側のバンド図を見ると、電子(黄色の点で表示)がフェルミ準位$E_F$付近に多数存在していることがわかります。反転層が形成されると、ソース(S)とドレイン(D)の間にn型の導電路ができ、電流が流れるようになります。

\textbf{重要なポイント:}

MOSFETでは、BJTと異なり、制御端子(ゲート)に電流を流す必要がありません。ゲート電圧によって電界を形成するだけで、主電流を制御できます。これにより、\textbf{極めて低い入力電力}でスイッチングが可能となります。

\subsection{MOSFETのスイッチ特性}

MOSFETの大きな特徴の一つは、pn接合のエネルギー障壁がないことです。図\ref{fig:mosfet_characteristics}に示すように、これによりオン電圧が小さくなります。

\begin{figure}[H]
\centering
\fbox{\includegraphics[width=0.95\textwidth]{chapters/chapter03/images/page-20.pdf}}
\caption{MOSFETのスイッチ特性}
\label{fig:mosfet_characteristics}
\end{figure}

グラフは、ドレイン-ソース間電圧$v_{\text{ds}}$とドレイン電流$i_d$の関係を、ゲート-ソース間電圧$V_{\text{gs}}$をパラメータとして示しています。ゲート電圧を変化させることで、ドレイン電流を制御できることがわかります。

図中に示されているように、pn接合のエネルギー障壁がないため、オン電圧が小さいという利点があります。これにより、スイッチング損失を低減できます。

\section{IGBT(絶縁ゲートバイポーラトランジスタ)}

\subsection{IGBTの動作原理}

IGBT(Insulated Gate Bipolar Transistor)は、MOSFETの制御の容易さとBJTの低オン抵抗の特性を組み合わせた素子です。図\ref{fig:igbt}に示すように、IGBTはnpnp型の4層構造を持ち、ゲート電圧で制御されます。

\begin{figure}[H]
\centering
\fbox{\includegraphics[width=0.95\textwidth]{chapters/chapter03/images/page-25.pdf}}
\caption{IGBTの動作原理}
\label{fig:igbt}
\end{figure}

IGBTの構造的特徴は以下の通りです:

\begin{itemize}
\item n型、p型、n型、p型の4層構造
\item ゲートに電圧をかけることで、絶縁体を介して制御
\item エミッタ(E)、コレクタ(C)、ゲート(G)の3端子構造
\end{itemize}

バイアスをかけて電流を流れやすくした場合、図の下部に示されているように、n型領域とp型領域が交互に配置された構造となります。ゲート電圧によって、エミッタからコレクタへの電流経路が制御されます。

\subsection{IGBTのハイブリッド構造}

IGBTの最大の特徴は、\textbf{MOSFETの入力特性}と\textbf{BJTの出力特性}を組み合わせたハイブリッド構造にあります。図\ref{fig:igbt_hybrid}に、ゲート電圧を印加した際の動作メカニズムを示します。

\begin{figure}[H]
\centering
\fbox{\includegraphics[width=0.95\textwidth]{chapters/chapter03/images/page-26.pdf}}
\caption{IGBTのハイブリッド構造と拡散層の形成}
\label{fig:igbt_hybrid}
\end{figure}

図に示されているように、\textbf{ゲート電圧をかけた場合}には以下の動作が起こります:

\begin{enumerate}
\item \textbf{MOSFET部分の動作}:
\begin{itemize}
\item ゲート電圧により、p型領域表面に反転層(nチャネル)が形成される
\item この反転層を通じて、エミッタからn型領域へ電子が流入する
\end{itemize}

\item \textbf{拡散層の形成}(図の注釈「拡散層が形成され、電子が流れる」):
\begin{itemize}
\item n型領域に注入された電子(黒丸)が、p型領域へ拡散していく
\item 同時に、p型領域から正孔(白丸)がn型領域へ注入される
\item この双方向のキャリア注入により、\textbf{導電率変調(conductivity modulation)}が発生
\end{itemize}

\item \textbf{導電率変調の効果}:
\begin{itemize}
\item n型ドリフト領域に正孔が注入されることで、この領域の導電率が大幅に向上
\item 結果として、オン抵抗が低減される
\item これがBJTの特性を活かした低損失化のメカニズム
\end{itemize}
\end{enumerate}

図の下部のバンド図を見ると、n-p-n-pの4層構造が明確に示されています。ゲート電圧によって形成されたチャネルを通じて電子が流れ(ピンクの矢印)、同時に正孔も注入されることで(青い矢印)、全体として大きな電流を流すことができます。

\subsection{IGBTの利点と特性}

IGBTは、MOSFETとBJTの長所を組み合わせることで、以下の優れた特性を実現しています:

\begin{itemize}
\item \textbf{電圧駆動}:MOSFETと同様に、ゲート電圧で制御できるため、駆動回路が簡単
\item \textbf{低オン抵抗}:導電率変調により、BJTのような低いオン抵抗を実現
\item \textbf{高耐圧}:npnp構造により、高電圧に耐えられる
\item \textbf{大電流容量}:導電率変調により、大電流を流すことが可能
\end{itemize}

これらの特性により、IGBTは中~大容量の電力変換器で広く使用されています。特に、電気自動車のインバータや産業用モータドライブなどで重要な役割を果たしています。

\textbf{MOSFETとIGBTの使い分け}:

\begin{itemize}
\item \textbf{MOSFET}:低電圧・高周波数・小~中容量の用途に適する
\item \textbf{IGBT}:高電圧・中周波数・中~大容量の用途に適する
\end{itemize}

\section{サイリスタの動作原理}

\subsection{サイリスタの基本動作}

サイリスタは、4層構造(npnp)を持つ半導体素子で、一度オンすると外部から制御しない限りオン状態を保持する特性を持ちます。図\ref{fig:thyristor}に示すように、何も接続していない場合の熱平衡状態では、各層のバンドが図のように形成されます。

\begin{figure}[H]
\centering
\fbox{\includegraphics[width=0.95\textwidth]{chapters/chapter03/images/page-30.pdf}}
\caption{サイリスタの動作原理}
\label{fig:thyristor}
\end{figure}

サイリスタの構造は、IGBTと同様にnpnpの4層構造ですが、制御方法が異なります。サイリスタでは、ゲート端子(G)に信号を与えることで、素子をオン状態にすることができます。

熱平衡状態のバンド図を見ると、n型とp型の接合部分でバンドが曲がっていることがわかります。この状態では、電子(黒丸)と正孔(白丸)が各領域に留まっており、電流は流れません。

ゲート信号によってサイリスタがオンすると、エミッタ(E)からコレクタ(C)へ電流が流れ始め、その後はゲート信号がなくても電流が流れ続けます。この特性は、ラッチアップと呼ばれます。

\section{ワイドギャップ半導体}

\subsection{ワイドギャップ半導体とは}

ワイドギャップ半導体は、従来のシリコン(Si)よりも大きなバンドギャップを持つ半導体材料です。図\ref{fig:widegap}に示すように、代表的なワイドギャップ半導体には、4H-SiC(シリコンカーバイド)、GaN(窒化ガリウム)、ダイヤモンドなどがあります。

\begin{figure}[H]
\centering
\fbox{\includegraphics[width=0.95\textwidth]{chapters/chapter03/images/page-35.pdf}}
\caption{ワイドギャップ半導体}
\label{fig:widegap}
\end{figure}

表に示されているように、ワイドギャップ半導体は以下の特性を持ちます:

\begin{itemize}
\item \textbf{大きなバンドギャップ}:Siの1.12 eVに対し、4H-SiCは3.26 eV、GaNは3.39 eV
\item \textbf{高い絶縁破壊電界強度}:Siの0.3 MV/cmに対し、4H-SiCは2.5 MV/cm、GaNは3.3 MV/cm
\item \textbf{高い熱伝導度}:Siの1.5 W/cmKに対し、4H-SiCは4.9 W/cmK、GaNは2 W/cmK
\end{itemize}

これらの特性により、高耐圧・低損失化が可能となり、放熱にも優れています。ただし、デメリットとして生産が難しいという課題があります。

\subsection{ワイドギャップ半導体のその他の特徴}

ワイドギャップ半導体の優れた特性は、デバイスの小型化にも貢献します。図\ref{fig:widegap_features}に示すように、容量成分が小さくなることで、以下のメリットが得られます:

\begin{figure}[H]
\centering
\fbox{\includegraphics[width=0.95\textwidth]{chapters/chapter03/images/page-40.pdf}}
\caption{ワイドギャップ半導体のその他の特徴}
\label{fig:widegap_features}
\end{figure}

\begin{enumerate}
\item \textbf{容量成分が小さい}:入力容量と出力容量が小さくなる
\item \textbf{スイッチングスピードが速くなる}:容量が小さいため、充放電時間が短縮
\item \textbf{スイッチング周波数が高くなる}:より高速なスイッチングが可能
\item \textbf{受動部品(LやC)を小さくできる}:高周波化により、インダクタやコンデンサを小型化可能
\item \textbf{製品がより小型に}:全体としてデバイスの小型化が実現
\end{enumerate}

表には、SiC MOSFETとSi MOSFET、GaN HEMTの比較が示されており、特に入力容量、出力容量、逆方向転送容量において、GaN HEMTが優れた特性を示しています。

グラフでは、GaN HEMTとSi MOSFETのスイッチング特性の比較が示されており、GaN HEMTの方がスイッチング時間が短く、高速動作が可能であることがわかります。

\section{まとめ}

\subsection{本章の要点}

本章では、スイッチング素子の物理について説明しました。主な内容は以下の通りです:

\begin{enumerate}
\item \textbf{半導体の構造からスイッチの制御の原理を説明した}:
\begin{itemize}
\item BJTはベース電流で制御
\item MOSFETはゲート電圧で制御(電界効果)
\item IGBTはMOSFETとBJTの特性を組み合わせた素子
\item サイリスタはラッチアップ特性を持つ素子
\end{itemize}

\item \textbf{ワイドギャップ半導体の特徴を説明した}:
\begin{itemize}
\item 大きなバンドギャップにより高耐圧化が可能
\item 高い絶縁破壊電界強度により薄型化が可能
\item 高い熱伝導度により放熱性能が向上
\item 小さい容量により高速スイッチングが可能
\item デバイスの小型化・高効率化が実現
\end{itemize}
\end{enumerate}

\begin{figure}[H]
\centering
\fbox{\includegraphics[width=0.95\textwidth]{chapters/chapter03/images/page-41.pdf}}
\caption{本日のまとめ}
\label{fig:summary}
\end{figure}

\subsection{次回の予告}

次回の講義では、これらのスイッチング素子を用いた実際の電力変換回路について学習します。具体的には、以下の内容を扱う予定です:

\begin{itemize}
\item 整流回路
\item インバータ回路
\item DC-DCコンバータ
\item PWM制御技術
\end{itemize}

本章で学んだスイッチング素子の特性を理解していることが、次回の内容を理解する上で重要となります。

\subsection{演習問題}

本章の理解を深めるために、以下の演習問題に取り組んでください:

\begin{enumerate}
\item BJTとMOSFETの制御方法の違いを説明してください。

\item パワー半導体において、耐圧と損失の間にトレードオフがある理由を、ドーピング濃度の観点から説明してください。

\item ワイドギャップ半導体がSiに比べて優れている点を3つ挙げ、それぞれについて説明してください。

\item IGBTがMOSFETとBJTの特性をどのように組み合わせているかを説明してください。

\item サイリスタのラッチアップ特性について説明し、この特性がどのような応用に適しているかを考察してください。
\end{enumerate}


% 第4章以降(将来追加予定)
% % 第4章:LCR回路の復習
\chapter{LCR回路の復習}

\section{はじめに}

\subsection{本章の目的と学習目標}

パワーエレクトロニクス回路において、コイル(インダクタ)とコンデンサ(キャパシタ)は、スイッチング素子と並んで重要な役割を果たします。これらの受動素子は、スイッチングによって生じるパルス状の電圧や電流を平滑化し、安定した直流電圧・電流を出力するために不可欠です。

本章では、スイッチング回路におけるコイルとコンデンサの動作原理を、回路理論とエネルギーの観点から詳しく学びます。

\textbf{学習目標:}
\begin{itemize}
\item スイッチング時のコイルとコンデンサの特性を理解する
\item コイルとコンデンサに蓄えられるエネルギーを理解する
\item フィルタとしてのLCRの役割を理解する
\item 共振回路の基本原理を理解する
\end{itemize}

\begin{figure}[H]
\centering
\fbox{\includegraphics[width=0.95\textwidth]{chapters/chapter04/images/page-02.pdf}}
\caption{本日の目標}
\label{fig:ch04_objectives}
\end{figure}

\subsection{第1回の復習:スイッチを使った電力変換}

第1回の講義では、スイッチングによる電力変換の基本原理を学びました。ここで、その要点を復習しましょう。

\subsubsection{デューティ比による電圧制御}

スイッチングによる電圧制御では、スイッチのオン・オフを高速に繰り返し、その時間平均によって出力電圧を制御します。

\begin{figure}[H]
\centering
\fbox{\includegraphics[width=0.95\textwidth]{chapters/chapter04/images/page-03.pdf}}
\caption{スイッチを使った電力変換(第1回復習)}
\label{fig:ch04_switching_review}
\end{figure}

スイッチング電圧$v_{in}(t)$の時間平均$\bar{v}_o$は、デューティ比$D$を用いて以下のように表されます:

\begin{equation}
\bar{v}_o = \frac{1}{T_{\text{SW}}} \int_0^{T_{\text{SW}}} v(t)dt = \frac{1}{T_{\text{SW}}} \times 12 T_{\text{on}} = \frac{T_{\text{on}}}{T_{\text{SW}}} \times 12
\end{equation}

ここで、デューティ比$D$は以下のように定義されます:

\begin{equation}
D = \frac{T_{\text{on}}}{T_{\text{SW}}}
\end{equation}

したがって、平均出力電圧は:

\begin{equation}
\bar{v}_o = D \times V_{\text{in}}
\end{equation}

\subsubsection{電力が送れない時間の発生}

スイッチング波形を見ると、スイッチがオフの期間では電圧がゼロとなり、負荷に電力を供給できません。この問題を解決するために、\textbf{コイルとコンデンサ}を使用します。

コイルとコンデンサは、エネルギーを一時的に蓄積し、スイッチがオフの期間でも負荷に電力を供給し続けることができます。これにより、連続的な電力供給が可能になります。

\section{LCRの用途}

パワーエレクトロニクス回路において、コイル(L)、コンデンサ(C)、抵抗(R)は、主に以下の2つの用途で使用されます。

\begin{figure}[H]
\centering
\fbox{\includegraphics[width=0.95\textwidth]{chapters/chapter04/images/page-04.pdf}}
\caption{LCRの用途(電荷や電流を貯める)}
\label{fig:ch04_lcr_usage}
\end{figure}

\subsection{フィルタ(電圧や電流を平滑化するため)}

スイッチングによって生成されるパルス状の電圧・電流波形は、そのままでは負荷に適していません。コイルとコンデンサを組み合わせたフィルタ回路により、これらの変動を抑制し、滑らかな直流電圧・電流を得ることができます。

\textbf{フィルタの動作原理:}

\begin{itemize}
\item \textbf{コイル}:電流の変動を抑制する(電流の平滑化)
\begin{itemize}
\item コイルにかかる電圧:$v_L(t) = L \frac{di_L(t)}{dt}$
\item 電流が急激に変化しようとすると、大きな逆起電力が発生し、変化を妨げる
\end{itemize}

\item \textbf{コンデンサ}:電圧の変動を抑制する(電圧の平滑化)
\begin{itemize}
\item コンデンサに流れる電流:$i_C(t) = C \frac{dv_C(t)}{dt}$
\item 電圧が急激に変化しようとすると、大きな電流が流れ、変化を妨げる
\end{itemize}
\end{itemize}

スイッチング周波数を高くし、適切なLとCの値を選ぶことで、リプル(変動)を小さく抑えることができます。

\subsection{共振回路(モーターなどの駆動のため)}

コイルとコンデンサを組み合わせた回路は、特定の周波数で共振現象を起こします。この性質を利用して、モーターの駆動や高周波電源など、さまざまな応用が可能です。

共振周波数$f_0$は以下の式で表されます:

\begin{equation}
f_0 = \frac{1}{2\pi\sqrt{LC}}
\end{equation}

共振回路については、本章の後半で詳しく説明します。

\section{スイッチング時のコイルの振る舞い}

\subsection{コイルの基本特性}

コイル(インダクタ)は、電流の変化を妨げる性質を持つ受動素子です。その電圧-電流関係は、以下の微分方程式で表されます:

\begin{equation}
v_L(t) = L \frac{di_L(t)}{dt}
\end{equation}

ここで、$v_L(t)$はコイルにかかる電圧[V]、$i_L(t)$はコイルに流れる電流[A]、$L$はインダクタンス[H]です。

\begin{figure}[H]
\centering
\fbox{\includegraphics[width=0.95\textwidth]{chapters/chapter04/images/page-05.pdf}}
\caption{スイッチング時のコイルの振る舞い}
\label{fig:ch04_inductor_switching}
\end{figure}

\textbf{重要なポイント:}

パワーエレクトロニクスでは、\textbf{スイッチング信号を入れた時の振る舞い}が重要です。スイッチがオン・オフを繰り返す際、コイルにかかる電圧が矩形波状に変化し、それに応じてコイル電流がどのように変化するかを理解する必要があります。

\subsection{スイッチング時のコイル電流の時間変化}

コイルにスイッチング電圧$v_L(t)$を印加した場合、コイル電流$i_L(t)$の時間変化を求めてみましょう。

\begin{figure}[H]
\centering
\fbox{\includegraphics[width=0.95\textwidth]{chapters/chapter04/images/page-06.pdf}}
\caption{初期条件の設定}
\label{fig:ch04_inductor_initial}
\end{figure}

\subsubsection{電流の積分計算}

コイルの電圧-電流関係式:

\begin{equation}
v_L(t) = L \frac{di_L(t)}{dt}
\end{equation}

両辺を時刻$t_s \leq t \leq t_e$の範囲で積分すると:

\begin{equation}
\int_{t_s}^{t_e} v_L(t)dt = L \int_{t_s}^{t_e} \frac{di_L(t)}{dt} dt
\end{equation}

右辺を計算すると:

\begin{equation}
\int_{t_s}^{t_e} v_L(t)dt = L \left[ i_L(t) \right]_{t_s}^{t_e} = L(i_L(t_e) - i_L(t_s))
\end{equation}

したがって、電流の変化は:

\begin{equation}
i_L(t_e) - i_L(t_s) = \frac{1}{L} \int_{t_s}^{t_e} v_L(t)dt
\end{equation}

初期条件として$t_s = 0$ [s]のとき$i_L = 0.0$ [A]とすると:

\begin{equation}
i_L(t_e) = \frac{1}{L} \int_0^{t_e} v_L(t)dt
\end{equation}

\textbf{重要な結論:}

\textcolor{blue}{コイルに流れる電流の時間変化は、コイルにかかる電圧の時間積分に比例します。}

\begin{figure}[H]
\centering
\fbox{\includegraphics[width=0.95\textwidth]{chapters/chapter04/images/page-07.pdf}}
\caption{スイッチング時のコイル電流の計算}
\label{fig:ch04_inductor_current}
\end{figure}

\subsection{コイルに正の電圧を印加し続けると}

スイッチング電圧$v_L(t)$が正の値(例えば$V_0$)の期間、コイル電流は一定の傾きで増加し続けます。

\begin{figure}[H]
\centering
\fbox{\includegraphics[width=0.95\textwidth]{chapters/chapter04/images/page-08.pdf}}
\caption{コイルに正の電圧を印加すると電流が際限なく増加する}
\label{fig:ch04_inductor_increasing}
\end{figure}

電圧が$V_0$で一定の期間、電流の増加率は:

\begin{equation}
\frac{di_L}{dt} = \frac{V_0}{L} = \text{一定}
\end{equation}

したがって、コイル電流は:

\begin{equation}
i_L(t) = \frac{V_0}{L} t
\end{equation}

と、時間に比例して増加し続けます。

\textbf{問題点:}

このままでは電流が際限なく増加してしまいます。\textcolor{red}{どうすれば定常状態になるのでしょうか?増加も減少もしなくなるのでしょうか?}

\textbf{解決策:}

\textcolor{blue}{負の電圧をかけると電流が減少します。}定常状態では、IとIIの領域の面積(電圧×時間の積分値)が等しくなる必要があります。

\subsection{定常状態時のコイルに流れる電流}

定常状態では、1スイッチング周期$T$における電流の変化がゼロになります。すなわち、スイッチオン時に増加した電流と、スイッチオフ時に減少した電流が釣り合う必要があります。

\begin{figure}[H]
\centering
\fbox{\includegraphics[width=0.95\textwidth]{chapters/chapter04/images/page-09.pdf}}
\caption{定常状態時のコイルに流れる電流}
\label{fig:ch04_inductor_steady}
\end{figure}

\textbf{定常状態の条件:}

\begin{equation}
\int_0^{DT} v_L(t)dt + \int_{DT}^{T} v_L(t)dt = 0
\end{equation}

つまり、領域Iと領域IIの面積が等しくなります:

\begin{equation}
DT \cdot V_0 = (T - DT) \cdot |v_L|
\end{equation}

\subsection{電圧源や負荷が接続された場合のコイル電圧}

実際のパワーエレクトロニクス回路では、コイルに電圧源や負荷(抵抗)が接続されています。この場合、キルヒホッフの電圧則(KVL)を用いて、コイルにかかる電圧を求めます。

\begin{figure}[H]
\centering
\fbox{\includegraphics[width=0.95\textwidth]{chapters/chapter04/images/page-10.pdf}}
\caption{定常状態時の出力電圧とスイッチング電圧の関係(問題設定)}
\label{fig:ch04_inductor_load_problem}
\end{figure}

\subsubsection{回路解析}

電圧源$v_{in}(t)$、コイル$L$、コンデンサ$C$、出力電圧$V_1$が接続された回路を考えます。

\textbf{KVL(キルヒホッフの電圧則):}

\begin{equation}
v_{in}(t) - v_L(t) - V_1 = 0
\end{equation}

したがって、コイルにかかる電圧は:

\begin{equation}
v_L(t) = v_{in}(t) - V_1
\end{equation}

\begin{figure}[H]
\centering
\fbox{\includegraphics[width=0.95\textwidth]{chapters/chapter04/images/page-11.pdf}}
\caption{KVLによるコイル電圧の導出}
\label{fig:ch04_inductor_load_kvl}
\end{figure}

コンデンサは、定常状態において電圧がほぼ一定($V_1$)となります。したがって、コイルにかかる電圧$v_L(t)$は、以下のようになります:

\textbf{スイッチON時($0 \leq t < DT$):}
\begin{equation}
v_L(t) = V_0 - V_1
\end{equation}

\textbf{スイッチOFF時($DT \leq t < T$):}
\begin{equation}
v_L(t) = 0 - V_1 = -V_1
\end{equation}

\begin{figure}[H]
\centering
\fbox{\includegraphics[width=0.95\textwidth]{chapters/chapter04/images/page-12.pdf}}
\caption{負荷電圧に応じて電流の大きさが変わる}
\label{fig:ch04_inductor_load_voltage}
\end{figure}

\subsubsection{定常状態条件と出力電圧の導出}

定常状態では、1周期における電圧の時間積分がゼロになります:

\begin{equation}
\int_0^{T} v_L(t)dt = 0
\end{equation}

つまり、領域Iと領域IIの面積が等しくなります:

\begin{equation}
DT(V_0 - V_1) = (T - DT)V_1
\end{equation}

展開すると:

\begin{equation}
DTV_0 - DTV_1 = TV_1 - DTV_1
\end{equation}

\begin{equation}
DTV_0 = TV_1
\end{equation}

したがって、定常状態時の出力電圧$V_1$は:

\begin{equation}
\boxed{V_1 = DV_0}
\end{equation}

\textbf{重要な結論:}

\textcolor{red}{デューティ比によって出力電圧が制御可能です。}定常状態の時、$V_1$はスイッチング電圧の平均電圧となります。

これが、スイッチングとコイルを用いた降圧型DC-DCコンバータ(バックコンバータ)の基本原理です。

\subsection{定常状態時の電流の振動の大きさ}

定常状態において、コイル電流は平均値を中心に上下に振動します。この振動の大きさ(リプル)を計算してみましょう。

\begin{figure}[H]
\centering
\fbox{\includegraphics[width=0.95\textwidth]{chapters/chapter04/images/page-13.pdf}}
\caption{定常状態時の電流の振動の大きさ(問題設定)}
\label{fig:ch04_inductor_ripple_problem}
\end{figure}

理想的には直流電流を流したいところですが、実際にはスイッチングによる電流リプルが発生します。この振動を小さくするには、どうすれば良いでしょうか?

\subsubsection{電流リプルの計算}

コイル電圧が$v_L(t) = V_0 - DV_0 = V_0(1-D)$の期間($0 \leq t \leq DT$)における電流の変化を考えます。

\begin{figure}[H]
\centering
\fbox{\includegraphics[width=0.95\textwidth]{chapters/chapter04/images/page-14.pdf}}
\caption{電流リプルの計算}
\label{fig:ch04_inductor_ripple_calc}
\end{figure}

電流の増加量は:

\begin{equation}
\Delta i_L = I_{\max} - I_{\min} = \frac{1}{L} \int_0^{DT} v_L(t)dt
\end{equation}

スイッチON期間では$v_L(t) = V_0 - DV_0$が一定なので:

\begin{equation}
I_{\max} - I_{\min} = \frac{1}{L} DT(V_0 - DV_0)
\end{equation}

簡略化すると:

\begin{equation}
\boxed{\Delta i_L = \frac{DT(1-D)V_0}{L}}
\end{equation}

\textbf{重要な考察:}

\begin{itemize}
\item 電流リプル$\Delta i_L$は、\textbf{インダクタンス$L$に反比例}します
\item $L$を大きくすると、電流の変動が抑えられます
\item 周期$T$(すなわちスイッチング周波数$f_{\text{sw}} = 1/T$)を小さくする(周波数を高くする)と、リプルが小さくなります
\end{itemize}

\subsection{負荷(抵抗)を繋げた時の振る舞い}

実際の回路では、コイルの出力側に負荷抵抗$R$が接続されています。この場合の動作を考えてみましょう。

\begin{figure}[H]
\centering
\fbox{\includegraphics[width=0.95\textwidth]{chapters/chapter04/images/page-15.pdf}}
\caption{負荷(抵抗)を繋げた時の振る舞い(問題設定)}
\label{fig:ch04_inductor_resistor_problem}
\end{figure}

抵抗$R$を負荷として接続した場合、この抵抗にかかる電圧$v_R(t)$はどのようになるでしょうか?

\subsubsection{逐次近似法による解析}

この問題を厳密に解くためには、微分方程式を解く必要がありますが、ここでは\textbf{逐次近似法}(successive approximation method)を用いて近似解を求めます。

\textbf{逐次近似法とは?}

逐次近似法は、複雑な方程式を解く際に、初期推定値から始めて徐々に解を改善していく数値解法です。以下のような状況で有効です:

\begin{itemize}
\item 解析的に厳密解を求めるのが困難な場合
\item 複数の変数が相互に依存している場合
\item 微分方程式と代数方程式が連立している場合
\end{itemize}

\textbf{なぜこの方法が必要か?}

本問題では、以下の2つの関係式が同時に成り立ちます:

\begin{itemize}
\item KVL(キルヒホッフの電圧則):$v_{in}(t) = v_L(t) + v_R(t)$
\item コイルの電圧-電流関係:$v_L(t) = L \frac{di_L(t)}{dt}$
\item オームの法則:$v_R(t) = R \cdot i_L(t)$
\end{itemize}

これらを組み合わせると、以下の微分方程式が得られます:

\begin{equation}
v_{in}(t) = L \frac{di_L(t)}{dt} + R \cdot i_L(t)
\end{equation}

この1階線形微分方程式は、$v_{in}(t)$がステップ関数の場合は厳密解を求められますが、スイッチング波形(矩形波)の場合は各区間で解を求め、境界条件を一致させる必要があり、計算が煩雑になります。

\textbf{逐次近似法の手順}

逐次近似法では、以下のプロセスを繰り返します:

\begin{figure}[H]
\centering
\fbox{\includegraphics[width=0.95\textwidth]{chapters/chapter04/images/page-16.pdf}}
\caption{逐次近似法による解析}
\label{fig:ch04_inductor_resistor_iteration}
\end{figure}

\textbf{第0次近似(初期推定):}

初期状態として、抵抗電圧$v_R^{(0)}(t)$が定常状態の平均値で一定であると仮定します:

\begin{equation}
v_R^{(0)}(t) = DV_0 \quad \text{(一定)}
\end{equation}

この仮定は、コイルが十分に大きく、電流リプルが小さい場合に妥当です。

\textbf{第1次近似:}

第0次近似の抵抗電圧を使って、コイル電圧を計算します:

\begin{equation}
v_L^{(1)}(t) = v_{in}(t) - v_R^{(0)}(t) = v_{in}(t) - DV_0
\end{equation}

スイッチング波形を考慮すると:

\begin{equation}
v_L^{(1)}(t) = \begin{cases}
V_0 - DV_0 = V_0(1-D) & (0 \leq t < DT) \\
0 - DV_0 = -DV_0 & (DT \leq t < T)
\end{cases}
\end{equation}

このコイル電圧から、コイル電流を積分により計算します:

\begin{equation}
i_L^{(1)}(t) = i_L(0) + \frac{1}{L} \int_0^t v_L^{(1)}(\tau) d\tau
\end{equation}

スイッチON期間($0 \leq t < DT$)では:

\begin{equation}
i_L^{(1)}(t) = I_{\min} + \frac{V_0(1-D)}{L} t
\end{equation}

電流は直線的に増加します。時刻$t = DT$での電流は:

\begin{equation}
i_L^{(1)}(DT) = I_{\min} + \frac{V_0(1-D)DT}{L}
\end{equation}

スイッチOFF期間($DT \leq t < T$)では:

\begin{equation}
i_L^{(1)}(t) = i_L^{(1)}(DT) - \frac{DV_0}{L}(t - DT)
\end{equation}

電流は直線的に減少します。

次に、オームの法則を使って、新しい抵抗電圧を計算します:

\begin{equation}
v_R^{(1)}(t) = R \cdot i_L^{(1)}(t)
\end{equation}

この$v_R^{(1)}(t)$は、もはや一定ではなく、電流と同じく三角波状に変動します。

\textbf{第2次近似以降:}

第1次近似で得られた$v_R^{(1)}(t)$を使って、同じプロセスを繰り返します:

\begin{enumerate}
\item $v_L^{(2)}(t) = v_{in}(t) - v_R^{(1)}(t)$を計算
\item $v_L^{(2)}(t)$から$i_L^{(2)}(t)$を積分により計算
\item $v_R^{(2)}(t) = R \cdot i_L^{(2)}(t)$を計算
\end{enumerate}

この反復を繰り返すと、解は真の解に収束していきます:

\begin{equation}
v_R^{(0)}(t) \rightarrow v_R^{(1)}(t) \rightarrow v_R^{(2)}(t) \rightarrow \cdots \rightarrow v_R^{(\infty)}(t) = v_R(t)
\end{equation}

\textbf{収束の判定:}

通常、以下のような収束条件を使います:

\begin{equation}
\max_t |v_R^{(n+1)}(t) - v_R^{(n)}(t)| < \epsilon
\end{equation}

ここで、$\epsilon$は許容誤差(例えば$10^{-6}$ V)です。

\textbf{逐次近似法の利点と限界}

\textit{利点:}
\begin{itemize}
\item 複雑な微分方程式を解かなくても、近似解が得られる
\item プログラミングによる数値計算が容易
\item 各反復で解の精度が向上していく様子が視覚的に理解できる
\end{itemize}

\textit{限界:}
\begin{itemize}
\item 収束が保証されない場合もある(適切な初期値が必要)
\item 収束が遅い場合、多くの反復が必要
\item 厳密解ではなく、あくまで近似解
\end{itemize}

\textbf{実際の回路での意味}

この解析から分かることは、実際の回路では:

\begin{itemize}
\item 抵抗電圧$v_R(t)$は完全に一定ではなく、わずかに変動(リプル)を持つ
\item 電流リプルが小さい場合($L$が大きい場合)、第0次近似でも十分な精度が得られる
\item リプルを考慮した厳密な解析には、逐次近似法や直接的な微分方程式の解法が必要
\end{itemize}

\textbf{数値例}

$V_0 = 12$ V、$D = 0.5$、$L = 10$ mH、$R = 1~\Omega$、$T = 10~\mu$sの場合を考えます。

第0次近似では:
\begin{equation}
v_R^{(0)} = DV_0 = 6~\text{V(一定)}
\end{equation}

第1次近似では、電流リプルにより抵抗電圧は約$\pm 3$mV程度変動します。これは平均値の約0.05\%であり、多くの実用的な目的では第0次近似で十分です。

しかし、高精度な制御や、リプルが制御性能に影響を与える場合には、より高次の近似や厳密解が必要になります。

\subsection{フィルタとしてのコイルの役割}

ここまでの解析から、コイルの重要な役割が明らかになります。

\begin{figure}[H]
\centering
\fbox{\includegraphics[width=0.95\textwidth]{chapters/chapter04/images/page-17.pdf}}
\caption{フィルタとしてのコイルの役割}
\label{fig:ch04_inductor_filter}
\end{figure}

\textbf{コイルによる電圧の変動抑制:}

\begin{itemize}
\item コイルによって、電圧の変動が小さくなる
\item $L$を大きくすると電圧の変動が抑えられる
\item これを\textbf{ローパスフィルタ}として用いる
\end{itemize}

\textbf{平均電圧の制御:}

\begin{itemize}
\item 平均電圧はデューティ比に比例する:$\bar{v}_R = DV_0$
\item スイッチと$L$によって、電圧を降下することができる
\end{itemize}

\subsection{応用例:スイッチング信号から交流信号の生成}

デューティ比を時間的に変化させることで、任意の波形を生成できます。これは、インバータ(DC-AC変換器)の基本原理です。

\begin{figure}[H]
\centering
\fbox{\includegraphics[width=0.95\textwidth]{chapters/chapter04/images/page-18.pdf}}
\caption{応用例:スイッチング信号から交流信号の生成}
\label{fig:ch04_ac_generation}
\end{figure}

\textbf{動作原理:}

\begin{itemize}
\item デューティ比$D(t)$を正弦波状に変化させる:$D(t) = 0.5 + 0.5\sin(\omega t)$
\item スイッチング電圧の平均値が正弦波状に変化する
\item ローパスフィルタ(LCフィルタ)で高周波成分を除去
\item 結果として、滑らかな正弦波交流電圧が得られる
\end{itemize}

\textcolor{blue}{平均電圧を変えることで、任意の波形を生成できます。}これが、PWM(Pulse Width Modulation:パルス幅変調)インバータの基本原理です。

\section{コイルに流出入するエネルギー}

コイルは、磁気エネルギーとしてエネルギーを蓄積する能力を持ちます。この性質が、スイッチングによる電力変換において重要な役割を果たします。

\subsection{磁気エネルギーの式}

コイルに蓄積される磁気エネルギー$U_m$は、以下の式で表されます:

\begin{equation}
\boxed{U_m = \frac{1}{2}\Phi i_L = \frac{1}{2}Li_L^2}
\end{equation}

ここで、$\Phi$は磁束[Wb]、$i_L$はコイル電流[A]、$L$はインダクタンス[H]です。

\begin{figure}[H]
\centering
\fbox{\includegraphics[width=0.95\textwidth]{chapters/chapter04/images/page-19.pdf}}
\caption{コイルに流出入するエネルギー}
\label{fig:ch04_inductor_energy}
\end{figure}

\subsection{エネルギーの蓄積と放出}

スイッチング動作において、コイルのエネルギーは時間とともに変化します。

\textbf{領域I:コイルに磁気エネルギーが蓄積}

スイッチON期間では、コイル電流が増加し、磁気エネルギーが蓄積されます。

\textbf{領域II:コイルから磁気エネルギーが放出}

スイッチOFF期間では、コイル電流が減少し、蓄積されていた磁気エネルギーが負荷に供給されます。

\textbf{重要なポイント:}

\textcolor{blue}{コイルは、エネルギーを一時的に蓄積し、後で放出する「エネルギーバッファ」として機能します。}これにより、スイッチがOFFの期間でも負荷に電力を供給し続けることができます。

\section{スイッチング時のコンデンサの振る舞い}

\subsection{コンデンサの基本特性}

コンデンサ(キャパシタ)は、電圧の変化を妨げる性質を持つ受動素子です。その電流-電圧関係は、以下の微分方程式で表されます:

\begin{equation}
i_C(t) = C \frac{dv_C(t)}{dt}
\end{equation}

ここで、$i_C(t)$はコンデンサに流れる電流[A]、$v_C(t)$はコンデンサにかかる電圧[V]、$C$はキャパシタンス[F]です。

\begin{figure}[H]
\centering
\fbox{\includegraphics[width=0.95\textwidth]{chapters/chapter04/images/page-20.pdf}}
\caption{スイッチング時のコンデンサの振る舞い}
\label{fig:ch04_capacitor_switching}
\end{figure}

\textbf{重要なポイント:}

コンデンサはコイルの電流と電圧の関係を入れ替えただけです。コイルが電流源として振る舞うのに対し、\textcolor{blue}{コンデンサは電圧源として振る舞います。}

\subsection{スイッチング時のコンデンサ電圧の時間変化}

コンデンサにスイッチング電流$i_C(t)$を印加した場合、コンデンサ電圧$v_C(t)$の時間変化を求めてみましょう。

\begin{figure}[H]
\centering
\fbox{\includegraphics[width=0.95\textwidth]{chapters/chapter04/images/page-21.pdf}}
\caption{スイッチング時のコンデンサ電圧の計算}
\label{fig:ch04_capacitor_voltage}
\end{figure}

\subsubsection{電圧の積分計算}

コンデンサの電流-電圧関係式:

\begin{equation}
i_C(t) = C \frac{dv_C(t)}{dt}
\end{equation}

両辺を時刻$t_s \leq t \leq t_e$の範囲で積分すると:

\begin{equation}
\int_{t_s}^{t_e} i_C(t)dt = C \int_{t_s}^{t_e} \frac{dv_C(t)}{dt} dt
\end{equation}

右辺を計算すると:

\begin{equation}
\int_{t_s}^{t_e} i_C(t)dt = C \left[ v_C(t) \right]_{t_s}^{t_e} = C(v_C(t_e) - v_C(t_s))
\end{equation}

したがって、電圧の変化は:

\begin{equation}
v_C(t_e) - v_C(t_s) = \frac{1}{C} \int_{t_s}^{t_e} i_C(t)dt
\end{equation}

初期条件として$t_s = 0$ [s]のとき$v_C = 0.0$ [V]とすると:

\begin{equation}
v_C(t_e) = \frac{1}{C} \int_0^{t_e} i_C(t)dt
\end{equation}

\textbf{重要な結論:}

\textcolor{blue}{コンデンサにかかる電圧の時間変化は、コンデンサに流れる電流の時間積分に比例します。}これは、コイルの場合と双対的な関係にあります。

\subsection{コンデンサに正の電流を流し続けると}

スイッチング電流$i_C(t)$が正の値(例えば$I_0$)の期間、コンデンサ電圧は一定の傾きで増加し続けます。

\begin{figure}[H]
\centering
\fbox{\includegraphics[width=0.95\textwidth]{chapters/chapter04/images/page-23.pdf}}
\caption{コンデンサに正の電流を流すと電圧が際限なく増加する}
\label{fig:ch04_capacitor_increasing}
\end{figure}

電流が$I_0$で一定の期間、電圧の増加率は:

\begin{equation}
\frac{dv_C}{dt} = \frac{I_0}{C} = \text{一定}
\end{equation}

したがって、コンデンサ電圧は:

\begin{equation}
v_C(t) = \frac{I_0}{C} t
\end{equation}

と、時間に比例して増加し続けます。

\textbf{問題点:}

このままでは電圧が際限なく増加してしまいます。\textcolor{red}{どうすれば定常状態になるのでしょうか?増加も減少もしなくなるのでしょうか?}

\textbf{解決策:}

コイルの場合と同様に、\textcolor{blue}{負の電流を流すと電圧が減少します。}定常状態では、IとIIの領域の面積(電流×時間の積分値)が等しくなる必要があります。

\subsection{定常状態時のコンデンサにかかる電圧}

定常状態では、1スイッチング周期$T$における電圧の変化がゼロになります。

\begin{figure}[H]
\centering
\fbox{\includegraphics[width=0.95\textwidth]{chapters/chapter04/images/page-24.pdf}}
\caption{定常状態時のコンデンサにかかる電圧}
\label{fig:ch04_capacitor_steady}
\end{figure}

\textbf{定常状態の条件:}

\begin{equation}
\int_0^{DT} i_C(t)dt + \int_{DT}^{T} i_C(t)dt = 0
\end{equation}

つまり、領域Iと領域IIの面積が等しくなります。

\subsection{電流源や負荷が接続された場合のコンデンサ電流}

実際の回路では、コンデンサに電流源や負荷が接続されています。この場合、キルヒホッフの電流則(KCL)を用いて、コンデンサに流れる電流を求めます。

\begin{figure}[H]
\centering
\fbox{\includegraphics[width=0.95\textwidth]{chapters/chapter04/images/page-25.pdf}}
\caption{定常状態時の出力電圧とスイッチング電圧の関係(コンデンサと電流源)}
\label{fig:ch04_capacitor_load}
\end{figure}

\subsubsection{回路解析}

電流源$i_{in}(t)$、コンデンサ$C$、負荷電流$I_1$が接続された回路を考えます。

\textbf{KCL(キルヒホッフの電流則):}

\begin{equation}
-I_{in} + i_C + I_1 = 0
\end{equation}

したがって、コンデンサに流れる電流は:

\begin{equation}
i_C = I_{in} - I_1
\end{equation}

\textbf{負荷電流に応じて電圧の大きさが変わります。}

\subsection{フィルタとしてのコンデンサの役割}

コンデンサもコイルと同様に、フィルタとして重要な役割を果たします。

\begin{figure}[H]
\centering
\fbox{\includegraphics[width=0.95\textwidth]{chapters/chapter04/images/page-30.pdf}}
\caption{フィルタとしてのコンデンサの役割}
\label{fig:ch04_capacitor_filter}
\end{figure}

\textbf{コンデンサによる電流の変動抑制:}

\begin{itemize}
\item コンデンサによって、電流の変動が小さくなる
\item $C$を大きくすると電流の変動が抑えられる
\item これを\textbf{ローパスフィルタ}として用いる
\end{itemize}

\textbf{平均電流の制御:}

\begin{itemize}
\item 平均電流はデューティ比に比例する
\item スイッチと$C$によって、電流を降下することができる
\end{itemize}

\section{RLC直列回路のエネルギー}

コイルとコンデンサを組み合わせた回路では、エネルギーの交換による\textbf{共振現象}が発生します。

\subsection{RLC直列回路のインピーダンス}

RLC直列回路のインピーダンス$Z$は、以下のように表されます:

\begin{equation}
Z = j\left(\omega L - \frac{1}{\omega C}\right) + R
\end{equation}

ここで、$j$は虚数単位、$\omega = 2\pi f$は角周波数[rad/s]です。

\begin{figure}[H]
\centering
\fbox{\includegraphics[width=0.95\textwidth]{chapters/chapter04/images/page-35.pdf}}
\caption{RLC直列回路のエネルギー}
\label{fig:ch04_rlc_energy}
\end{figure}

\subsection{共振周波数}

インピーダンスの虚部がゼロになる周波数を\textbf{共振周波数}$f_0$と呼びます:

\begin{equation}
\omega_0 L - \frac{1}{\omega_0 C} = 0
\end{equation}

したがって:

\begin{equation}
\omega_0 = \frac{1}{\sqrt{LC}}
\end{equation}

\begin{equation}
\boxed{f_0 = \frac{1}{2\pi\sqrt{LC}}}
\end{equation}

共振周波数では、インピーダンスが最小($Z = R$)となり、回路に最大の電流が流れます。

\subsection{エネルギーの交換}

RLC回路では、コイルの磁気エネルギー$\frac{1}{2}Li_L^2$とコンデンサの静電エネルギー$\frac{1}{2}Cv_C^2$が周期的に交換されます。

\begin{figure}[H]
\centering
\fbox{\includegraphics[width=0.95\textwidth]{chapters/chapter04/images/page-40.pdf}}
\caption{コンデンサ主体のRLC直列回路のエネルギー}
\label{fig:ch04_rlc_capacitor_dominant}
\end{figure}

\textbf{エネルギー交換のプロセス:}

\begin{enumerate}
\item \textbf{コイルのエネルギー蓄積}:電流が増加し、磁気エネルギーが蓄積
\item \textbf{Lのエネルギー放出}:蓄積された磁気エネルギーがコンデンサに転送
\item \textbf{コンデンサのエネルギー蓄積}:電圧が上昇し、静電エネルギーが蓄積
\item \textbf{Cのエネルギー放出}:蓄積された静電エネルギーがコイルに転送
\end{enumerate}

このエネルギー交換が周期的に繰り返されることで、共振現象が発生します。抵抗$R$が小さい場合、エネルギー損失が少なく、振動が持続します。

\subsection{フィルタと共振器の違い}

同じLCR回路でも、周波数領域によって動作が異なります。

\textbf{低周波数領域(フィルタ動作):}

\begin{itemize}
\item スイッチング周波数$f_{\text{sw}} \ll f_0$の場合
\item LとCが独立に動作し、それぞれ電流と電圧の平滑化を担当
\item エネルギー交換はほとんど起こらない
\end{itemize}

\textbf{共振周波数付近(共振器動作):}

\begin{itemize}
\item 動作周波数$f \approx f_0$の場合
\item LとCの間でエネルギーが激しく交換される
\item 大きな電圧・電流振動が発生
\end{itemize}

\section{コンデンサに電圧源を繋げてはいけない理由}

最後に、コンデンサ回路における重要な注意点を説明します。

\begin{figure}[H]
\centering
\fbox{\includegraphics[width=0.95\textwidth]{chapters/chapter04/images/page-45.pdf}}
\caption{コンデンサに電圧源を繋げてはいけない理由}
\label{fig:ch04_capacitor_voltage_source}
\end{figure}

\subsection{コンデンサに電圧源を繋ぐとどうなる?}

理想的な電圧源を直接コンデンサに接続すると、以下のような問題が発生します。

\textbf{電流の発散:}

コンデンサの電流は:

\begin{equation}
i_C(t) = C \frac{dv_C(t)}{dt}
\end{equation}

スイッチをONにした瞬間、コンデンサ電圧は瞬時に$V_0$に変化しようとします。このとき、電圧の変化率$\frac{dv_C(t)}{dt}$が無限大となり、電流$i_C(t)$も無限大になります:

\begin{equation}
i_C(t) = C \frac{V_0}{0} \rightarrow \infty
\end{equation}

\textbf{実際の回路での影響:}

\begin{itemize}
\item 実際には、配線の抵抗やインダクタンスにより、電流は有限値に制限されます
\item しかし、非常に大きな突入電流(インラッシュカレント)が流れ、部品が破壊される可能性があります
\item スイッチング素子や電源に大きなストレスがかかります
\end{itemize}

\subsection{正しい接続方法}

コンデンサを電圧源に接続する場合は、必ず抵抗やインダクタを直列に接続し、電流を制限する必要があります。

\textbf{推奨される回路構成:}

\begin{itemize}
\item \textbf{突入電流制限抵抗}:コンデンサの充電電流を制限
\item \textbf{インダクタ}:電流の急激な変化を防ぐ
\item \textbf{ソフトスタート回路}:徐々に電圧を印加
\end{itemize}

\section{まとめ}

本章では、パワーエレクトロニクス回路におけるコイルとコンデンサの動作原理を学習しました。

\begin{figure}[H]
\centering
\fbox{\includegraphics[width=0.95\textwidth]{chapters/chapter04/images/page-47.pdf}}
\caption{まとめ}
\label{fig:ch04_summary}
\end{figure}

\textbf{主な学習内容:}

\begin{itemize}
\item スイッチング時のコイルとコンデンサの特性について説明した(平滑化と共振)
\item コイルとコンデンサに蓄えられるエネルギーの観点から平滑化と共振器の原理について説明した
\end{itemize}

\subsection{コイルの役割}

\begin{enumerate}
\item \textbf{電流源として動作}:$v_L(t) = L \frac{di_L(t)}{dt}$
\item \textbf{電流の平滑化}:電流の急激な変化を抑制
\item \textbf{磁気エネルギーの蓄積}:$U_m = \frac{1}{2}Li_L^2$
\item \textbf{定常状態条件}:$\int_0^T v_L(t)dt = 0$
\end{enumerate}

\subsection{コンデンサの役割}

\begin{enumerate}
\item \textbf{電圧源として動作}:$i_C(t) = C \frac{dv_C(t)}{dt}$
\item \textbf{電圧の平滑化}:電圧の急激な変化を抑制
\item \textbf{静電エネルギーの蓄積}:$U_e = \frac{1}{2}Cv_C^2$
\item \textbf{定常状態条件}:$\int_0^T i_C(t)dt = 0$
\end{enumerate}

\subsection{LCフィルタの設計指針}

パワーエレクトロニクス回路におけるLCフィルタの設計では、以下の点を考慮します:

\begin{itemize}
\item \textbf{リプルの低減}:$L$と$C$を大きくすると、電流・電圧リプルが小さくなる
\item \textbf{共振周波数}:$f_0 = \frac{1}{2\pi\sqrt{LC}}$がスイッチング周波数より十分低くなるように設計
\item \textbf{サイズとコスト}:$L$と$C$が大きすぎると、部品サイズとコストが増大
\item \textbf{過渡応答}:$L$と$C$が大きすぎると、応答速度が遅くなる
\end{itemize}

\subsection{次回の予告}

次回は、これらの知識を基に、実際のDC-DCコンバータ(降圧型・昇圧型)の動作原理と設計方法について学習します。


% \chapter{直流-直流変換(1)}

\section{はじめに}

本章では、パワーエレクトロニクスの基本的な応用である\textbf{DC-DC変換}(直流-直流変換)について学習します。DC-DC変換は、直流電圧を異なる直流電圧に変換する技術であり、電源回路、モータ駆動回路、バッテリー充電回路など、幅広い分野で利用されています。

\begin{figure}[H]
\centering
\fbox{\includegraphics[width=0.95\textwidth]{chapters/chapter05/images/page-02.pdf}}
\caption{本日の目標}
\label{fig:ch05_objectives}
\end{figure}

\textbf{本章の学習目標:}

\begin{itemize}
\item 降圧チョッパー回路の構成と原理を理解する
\item 昇圧チョッパー回路の構成と原理を理解する
\item 昇降圧チョッパー回路の構成と原理を理解する
\item 定常状態のスイッチのオンオフ時における各素子の電圧や電流波形を理解する
\end{itemize}

\section{DC-DC変換とは}

\subsection{DC-DC変換の基本概念}

DC-DC変換とは、ある直流電圧を別の直流電圧に変換することです。変換の方法には、大きく分けて以下の3種類があります。

\begin{figure}[H]
\centering
\fbox{\includegraphics[width=0.95\textwidth]{chapters/chapter05/images/page-03.pdf}}
\caption{直流-直流(DC-DC)変換とは}
\label{fig:ch05_dcdc_overview}
\end{figure}

\textbf{DC-DC変換の種類:}

\begin{enumerate}
\item \textbf{非絶縁型チョッパ回路}(昇圧・降圧・昇降圧)- 第5回(本章)
\item \textbf{絶縁型チョッパ回路}(昇圧・降圧)- 第6回
\item \textbf{リニアレギュレータ}(降圧)- 第7回
\end{enumerate}

本章では、最も基本的な非絶縁型チョッパ回路について学習します。チョッパ回路は、スイッチング素子を用いて電圧を高速にオン・オフすることで、平均電圧を制御する回路です。

\subsection{チョッパ回路の利点}

チョッパ回路には、以下のような利点があります:

\begin{itemize}
\item \textbf{高効率}:スイッチング素子が理想的にはオンまたはオフのいずれかの状態にあるため、損失が少ない
\item \textbf{小型・軽量}:高周波スイッチングにより、リアクタンス素子(コイル、コンデンサ)を小型化できる
\item \textbf{柔軟な電圧制御}:デューティ比を変化させることで、出力電圧を広範囲に制御できる
\end{itemize}

\section{定常状態におけるコイルの振る舞い(復習)}

DC-DC変換回路を理解するためには、第4章で学習したコイルの定常状態における振る舞いを理解することが重要です。

\begin{figure}[H]
\centering
\fbox{\includegraphics[width=0.95\textwidth]{chapters/chapter05/images/page-04.pdf}}
\caption{定常状態におけるコイルの振る舞い(前回の復習)}
\label{fig:ch05_inductor_review}
\end{figure}

\textbf{コイルの基本特性:}

コイルに印加される電圧$v_L(t)$と電流$i_L(t)$の関係は:

\begin{equation}
v_L(t) = L \frac{di_L(t)}{dt}
\end{equation}

これより、電流の変化は:

\begin{equation}
\frac{di_L(t)}{dt} = \frac{v_L(t)}{L}
\end{equation}

したがって、電流の変化は電圧に比例します。

\textbf{定常状態の条件:}

\begin{itemize}
\item スイッチングに応じてプラスとマイナスに変動する
\item \textcolor{red}{1周期の積分の値は0になる}(電圧-秒バランス)
\end{itemize}

電流については:

\begin{itemize}
\item スイッチングに応じて増減しているが
\item \textcolor{red}{1周期あたりの増減量は等しい}
\end{itemize}

\subsection{定常状態の数学的表現}

定常状態では、1周期$T$におけるコイル電圧の積分がゼロになります:

\begin{equation}
\int_0^T v_L(t)dt = 0
\end{equation}

これは、コイルに蓄えられる磁気エネルギーが周期的に増減し、1周期後には元の状態に戻ることを意味します。

この条件を用いることで、DC-DC変換回路の出力電圧を求めることができます。

\begin{figure}[H]
\centering
\fbox{\includegraphics[width=0.95\textwidth]{chapters/chapter05/images/page-05.pdf}}
\caption{定常状態について}
\label{fig:ch05_steady_state}
\end{figure}

図に示すように、実際の回路シミュレーションでも、定常状態に達すると同じ波形が繰り返されることが確認できます。

\section{降圧チョッパー回路(Buck Converter)}

\subsection{降圧チョッパー回路とは}

降圧チョッパー回路(Buck Converter)は、入力電圧$V_{\text{in}}$より低い出力電圧$V_{\text{out}}$を得るための回路です。「降圧」という名前の通り、電圧を下げる(降ろす)ことができます。

\textbf{回路の特徴:}

\begin{itemize}
\item 入力電圧 $V_{\text{in}} > $ 出力電圧 $V_{\text{out}}$
\item スイッチング素子(MOSFET)とダイオードを用いる
\item コイル(L)とコンデンサ(C)でフィルタを構成
\item デューティ比$D$により出力電圧を制御:$V_{\text{out}} = D \cdot V_{\text{in}}$
\end{itemize}

\subsection{降圧チョッパー回路の基本構成}

\begin{figure}[H]
\centering
\fbox{\includegraphics[width=0.95\textwidth]{chapters/chapter05/images/page-06.pdf}}
\caption{降圧チョッパー回路}
\label{fig:ch05_buck_converter}
\end{figure}

降圧チョッパー回路は、以下の素子で構成されます:

\begin{itemize}
\item \textbf{制御スイッチ}(SW):MOSFETなどのスイッチング素子
\item \textbf{ダイオード}(D):フリーホイーリングダイオード(還流ダイオード)
\item \textbf{コイル}(L):電流を平滑化するインダクタ
\item \textbf{負荷抵抗}(R):出力負荷
\end{itemize}

後の章で学習するコンデンサ(C)を出力に並列接続することで、出力電圧の平滑化も行います。

\subsection{降圧チョッパー回路の動作原理}

降圧チョッパー回路は、スイッチのオン・オフにより動作が切り替わります。以下、それぞれの状態について説明します。

\subsubsection{等価回路を作る}

DC-DC変換回路の動作を理解するためには、スイッチのオン時とオフ時の等価回路を作成することが重要です。

\begin{figure}[H]
\centering
\fbox{\includegraphics[width=0.95\textwidth]{chapters/chapter05/images/page-10.pdf}}
\caption{スイッチのオンオフ時にコイルにかかる電圧}
\label{fig:ch05_switch_voltage}
\end{figure}

\textbf{【オン時($0 \le t < DT$)】}

スイッチがオンの期間、電流は以下の経路を流れます:

\begin{center}
入力電源 $\rightarrow$ スイッチ $\rightarrow$ コイル $\rightarrow$ 負荷 $\rightarrow$ GND
\end{center}

\begin{figure}[H]
\centering
\fbox{\includegraphics[width=0.95\textwidth]{chapters/chapter05/images/page-07.pdf}}
\caption{降圧チョッパー回路:スイッチON時の等価回路}
\label{fig:ch05_buck_on}
\end{figure}

この時、キルヒホッフの電圧則(KVL)より:

\begin{equation}
V_{\text{in}} = v_L + v_R
\end{equation}

コイルにかかる電圧は:

\begin{equation}
v_L = V_{\text{in}} - v_R
\end{equation}

$v_R \ll V_{\text{in}}$の場合、近似的に:

\begin{equation}
v_L \approx V_{\text{in}} \quad (\text{正の電圧})
\end{equation}

したがって、オン時にはコイルにエネルギーが蓄積され、電流$i_L$が増加します。

\textbf{【オフ時($DT \le t < T$)】}

スイッチがオフの期間、コイルに蓄えられた磁気エネルギーにより、電流は以下の経路を流れます:

\begin{center}
GND $\rightarrow$ ダイオード $\rightarrow$ コイル $\rightarrow$ 負荷 $\rightarrow$ GND
\end{center}

\begin{figure}[H]
\centering
\fbox{\includegraphics[width=0.95\textwidth]{chapters/chapter05/images/page-08.pdf}}
\caption{降圧チョッパー回路:スイッチOFF時の等価回路}
\label{fig:ch05_buck_off}
\end{figure}

この時、キルヒホッフの電圧則(KVL)より:

\begin{equation}
0 = v_D + v_L + v_R
\end{equation}

ダイオードの順方向電圧降下を無視すると($v_D \approx 0$):

\begin{equation}
v_L = -v_R
\end{equation}

したがって、オフ時にはコイルに負の電圧が印加され、電流$i_L$が減少します。

\subsection{定常状態における波形}

\begin{figure}[H]
\centering
\fbox{\includegraphics[width=0.95\textwidth]{chapters/chapter05/images/page-09.pdf}}
\caption{降圧チョッパー回路:過渡状態}
\label{fig:ch05_buck_transient}
\end{figure}

回路をスタートさせると、最初は過渡状態を経て、やがて定常状態に達します。図は回路シミュレーション結果を示しています。

\begin{figure}[H]
\centering
\fbox{\includegraphics[width=0.95\textwidth]{chapters/chapter05/images/page-15.pdf}}
\caption{スイッチのオンオフ時の各素子の過渡応答}
\label{fig:ch05_buck_element_transient}
\end{figure}

過渡状態では、コイル電圧$v_L$、コイル電流$i_L$、負荷電圧$v_R$がそれぞれ変化していきます。

\begin{figure}[H]
\centering
\fbox{\includegraphics[width=0.95\textwidth]{chapters/chapter05/images/page-20.pdf}}
\caption{定常状態における降圧チョッパー回路の各波形}
\label{fig:ch05_buck_steady_waveforms}
\end{figure}

定常状態では、以下のような波形となります:

\begin{itemize}
\item \textbf{スイッチ電圧$v_{\text{sw}}$}:オン時は0V、オフ時は$V_{\text{in}}$
\item \textbf{コイル電圧$v_L$}:オン時は正、オフ時は負で、1周期の積分は0
\item \textbf{コイル電流$i_L$}:オン時は増加、オフ時は減少(三角波)
\item \textbf{負荷電圧$v_R$}:コイル電流に比例して変動(リプルを含む)
\end{itemize}

\textbf{重要な観察:}

図中の注釈にあるように、以下の点を理解することが重要です:

\begin{itemize}
\item \textcolor{red}{電流の振動(リプル)の振幅は?}
\item \textcolor{red}{平均電圧は?}
\end{itemize}

これらを求めるために、次節で定常状態の条件を用いて出力電圧を導出します。

\subsection{出力電圧とデューティ比の関係}

定常状態における出力電圧を求めるために、\textcolor{blue}{コイル電圧の電圧-秒バランス}を利用します。

定常状態では、1周期$T$におけるコイル電圧の積分がゼロになります:

\begin{equation}
\int_0^T v_L(t)dt = 0
\end{equation}

これを、オン期間($0 \sim DT$)とオフ期間($DT \sim T$)に分けて計算すると:

\begin{equation}
\int_0^{DT} v_L(t)dt + \int_{DT}^{T} v_L(t)dt = 0
\end{equation}

オン時の電圧を$v_L^{\text{ON}}$、オフ時の電圧を$v_L^{\text{OFF}}$とすると:

\begin{equation}
v_L^{\text{ON}} \cdot DT + v_L^{\text{OFF}} \cdot (1-D)T = 0
\end{equation}

前述の等価回路解析(式(10)および式(13))より、$v_R = V_{\text{out}}$として:
\begin{align}
v_L^{\text{ON}} &\approx V_{\text{in}} \\
v_L^{\text{OFF}} &= -v_R = -V_{\text{out}}
\end{align}

これらを代入すると:

\begin{equation}
V_{\text{in}} \cdot DT - V_{\text{out}} \cdot (1-D)T = 0
\end{equation}

整理すると:

\begin{equation}
V_{\text{in}} \cdot D = V_{\text{out}} \cdot (1-D)
\end{equation}

したがって、出力電圧は:

\begin{equation}
\boxed{V_{\text{out}} = D \cdot V_{\text{in}}}
\end{equation}

ここで、$D$はデューティ比($0 < D < 1$)です。

\textbf{結論:}

降圧チョッパー回路では、出力電圧は入力電圧にデューティ比を掛けた値となります。つまり、デューティ比を調整することで、$0$から$V_{\text{in}}$の範囲で出力電圧を制御できます。

\subsection{電流リプルの計算}

コイル電流の変動幅(リプル$\Delta i_L$)を求めます。

オン期間において、コイル電圧は:

\begin{equation}
v_L = L \frac{di_L}{dt} \approx V_{\text{in}} - V_{\text{out}}
\end{equation}

したがって、電流の増加率は:

\begin{equation}
\frac{di_L}{dt} = \frac{V_{\text{in}} - V_{\text{out}}}{L}
\end{equation}

オン期間$DT$における電流の増加量は:

\begin{equation}
\Delta i_L^{\text{up}} = \frac{V_{\text{in}} - V_{\text{out}}}{L} \cdot DT
\end{equation}

$V_{\text{out}} = D \cdot V_{\text{in}}$を代入すると:

\begin{equation}
\Delta i_L^{\text{up}} = \frac{V_{\text{in}}(1-D)}{L} \cdot DT = \frac{V_{\text{in}}D(1-D)T}{L}
\end{equation}

電流リプル(ピーク・ピーク値)は、この増加量の2倍です:

\begin{equation}
\boxed{\Delta i_L = \frac{V_{\text{in}}D(1-D)T}{L}}
\end{equation}

\textbf{リプルを小さくする方法:}

\begin{itemize}
\item インダクタンス$L$を大きくする
\item スイッチング周期$T$を小さくする(周波数を高くする)
\end{itemize}

\textbf{注意:} 数式上、$D(1-D)$は$D=0.5$で最大となるため、$D$を0.5から離すとリプルは減少します。しかし、デューティ比$D$は出力電圧$V_{\text{out}} = D \cdot V_{\text{in}}$を決定する重要なパラメータであるため、リプル低減のために$D$を変更することは適切ではありません。リプル低減には、$L$や$T$(スイッチング周波数)を調整すべきです。

\section{昇圧チョッパー回路(Boost Converter)}

\subsection{昇圧チョッパー回路とは}

昇圧チョッパー回路(Boost Converter)は、入力電圧$V_{\text{in}}$より高い出力電圧$V_{\text{out}}$を得るための回路です。

\begin{figure}[H]
\centering
\fbox{\includegraphics[width=0.95\textwidth]{chapters/chapter05/images/page-25.pdf}}
\caption{昇圧チョッパー回路}
\label{fig:ch05_boost_converter}
\end{figure}

\textbf{昇圧の原理:}

昇圧チョッパー回路は、以下の2つのステップでエネルギーを移送することで電圧を上げます:

\begin{enumerate}
\item \textbf{電源からコイルにエネルギーを蓄積}:スイッチON時
\item \textbf{コイルからコンデンサにエネルギーを蓄積}:スイッチOFF時
\end{enumerate}

コイルに蓄えられた磁気エネルギーを利用して、入力電圧より高い電圧を出力に供給します。

\subsection{昇圧チョッパー回路の基本構成}

\begin{figure}[H]
\centering
\fbox{\includegraphics[width=0.95\textwidth]{chapters/chapter05/images/page-26.pdf}}
\caption{昇圧チョッパー回路の回路構成}
\label{fig:ch05_boost_circuit}
\end{figure}

昇圧チョッパー回路は、以下の素子で構成されます:

\begin{itemize}
\item \textbf{コイル}(L):入力側に配置
\item \textbf{制御スイッチ}(SW):コイルとGND間に配置
\item \textbf{ダイオード}(D):コイルと出力の間に配置
\item \textbf{コンデンサ}(C):出力電圧を平滑化
\item \textbf{負荷抵抗}(R):出力負荷
\end{itemize}

降圧チョッパーと比較すると、コイルの位置とスイッチ・ダイオードの配置が異なります。

\subsection{昇圧チョッパー回路の動作原理}

\begin{figure}[H]
\centering
\fbox{\includegraphics[width=0.95\textwidth]{chapters/chapter05/images/page-30.pdf}}
\caption{スイッチのオンオフ時にコイルにかかる電圧}
\label{fig:ch05_boost_voltage}
\end{figure}

\textbf{【オン時($0 \le t < DT$)】}

スイッチがオンの期間、電流は以下の経路を流れます:

\begin{center}
入力電源 $\rightarrow$ コイル $\rightarrow$ スイッチ $\rightarrow$ GND
\end{center}

\begin{figure}[H]
\centering
\fbox{\includegraphics[width=0.95\textwidth]{chapters/chapter05/images/page-27.pdf}}
\caption{昇圧チョッパー回路:スイッチON時の等価回路}
\label{fig:ch05_boost_on}
\end{figure}

この時、キルヒホッフの電圧則(KVL)より:

\begin{equation}
V_0 - v_L = 0
\end{equation}

したがって、コイルにかかる電圧は:

\begin{equation}
v_L = V_0 \quad (\text{正の電圧})
\end{equation}

オン時には、入力電源からコイルにエネルギーが蓄積され、コイル電流が増加します。この時、ダイオードは逆バイアスとなり、出力側には電流が流れません。

\textbf{【オフ時($DT \le t < T$)】}

スイッチがオフの期間、コイルに蓄えられた磁気エネルギーにより、電流は以下の経路を流れます:

\begin{center}
入力電源 $\rightarrow$ コイル $\rightarrow$ ダイオード $\rightarrow$ コンデンサ・負荷 $\rightarrow$ GND
\end{center}

\begin{figure}[H]
\centering
\fbox{\includegraphics[width=0.95\textwidth]{chapters/chapter05/images/page-28.pdf}}
\caption{昇圧チョッパー回路:スイッチOFF時の等価回路}
\label{fig:ch05_boost_off}
\end{figure}

この時、キルヒホッフの電圧則(KVL)より:

\begin{equation}
V_0 - v_L - v_C = 0
\end{equation}

したがって、コイルにかかる電圧は:

\begin{equation}
v_L = V_0 - v_C \quad (\text{負の電圧})
\end{equation}

ここで、$v_C$は出力電圧$V_{\text{out}}$に相当します。

オフ時には、コイルに蓄えられたエネルギーが出力側に放出され、コイル電流が減少します。

\subsection{出力電圧とデューティ比の関係}

定常状態における出力電圧を求めるために、降圧チョッパーと同様に電圧-秒バランスを利用します。

定常状態では:

\begin{equation}
\int_0^T v_L(t)dt = 0
\end{equation}

オン期間とオフ期間に分けて:

\begin{equation}
v_L^{\text{ON}} \cdot DT + v_L^{\text{OFF}} \cdot (1-D)T = 0
\end{equation}

前述の等価回路解析(式(26)および式(28))より、$v_C = V_{\text{out}}$として:
\begin{align}
v_L^{\text{ON}} &= V_0 \\
v_L^{\text{OFF}} &= V_0 - V_{\text{out}}
\end{align}

これらを代入すると:

\begin{equation}
V_0 \cdot DT + (V_0 - V_{\text{out}}) \cdot (1-D)T = 0
\end{equation}

展開すると:

\begin{equation}
V_0 \cdot DT + V_0 \cdot (1-D)T - V_{\text{out}} \cdot (1-D)T = 0
\end{equation}

\begin{equation}
V_0 \cdot T = V_{\text{out}} \cdot (1-D)T
\end{equation}

したがって、出力電圧は:

\begin{equation}
\boxed{V_{\text{out}} = \frac{V_0}{1-D}}
\end{equation}

\textbf{結論:}

昇圧チョッパー回路では、出力電圧は入力電圧を$(1-D)$で割った値となります。$D$が1に近づくほど、出力電圧は高くなります。

例えば、$D = 0.5$の場合、$V_{\text{out}} = 2V_0$となり、入力電圧の2倍の出力が得られます。

\subsection{定常状態における波形}

\begin{figure}[H]
\centering
\fbox{\includegraphics[width=0.95\textwidth]{chapters/chapter05/images/page-31.pdf}}
\caption{昇圧チョッパー回路:定常状態の波形}
\label{fig:ch05_boost_waveforms}
\end{figure}

定常状態では、以下のような波形となります:

\begin{itemize}
\item \textbf{コイル電圧$v_L$}:オン時は$V_0$、オフ時は$V_0 - V_{\text{out}}$(負)
\item \textbf{コイル電流$i_L$}:オン時は増加、オフ時は減少(三角波)
\item \textbf{出力電圧$v_C$}:ほぼ一定(コンデンサによる平滑化)
\end{itemize}

\section{昇降圧チョッパー回路(Buck-Boost Converter)}

\subsection{昇降圧チョッパー回路とは}

昇降圧チョッパー回路(Buck-Boost Converter)は、入力電圧$V_{\text{in}}$より高い電圧も低い電圧も出力できる回路です。つまり、昇圧と降圧の両方の機能を持ちます。

\textbf{回路の特徴:}

\begin{itemize}
\item 入力電圧に対して、出力電圧を昇圧または降圧できる
\item 出力電圧の極性が反転する(負の電圧が出力される)
\item デューティ比$D$により出力電圧を制御
\end{itemize}

\subsection{昇降圧チョッパー回路の基本構成}

\begin{figure}[H]
\centering
\fbox{\includegraphics[width=0.95\textwidth]{chapters/chapter05/images/page-35.pdf}}
\caption{昇降圧チョッパー回路の基本構成}
\label{fig:ch05_buckboost_circuit}
\end{figure}

昇降圧チョッパー回路は、以下の素子で構成されます:

\begin{itemize}
\item \textbf{制御スイッチ}(SW):入力電源とコイルの間に配置
\item \textbf{コイル}(L):エネルギー蓄積素子
\item \textbf{ダイオード}(D):コイルとGND間に配置
\item \textbf{コンデンサ}(C):出力電圧を平滑化
\item \textbf{負荷抵抗}(R):出力負荷
\end{itemize}

\textbf{回路の動作を確認する手順:}

\begin{enumerate}
\item 制御スイッチがオン時とオフ時の等価回路を作る
\item オン時とオフ時のコイルにかかる電圧$v_L$を求める
\item 定常状態の条件(オンとオフ時の電圧の積分が等しい)から負荷電圧を求める
\end{enumerate}

\subsection{昇降圧チョッパー回路の動作原理}

\textbf{【オン時($0 \le t < DT$)】}

スイッチがオンの期間、電流は以下の経路を流れます:

\begin{center}
入力電源 $\rightarrow$ スイッチ $\rightarrow$ コイル $\rightarrow$ GND
\end{center}

\begin{figure}[H]
\centering
\fbox{\includegraphics[width=0.95\textwidth]{chapters/chapter05/images/page-36.pdf}}
\caption{昇降圧チョッパー回路:スイッチON時の等価回路}
\label{fig:ch05_buckboost_on}
\end{figure}

この時、コイルにかかる電圧は:

\begin{equation}
v_L = V_{\text{in}} \quad (\text{正の電圧})
\end{equation}

オン時には、入力電源からコイルにエネルギーが蓄積されます。

\textbf{【オフ時($DT \le t < T$)】}

スイッチがオフの期間、コイルに蓄えられたエネルギーが出力側に放出されます:

\begin{center}
GND $\rightarrow$ ダイオード $\rightarrow$ コイル $\rightarrow$ GND(コンデンサ・負荷経由)
\end{center}

\begin{figure}[H]
\centering
\fbox{\includegraphics[width=0.95\textwidth]{chapters/chapter05/images/page-37.pdf}}
\caption{昇降圧チョッパー回路:スイッチOFF時の等価回路}
\label{fig:ch05_buckboost_off}
\end{figure}

この時、コイルにかかる電圧は:

\begin{equation}
v_L = -V_{\text{out}} \quad (\text{負の電圧})
\end{equation}

\subsection{出力電圧とデューティ比の関係}

定常状態における電圧-秒バランスより:

\begin{equation}
v_L^{\text{ON}} \cdot DT + v_L^{\text{OFF}} \cdot (1-D)T = 0
\end{equation}

\begin{equation}
V_{\text{in}} \cdot DT - V_{\text{out}} \cdot (1-D)T = 0
\end{equation}

したがって、出力電圧は:

\begin{equation}
\boxed{V_{\text{out}} = \frac{D}{1-D} \cdot V_{\text{in}}}
\end{equation}

\textbf{結論:}

昇降圧チョッパー回路では:

\begin{itemize}
\item $D < 0.5$の場合:$V_{\text{out}} < V_{\text{in}}$(降圧動作)
\item $D = 0.5$の場合:$V_{\text{out}} = V_{\text{in}}$
\item $D > 0.5$の場合:$V_{\text{out}} > V_{\text{in}}$(昇圧動作)
\end{itemize}

ただし、出力電圧の極性は入力電圧と逆になります(負の電圧)。

\subsection{定常状態における波形}

\begin{figure}[H]
\centering
\fbox{\includegraphics[width=0.95\textwidth]{chapters/chapter05/images/page-40.pdf}}
\caption{定常状態時の出力電圧とスイッチング電圧の関係}
\label{fig:ch05_buckboost_waveforms}
\end{figure}

定常状態では、降圧チョッパーや昇圧チョッパーと同様に、コイル電圧は周期的にプラスとマイナスを繰り返し、コイル電流は三角波状に変動します。

\section{3つのチョッパー回路の比較}

\begin{table}[H]
\centering
\caption{3つのチョッパー回路の比較}
\begin{tabular}{|l|c|c|c|}
\hline
\textbf{回路名} & \textbf{出力電圧} & \textbf{電圧範囲} & \textbf{極性} \\
\hline
降圧チョッパー & $D \cdot V_{\text{in}}$ & $0 < V_{\text{out}} < V_{\text{in}}$ & 同じ \\
\hline
昇圧チョッパー & $\frac{V_{\text{in}}}{1-D}$ & $V_{\text{in}} < V_{\text{out}} < \infty$ & 同じ \\
\hline
昇降圧チョッパー & $\frac{D}{1-D} \cdot V_{\text{in}}$ & $0 < V_{\text{out}} < \infty$ & 反転 \\
\hline
\end{tabular}
\end{table}

\textbf{用途に応じた選択:}

\begin{itemize}
\item \textbf{降圧チョッパー}:バッテリー駆動機器、CPUの電源など
\item \textbf{昇圧チョッパー}:LED駆動回路、フラッシュカメラの充電回路など
\item \textbf{昇降圧チョッパー}:バッテリー電圧が変動する機器、極性反転が必要な回路など
\end{itemize}

\section{まとめ}

\begin{figure}[H]
\centering
\fbox{\includegraphics[width=0.95\textwidth]{chapters/chapter05/images/page-44.pdf}}
\caption{まとめ}
\label{fig:ch05_summary}
\end{figure}

本章では、DC-DC変換の基本である3つのチョッパー回路について学習しました。

\textbf{主な学習内容:}

\begin{itemize}
\item 降圧・昇圧・昇降圧チョッパー回路について説明した
\item 各回路の構成とそれぞれの役割について説明した
\item 定常状態のスイッチのオンオフ時における各素子の電圧や電流波形について説明した
\end{itemize}

\subsection{重要なポイント}

\begin{enumerate}
\item \textbf{等価回路の作成}:スイッチのオン時とオフ時の等価回路を描くことで、回路の動作が理解できる
\item \textbf{電圧-秒バランス}:定常状態では、コイル電圧の1周期の積分が0になる条件から、出力電圧を導出できる
\item \textbf{デューティ比制御}:デューティ比$D$を調整することで、出力電圧を制御できる
\item \textbf{連続電流モード}:本章では、コイル電流が常に正(連続)である連続電流モード(CCM: Continuous Conduction Mode)を扱った
\end{enumerate}

\subsection{設計上の考慮事項}

実際のDC-DC変換回路を設計する際には、以下の点を考慮する必要があります:

\begin{itemize}
\item \textbf{インダクタンス$L$の選定}:電流リプルと応答速度のトレードオフ
\item \textbf{キャパシタンス$C$の選定}:出力電圧リプルの低減
\item \textbf{スイッチング周波数$f_{\text{sw}}$}:素子サイズと損失のトレードオフ
\item \textbf{スイッチング素子の選定}:電圧・電流定格、オン抵抗、スイッチング速度
\item \textbf{ダイオードの選定}:逆回復時間、順方向電圧降下
\end{itemize}

\subsection{次回の予告}

次回(第6章)では、以下の内容について学習します:

\begin{itemize}
\item 絶縁型DC-DCコンバータ(フライバックコンバータ、フォワードコンバータ)
\item トランスを用いた昇圧・降圧
\item 入出力の絶縁による安全性向上
\end{itemize}

% ...

%========================================
% 後付け
%========================================
\backmatter

% 付録(必要に応じて追加)
% \appendix
% \include{appendix/appendixA}

% 参考文献(必要に応じて追加)
% \begin{thebibliography}{99}
% \bibitem{ref1} 著者名, ``書籍名'', 出版社, 発行年.
% \end{thebibliography}

% 索引(必要に応じて追加)
% \printindex

\end{document}
